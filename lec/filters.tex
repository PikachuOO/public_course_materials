% https://groups.google.com/forum/#!topic/comp.text.tex/s6z9Ult_zds
\makeatletter\let\ifGm@compatii\relax\makeatother

\ifx\notes\undefined
    \documentclass{beamer}
 \else
   \documentclass[handout]{beamer}
    \mode<handout>{
        \usepackage{pgfpages}
    \pgfpagesuselayout{4 on 1}[a4paper ,border shrink=2mm, landscape]
	\pgfpageslogicalpageoptions{1}{border code=\pgfsetlinewidth{1mm}\pgfusepath{stroke}}
	\pgfpageslogicalpageoptions{2}{border code=\pgfsetlinewidth{1mm}\pgfusepath{stroke}}
	\pgfpageslogicalpageoptions{3}{border code=\pgfsetlinewidth{1mm}\pgfusepath{stroke}}
	\pgfpageslogicalpageoptions{4}{border code=\pgfsetlinewidth{1mm}\pgfusepath{stroke}}
%    \setbeamercolor{background canvas}{bg=black!3}
    }
\fi
%http://www.cpt.univ-mrs.fr/~masson/latex/Beamer-appearance-cheat-sheet.pdf
\definecolor{Dandelion}         {RGB}{253, 200, 47}
\setbeamertemplate{frametitle}
{\vspace{1mm}
\begin{centering}
\insertframetitle
\end{centering}
{\color{Dandelion} \rule{\textwidth}{1mm}}
}
\setbeamertemplate{itemize item}{$\bullet$}
\setbeamertemplate{navigation symbols}{}
%\setbeamercolor{title}{fg=Fern}
%\setbeamercolor{frametitle}{fg=Fern}
%\setbeamercolor{normal text}{fg=Charcoal}
%\setbeamercolor{block title}{fg=black,bg=Fern!25!white}
%\setbeamercolor{block body}{fg=black,bg=Fern!25!white}
%\setbeamercolor{alerted text}{fg=AlertColor}
\setbeamercolor{itemize item}{fg=black}

\usepackage[latin1]{inputenc}
\usepackage{epic,eepic,epsf}
%\usepackage{latexsym}
\usepackage{verbatim}
\usepackage{listings}
\usepackage{alltt}
\usepackage{minted}

%% all code examples use "code" environment
\newenvironment{code}{\begin{minted}{c}}{\end{minted}}

%\newcommand{\UnderscoreCommands}{\do\verbatiminput%
%\do\citeNP \do\citeA \do\citeANP \do\citeN \do\shortcite%
%\do\shortciteNP \do\shortciteA \do\shortciteANP \do\shortciteN%
%\do\citeyear \do\citeyearNP%
%}
%\usepackage[strings]{underscore}
%\usetheme{Pittsburgh}

%\makeatletter
%\setbeamercolor{background canvas}{bg=white}
%\setbeamercolor{background}{bg=white}
%\makeatother

\title[COMP1511 Introduction to Programming]{COMP1511}
\author{Andrew Taylor/John Shepherd}
\institute{CSE, UNSW}
\date{\today}

\newcommand{\red}[1]{\textcolor{red}{#1}}
\newcommand{\blue}[1]{\textcolor{blue}{#1}}
\newcommand<>{\RedNote}[2]{%
  \begin{alertblock}#3{#1}
    #2
  \end{alertblock}}
\newcommand<>{\Note}[2]{%
  \begin{exampleblock}#3{#1}
    #2
  \end{exampleblock}}
\newcommand{\NotesFrame}{\frame<handout>[plain]{\frametitle{My Notes}}}
\newcommand{\ttblue}[1]{\texttt{\blue{#1}}}

\usepackage{epstopdf}


% prefer pdf over png if both versions exist
\DeclareGraphicsExtensions{%
    .pdf,.PDF,%
    .png,.PNG,%
    .jpg,.mps,.jpeg,.jbig2,.jb2,.JPG,.JPEG,.JBIG2,.JB2}

\setbeamercolor{block body}{bg=gray!7}

\usemintedstyle{tango}

\newenvironment{C}
 {\VerbatimEnvironment
  \setbeamertemplate{blocks}[rounded][shadow=true]
  \begin{block}{}
  \begin{minted}{c}}
 {\end{minted}
  \end{block}}
  
\newenvironment{sh}
 {\VerbatimEnvironment
  \setbeamertemplate{blocks}[rounded][shadow=true]
  \begin{block}{}
  \begin{minted}{sh}}
 {\end{minted}
  \end{block}}
  
\newenvironment{perl}
 {\VerbatimEnvironment
  \setbeamertemplate{blocks}[rounded][shadow=true]
  \begin{block}{}
  \begin{minted}{perl}}
 {\end{minted}
  \end{block}}
  
\newenvironment{python}
 {\VerbatimEnvironment
  \setbeamertemplate{blocks}[rounded][shadow=true]
  \begin{block}{}
  \begin{minted}{python}}
 {\end{minted}
  \end{block}}
  
\newenvironment{html}
 {\VerbatimEnvironment
  \setbeamertemplate{blocks}[rounded][shadow=true]
  \begin{block}{}
  \begin{minted}{html}}
 {\end{minted}
  \end{block}}
  
\newenvironment{txt}
 {\VerbatimEnvironment
  \setbeamertemplate{blocks}[rounded][shadow=true]
  \begin{block}{}
  \begin{minted}{text}}
 {\end{minted}
  \end{block}}

\begin{document}
\section{Filters and Regular Expressions}
\begin{frame}
\frametitle{Unix Processes}
A Unix process executes in this environment

	\begin{figure}
    	\centering
    	\includegraphics[width = 0.9\textwidth]{Pic/unixproc}
    	% \caption{Caption}
  	\end{figure}
\end{frame}

\begin{frame}
\frametitle{Unix Processes: C programmer's view}
Components of process environment {\small (C programmer's view)}:
\begin{itemize}
\item  \textbf{\tt{char *argv[]}} - command line arguments
\item  \textbf{\tt{int argc}} - size of \textbf{\tt{argv[]}}
\item  \textbf{\tt{char *env[]}} - \textbf{\tt{name-value}} pairs from parent process
\item  \textbf{\tt{FILE *stdin}} - input byte-stream, e.g. \textbf{\tt{getchar()}}
\item  \textbf{\tt{FILE *stdout}} - output byte-stream, e.g. \textbf{\tt{putchar()}}
\item  \textbf{\tt{FILE *stderr}} - output byte-stream, e.g. \textbf{\tt{fputc(c, stderr)}}
\item  \textbf{\tt{exit(int)}} - terminate program, set exit status 
\item  \textbf{\tt{return int}} - terminate \textbf{\tt{main()}}, set exit status
\end{itemize}
\end{frame}


\begin{frame}[fragile]
\frametitle{Output a file: C Code}
\begin{C}
// write bytes of stream to stdout
void process_stream(FILE *in) {
    while (1) {
        int ch = fgetc(in);
        if (ch == EOF)
             break;
        if (fputc(ch, stdout) == EOF) {
            fprintf(stderr, "cat:");
            perror("");
            exit(1);
        }
    }
}
\end{C}
\end{frame}

\begin{frame}[fragile]
\frametitle{Process files/stdin: C Code}
\begin{small}
\begin{C}
// process files given as arguments
// if no arguments process stdin
int main(int argc, char *argv[]) {
    if (argc == 1)
        process_stream(stdin);
    else
        for (int i = 1; i < argc; i++) {
            FILE *in = fopen(argv[i], "r");
            if (in == NULL) {
                fprintf(stderr, "cat: %s: ", argv[i]);
                perror("");
                return 1;
            }
            process_stream(in);
            fclose(in);
        }
    return 0;
}
\end{C}
\end{small}
\end{frame}

\begin{frame}[fragile]
\frametitle{Count Lines, Words, Chars: C Code}
\begin{C}
// count lines, words, chars in stream 
void count_file(FILE *in) {
    int n_lines = 0, n_words = 0, n_chars = 0;
    int in_word = 0, c;
    while ((c = fgetc(in)) != EOF) {
        n_chars++;
        if (c == '\n')
            n_lines++;
        if (isspace(c))
            in_word = 0;
        else if (!in_word) {
            in_word = 1;
            n_words++;
        }
    }
    printf("%6d %6d %6d", n_lines, n_words, n_chars);
}
\end{C}
\end{frame}

\begin{frame}
\frametitle{What is a filter?}
{\em{Filter}}: a program that transforms a data stream.

On Unix, filters are commands that:
\begin{itemize}
\item  read text from their standard input or specified files
\item  perform useful transformations on the text stream
\item  write the transformed text to their standard output
\end{itemize}
\end{frame}

\begin{frame}
\frametitle{What is a filter?}
Example: ~ \textbf{\tt{cat MyProg.c}}
\begin{itemize}
\item  reads the text of the program in the file \textbf{\tt{MyProg.c}}
\item  writes the (untransformed) text to standard output (i.e. the screen)
\end{itemize}
Example: ~ \textbf{\tt{cat {\textless}MyProg.c}}
\begin{itemize}
\item  the shell (command interpreter)
connects the file \textbf{\tt{MyProg.c}} to standard input of \textbf{\tt{cat}}
\item  \textbf{\tt{cat}} reads its standard input
\item  writes the (untransformed) text to standard output (i.e. the screen)
\end{itemize}
\end{frame}

\begin{frame}
\frametitle{Using Filters}
Shell I/O redirection can be used to specify filter source and destination:

	\begin{figure}
    	\centering
    	\includegraphics[width = 0.9\textwidth]{Pic/filter1}
    	% \caption{Caption}
  	\end{figure}
Alternatively, most filters allow multiple sources to be specified:

	\begin{figure}
    	\centering
    	\includegraphics[width = 0.9\textwidth]{Pic/filter2}
    	% \caption{Caption}
  	\end{figure}
\end{frame}

\begin{frame}
\frametitle{Using Filters}
In isolation, filters are reasonably useful

In combination, they provide a very powerful problem-solving toolkit.

Filters are normally used in combination via a pipeline:

	\begin{figure}
    	\centering
    	\includegraphics[width = 0.9\textwidth]{Pic/filter3}
    	% \caption{Caption}
  	\end{figure}
Note: similar style of problem-solving to function composition.
\end{frame}

\begin{frame}[shrink,fragile]
\frametitle{Using Filters}
Unix filters use common conventions for command line arguments:
\begin{itemize}
\item  input can be specified by a list of file names
\item  if no files are mentioned, the filter reads from standard input
        {\small (which may have been connected to a file)}
\item  the filename ``{\bf{\textbf{\tt{-}}}}'' corresponds to standard input
\end{itemize}
Examples:
\begin{verbatim}
    # read from the file data1
    filter data1
    # or
    filter < data1
    
    # read from the files data1 data2 data3
    filter data1 data2 data3
    
    # read from data1, then stdin, then data2
    filter data1 - data2
\end{verbatim}

If filter doesn't cope with named sources, use \textbf{\tt{cat}}
at the start of the pipeline
\end{frame}

\begin{frame}[fragile]
\frametitle{Using Filters}
Filters normally perform multiple variations on a task.

Selection of the variation is accomplished via command-line options:
\begin{itemize}
\item  options are introduced by a {\bf{\textbf{\tt{-}}}} ("minus" or "dash")
\item  options have a "short" form, {\bf{\textbf{\tt{-}}}} followed by a single letter
        ~ {\small (e.g. \textbf{\tt{-v}})}
\item  options have a "long" form, {\bf{\textbf{\tt{--}}}} followed by a word
        ~ {\small (e.g. \textbf{\tt{--verbose}})}
\item  short form options can usually be combined
        ~ {\small (e.g. \textbf{\tt{-av}} ~{\small vs}~ \textbf{\tt{-a -v}})}
\item  \textbf{\tt{--help}} (or \textbf{\tt{-?}}) often gives a list of all command-line options
\end{itemize}
\end{frame}

\begin{frame}
\frametitle{Filters: Option}

Most filters have {\it{many}} options for controlling their behaviour.

Unix {\bf{man}}ual entries describe how each option works.

To find what filters are available: \textbf{\tt{man -k}} {\it{keyword}}

The solution to all your problems: {\em{{\bf{RTFM}}}}
\end{frame}

\begin{frame}[fragile]
\frametitle{Delimited Input}
Many filters are able to work with text data formatted as {\em{fields}}
(columns in spreadsheet terms).

Such filters typically have an option for specifying the delimiter or field separator.
\\
{\small (Unfortunately, they often make different assumptions about the default column separator)}

Example (tab-separated columns):
\begin{verbatim}
    John   99
    Anne   75
    Andrew 50
    Tim    95
    Arun   33
    Sowmya 76
\end{verbatim}

\end{frame}

\begin{frame}[fragile]
\frametitle{Delimited Input}
Example (verticalbar-separated columns, enrolment file):
\begin{verbatim}
    COMP1011|2252424|Abbot, Andrew John    |3727|1|M
    COMP2011|2211222|Abdurjh, Saeed        |3640|2|M
    COMP1011|2250631|Accent, Aac-Ek-Murhg  |3640|1|M
    COMP1021|2250127|Addison, Blair        |3971|1|F
    COMP4012|2190705|Allen, David Peter    |3645|4|M
    COMP4910|2190705|Allen, David Pater    |3645|4|M
\end{verbatim}
Example (colon-separated columns, old Unix password file):
\begin{small}
\begin{verbatim}
root:ZHolHAHZw8As2:0:0:root:/root:/bin/bash
jas:nJz3ru5a/44Ko:100:100:John Shepherd:/home/jas:/bin/bash
cs1021:iZ3sO90O5eZY6:101:101:COMP1021:/home/cs1021:/bin/bash
cs2041:rX9KwSSPqkLyA:102:102:COMP2041:/home/cs2041:/bin/bash
cs3311:mLRiCIvmtI9O2:103:103:COMP3311:/home/cs3311:/bin/bash
\end{verbatim}
\end{small}
\end{frame}

\begin{frame}[shrink,fragile]
\frametitle{\textbf{\tt{cat}}: the simplest filter}
The \textbf{\tt{cat}} command copies its input to output unchanged (identity filter).

When supplied a list of file names, it con{\bf{cat}}enates them onto stdout. 

Some options:


\begin{center}
\begin{tabular}{lll}

  \begin{minipage}{1cm}{\bf{\textbf{\tt{-n}}}} ~\end{minipage}
   & \begin{minipage}{18cm}{\bf{n}}umber output lines {\small (starting from 1)}~\end{minipage}
\\[1ex]

  \begin{minipage}{1cm}{\bf{\textbf{\tt{-s}}}} ~\end{minipage}
   & \begin{minipage}{18cm}{\bf{s}}queeze consecutive blank lines into single blank line~\end{minipage}
\\[1ex]

  \begin{minipage}{1cm}{\bf{\textbf{\tt{-v}}}} ~\end{minipage}
   & \begin{minipage}{18cm}display control-characters in {\bf{v}}isible form (e.g. \textbf{\tt{{\textasciicircum}C}})~\end{minipage}
\\[1ex]
\end{tabular}
\end{center}


The \textbf{\tt{tac}} command copies files, but reverses the order of lines.

\end{frame}

\begin{frame}[fragile,shrink]
\frametitle{\textbf{\tt{wc}}: word counter}
The \textbf{\tt{wc}} command is a summarizing filter.

Useful with other filters to count things.


\begin{center}
\begin{tabular}{lll}

  \begin{minipage}{1cm}{\bf{\textbf{\tt{-c}}}} ~\end{minipage}
   & \begin{minipage}{18cm}counts the number of {\bf{c}}haracters ~ {\small (incl. \textbf{\tt{\\n}})}~\end{minipage}
\\[1ex]

  \begin{minipage}{1cm}{\bf{\textbf{\tt{-w}}}} ~\end{minipage}
   & \begin{minipage}{18cm}counts the number of {\bf{w}}ords ~ {\small (non-white space)}~\end{minipage}
\\[1ex]

  \begin{minipage}{1cm}{\bf{\textbf{\tt{-l}}}} ~\end{minipage}
   & \begin{minipage}{18cm}counts the number of {\bf{l}}ines~\end{minipage}
\\[1ex]
\end{tabular}
\end{center}

{\small 
Some filters find counting so useful that they define
their own options for it (e.g. \textbf{\tt{grep -c}})
}

\end{frame}

\begin{frame}[fragile]
\frametitle{\textbf{\tt{tr}}: transliterate characters}
The \textbf{\tt{tr}} command converts text char-by-char according to a mapping.
\begin{verbatim}
    tr 'sourceChars' 'destChars' < dataFile
\end{verbatim}


Each input character from $sourceChars$ is
mapped to the corresponding character in $destChars$.

Example:
\begin{verbatim}
    tr 'abc' '123' < someText
\end{verbatim}

Has $sourceChars$=\textbf{\tt{'abc'}}, $destChars$=\textbf{\tt{'123'}},
        so: \\
        \textbf{\tt{a}} $\rightarrow$ \textbf{\tt{1}},
        \textbf{\tt{b}} $\rightarrow$ \textbf{\tt{2}},
        \textbf{\tt{c}} $\rightarrow$ \textbf{\tt{3}}

Note: \textbf{\tt{tr}} doesn't accept file name on command line.
\end{frame}

\begin{frame}
\frametitle{\textbf{\tt{tr}}: transliterate characters}
Characters that are not in $sourceChars$ are copied unchanged to output.

If there is no corresponding character (i.e. $destChars$ is shorter than $sourceChars$), then the last char in $destChars$ is used.

Shorthands are available for specifying character lists:

E.g. \textbf{\tt{'a-z'}} is equivalent to \textbf{\tt{'abcdefghijklmnopqrstuvwxyz'}}

Note: newlines will be modified if the mapping specification requires it.
\end{frame}

\begin{frame}
\frametitle{\textbf{\tt{tr}}: transliterate characters}
Some options:


\begin{center}
\begin{tabular}{lll}

  \begin{minipage}{1cm}{\bf{\textbf{\tt{-c}}}} ~\end{minipage}
   & \begin{minipage}{18cm}map all characters {\em{not}} occurring in $sourceChars$ {\small ({\bf{c}}omplement)}~\end{minipage}
\\[1ex]

  \begin{minipage}{1cm}{\bf{\textbf{\tt{-s}}}} ~\end{minipage}
   & \begin{minipage}{18cm}{\bf{s}}queeze adjacent repeated characters out (only copy the first)~\end{minipage}
\\[1ex]

  \begin{minipage}{1cm}{\bf{\textbf{\tt{-d}}}} ~\end{minipage}
   & \begin{minipage}{18cm}{\bf{d}}elete all characters in $sourceChars$ (no $destChars$)~\end{minipage}
\\[1ex]
\end{tabular}
\end{center}

\end{frame}

\begin{frame}[fragile]
\frametitle{\textbf{\tt{tr}}: transliterate characters}
Examples:
\begin{verbatim}
    # map all upper-case letters to lower-case equivalents
    tr 'A-Z' 'a-z' < text
    
    # simple encryption (a->b, b->c, ... z->a)
    tr 'a-zA-Z' 'b-zaB-ZA' < text
    
    # remove all digits from input
    tr -d '0-9' < text
    
    # break text file into individual words, one per line
    tr -cs 'a-zA-Z0-9' '\n' < text
\end{verbatim}

\end{frame}

\begin{frame}
\frametitle{\textbf{\tt{head}}/\textbf{\tt{tail}}: select lines}

\begin{itemize}
\item
\textbf{\tt{head}} prints the first $n$ (default 10) lines of input.

E.g. \textbf{\tt{head file}} prints first 10 lines of \textbf{\tt{file}}.

\textbf{\tt{-n}} option changes numbe rof lines head/tail prints.

\item

The \textbf{\tt{tail}} prints the last  $n$   lines of input.

E.g. \textbf{\tt{tail-n 30 file}} prints last 30 lines of \textbf{\tt{file}}.

\item
Combine \textbf{\tt{head}} and \textbf{\tt{tail}} to select a range of lines.

E.g. \textbf{\tt{head -n 100 | tail -n 20}} copies lines 81..100 to output.

\end{itemize}

With more than one file prefixes with name (see labs).
\end{frame}

\begin{frame}[shrink]
\frametitle{\textbf{\tt{egrep}}: select lines matching a pattern}
The \textbf{\tt{egrep}} command only copies to output those lines in the input
that match a specified pattern.

The pattern is supplied as a regular expression on the command line
{\small (and should be quoted using single-quotes)}.

Some options:


\begin{center}
\begin{tabular}{lll}

  \begin{minipage}{1cm}{\bf{\textbf{\tt{-i}}}} ~\end{minipage}
   & \begin{minipage}{18cm}ignore upper/lower-case difference in matching~\end{minipage}
\\[1ex]

  \begin{minipage}{1cm}{\bf{\textbf{\tt{-v}}}} ~\end{minipage}
   & \begin{minipage}{18cm}only display lines that {\em{do not}} match the pattern~\end{minipage}
\\[1ex]

  \begin{minipage}{1cm}{\bf{\textbf{\tt{-w}}}} ~\end{minipage}
   & \begin{minipage}{18cm}only match pattern if it makes a complete word~\end{minipage}
\\[1ex]
\end{tabular}
\end{center}


\end{frame}

\begin{frame}
\frametitle{The grep family}

\textbf{\tt{egrep}} is one of a group of related filters
using different kinds of pattern match:

\begin{itemize}
\item  \textbf{\tt{grep}} uses a limited form of POSIX regular expressions
(no \textbf{\tt{+ ? |}} or parentheses)
\item \textbf{\tt{egrep}} (extended grep) implements the full regex syntax
\item \textbf{\tt{fgrep}} finds any of several (maybe even thousands of)
fixed strings using an optimised algorithm.
\end{itemize}

{\small (The name \textbf{\tt{grep}} is an acronym for {\bf{G}}lobally search with {\bf{R}}egular {\bf{E}}xpressions and {\bf{P}}rint)}

\end{frame}


\begin{frame}
\frametitle{Regular Expressions}
A {\em{regular expression}} (regex) defines a
a set of strings.

A {\em{regular expression}}  usually thought of as a pattern

Specifies a possibly infinite set of strings.

They can be succinct and powerful.

Regular expressions libraries are available for most languages.

In the Unix environment:
\begin{itemize}
\item  a lot of data is available in plain text format
\item  many tools make use of regular expressions for searching
\item  effective use of regular expressions makes you more productive
\end{itemize}

{\small A POSIX standard for regular expressions defines
the "pattern language" used by many Unix tools.}
\end{frame}

\begin{frame}[shrink]
\frametitle{Regular expressions Basics}
Regular expressions  specify complex patterns concisely \& precisely.\\[2ex]

\begin{itemize}
\item  Default: a character matches itself.  E.g. \textbf{a} has no special meaning so it matches the character  \textbf{a}.

\item  Repetition: $p${\bf{\textbf{*}}} denotes zero or more repetitions of $p$.

\item  Alternation:  ~ $pattern_{1}$ {\bf{\textbf{\tt{|}}}} $pattern_{2}$ denotes the union of $pattern_{1}$ and {\bf{\textbf{\tt{|}}}} $pattern_{2}$.\\[2ex]  E.g. \textbf{\tt{perl|python|ruby}} matches any of the strings \textbf{\tt{perl}}, \textbf{\tt{python}} or \textbf{\tt{ruby}} \\[2ex]

\item  Parentheses are used for grouping e.g. $a$\textbf{(,$a$}\textbf{)*} denotes a comma separated list of $a$'s. 

\item  The special meanings of characters can be removed by escaping them with \textbf{\textbackslash}  e.g. \textbf{\textbackslash*} matches the \textbf{*} character anywhere in the input. 
\end{itemize}

The  above 5 special characters ()*|\textbackslash  are sufficient to express any regular expression
but many more features are present for convenience \& clarity.

\end{frame}

\begin{frame}[shrink]
\frametitle{Patterns for matching Single Characters}

\begin{itemize}
\item
The special pattern {\bf{.}} (dot) matches any single character.

\item
Square brackets provide convenient matching of any {\bf one} of a set of characters.

{\bf{\textbf{\tt{[}}}}$listOfCharacters${\bf{\textbf{\tt{]}}}}
matches any single character from the list of characters.  E.g. \textbf{\tt{[aeiou]}} matches any vowel.

\item
A shorthand is available for ranges of characters ~ {\bf{\textbf{\tt{[}}}}$first-last${\bf{\textbf{\tt{]}}}}

Examples: \textbf{\tt{[a-e]  [a-z]  [0-9]  [a-zA-Z]  [A-Za-z]  [a-zA-Z0-9] }}

\item
The matching can be inverted ~ {\bf{\textbf{\tt{[{\textasciicircum}}}}}$listOfCharacters${\bf{\textbf{\tt{]}}}}

E.g. \textbf{\tt{[{\textasciicircum}a-e]}} matches any character {\it{except}} one of the first five letters

\item
{\small Other characters  lose their special meaning inside bracket expressions.}
\end{itemize}
\end{frame}

\begin{frame}
\frametitle{Anchoring Matches}
We can insist that a pattern appears at the start or end of a string

\begin{itemize}
\item  the start of the line is denoted by {\bf{\textbf{\tt{{\textasciicircum}}}}} (uparrow)
E.g. \textbf{\tt{{\textasciicircum}[abc]}} matches either \textbf{\tt{a}} or \textbf{\tt{b}} or \textbf{\tt{c}} at the
start of a string.
\item  the end of the line is denoted by {\bf{\textbf{\tt{\$}}}} (dollar)
E.g. \textbf{\tt{cat\$}} matches \textbf{\tt{cat}} at the end of a string.
\end{itemize}
\end{frame}


\begin{frame}
\frametitle{Repetition}
We can specify repetitions of patterns
\begin{itemize}
\item  $p${\bf{\textbf{\tt{*}}}} denotes zero or more repetitions of $p$
\item  $p${\bf{\textbf{\tt{+}}}} denotes one or more repetitions of $p$
\item  $p${\bf{\textbf{\tt{?}}}} denotes zero or one occurence of $p$
\end{itemize}
E.g. \textbf{\tt{[0-9]+}} matches any sequence of digits (i.e. matches integers)

E.g. \textbf{\tt{[-'a-zA-Z]+}} matches any sequence of letters/hyphens/apostrophes \\
{\small (this pattern could be used to match words in a piece of English text, e.g. \textbf{\tt{it's}}, \textbf{\tt{John}}, ...)}

E.g. \textbf{\tt{[{\textasciicircum}X]*X}} matches any characters up to and including the first \textbf{\tt{X}}
\end{frame}

\begin{frame}[shrink]
\frametitle{Repetition}
If a pattern can match several parts of the input,
the first match is chosen.

Examples:

\begin{center}
\begin{tabular}{lll}

  \begin{minipage}{5cm}{\bf{Pattern}} ~\end{minipage}
   & \begin{minipage}{18cm}{\bf{Text}} (with match underlined)~\end{minipage}
\\[1ex]

  \begin{minipage}{5cm}\textbf{\tt{[0-9]+}}~\end{minipage}
   & \begin{minipage}{18cm}i={\em{1234}} ~ j=56789~\end{minipage}
\\[1ex]

  \begin{minipage}{5cm}\textbf{\tt{[ab]+}}~\end{minipage}
   & \begin{minipage}{18cm}{\em{aabbabababaaa}}cabba~\end{minipage}
\\[1ex]

  \begin{minipage}{5cm}\textbf{\tt{[+]+}}~\end{minipage}
   & \begin{minipage}{18cm}C{\em{++}} is a hack~\end{minipage}
\\[1ex]
\end{tabular}
\end{center}

\end{frame}

\begin{frame}[shrink]
\frametitle{Regular Expression Examples}

\begin{center}
\begin{tabular}{lll}

   \begin{minipage}{2cm}{\bf{Regex}} ~\end{minipage}
    & \begin{minipage}{9cm}{\bf{Matches}}~\end{minipage}
\\[1ex]

   \begin{minipage}{2cm}\textbf{\tt{abc}} ~\end{minipage}
    & \begin{minipage}{9cm}the string of letters \textbf{\tt{"abc"}} \\
   E.g. \textbf{\tt{abc}}~\end{minipage}
\\[1ex]

   \begin{minipage}{2cm}\textbf{\tt{a.c}} ~\end{minipage}
    & \begin{minipage}{9cm}strings of letters containing \textbf{\tt{'a'}} followed by \textbf{\tt{'c'}}
       with any single character in between\\
   E.g. \textbf{\tt{abc, aac, acc, aXc, a2c, ...}}~\end{minipage}
\\[1ex]

   \begin{minipage}{2cm}\textbf{\tt{ab*c}} ~\end{minipage}
    & \begin{minipage}{9cm}strings of letters containing \textbf{\tt{'a'}} followed by \textbf{\tt{'c'}}
       with any number of \textbf{\tt{'b'}} letters in between\\
   E.g. \textbf{\tt{ac, abc, abbc, abbbc, ...}}~\end{minipage}
\\[1ex]

   \begin{minipage}{2cm}\textbf{\tt{a|the}} ~\end{minipage}
    & \begin{minipage}{9cm}either the string \textbf{\tt{"a"}} or the string \textbf{\tt{"the"}} \\
   E.g. \textbf{\tt{a, the}}~\end{minipage}
\\[1ex]

   \begin{minipage}{2cm}\textbf{\tt{[a-z]}} ~\end{minipage}
    & \begin{minipage}{10cm}any single lower-case letter \\
   E.g. \textbf{\tt{a, b, c, ... z}}~\end{minipage}
\\[1ex]
\end{tabular}
\end{center}

\end{frame}

\begin{frame}[shrink]
\frametitle{\textbf{\tt{cut}}: vertical slice}
The \textbf{\tt{cut}} command prints selected parts of input lines.
\begin{itemize}
\item  can select fields (assumes tab-separated columnated input)
\item  can select a range of character positions
\end{itemize}
Some options:


\begin{center}
\begin{tabular}{lll}

  \begin{minipage}{5cm}{\bf{\textbf{\tt{-f}}$listOfCols$}} ~\end{minipage}
   & \begin{minipage}{18cm}print only the specified fields (tab-separated) on output~\end{minipage}
\\[1ex]

  \begin{minipage}{5cm}{\bf{\textbf{\tt{-c}}$listOfPos$}} ~\end{minipage}
   & \begin{minipage}{18cm}print only chars in the specified positions~\end{minipage}
\\[1ex] 

  \begin{minipage}{5cm}{\bf{\textbf{\tt{-d'}}$c$\textbf{\tt{'}}}} ~\end{minipage}
   & \begin{minipage}{18cm}use character $c$ as the field separator~\end{minipage} 
\\[1ex]
\end{tabular}
\end{center}


Lists are specified as ranges (e.g. \textbf{\tt{1-5}}) or comma-separated (e.g. \textbf{\tt{2,4,5}}).
\end{frame}

\begin{frame}[fragile,shrink]
\frametitle{\textbf{\tt{cut}}: vertical slice}
Examples:
\begin{verbatim}
    # print the first column
    cut -f1 data 
    
    # print the first three columns
    cut -f1-3 data
    
    # print the first and fourth columns
    cut -f1,4 data
    
    # print all columns after the third
    cut -f4- data
    
    # print the first three columns, if '|'-separated
    cut -d'|' -f1-3 data
    
    # print the first five chars on each line
    cut -c1-5 data
\end{verbatim}

{\small Unfortunately, there's no way to refer to "last column" without counting the columns.}
\end{frame}

\begin{frame}
\frametitle{\textbf{\tt{paste}}: combine files}
The \textbf{\tt{paste}} command displays several text files "in parallel" on output.

If the inputs are files \textbf{\tt{a}}, \textbf{\tt{b}}, \textbf{\tt{c}}
\begin{itemize}
\item  the first line of output is composed of the first lines of \textbf{\tt{a}}, \textbf{\tt{b}}, \textbf{\tt{c}}
\item  the second line of output is composed of the second lines of \textbf{\tt{a}}, \textbf{\tt{b}}, \textbf{\tt{c}}
\end{itemize}
Lines from each file are separated by a tab character or specified delimiter(s).

If files are different lengths, output has all lines from longest file,
with empty strings for missing lines.

Interleaves lines instead with \textbf{\tt{-s}} (serial) option.
\end{frame}

\begin{frame}[fragile]
\frametitle{\textbf{\tt{paste}}: combine files}
Example: using \textbf{\tt{paste}} to rebuild a file broken up by \textbf{\tt{cut}}.

\begin{verbatim}
    # assume "data" is a file with 3 tab-separated columns
    cut -f1 data > data1
    cut -f2 data > data2
    cut -f3 data > data3
    paste data1 data2 data3 > newdata
    # "newdata" should look the same as "data"
\end{verbatim}

\end{frame}

\begin{frame}
\frametitle{\textbf{\tt{sort}}: sort lines}
The \textbf{\tt{sort}} command copies input to output but ensures that the
output is arranged in some particular order of lines.

By default, sorting is based on the first characters in the line.

Other features of \textbf{\tt{sort}}:
\begin{itemize}
\item  understands that text data sometimes occurs in delimited fields. \\
(so, can also sort fields (columns) other than the first {\small (which is the default)})
\item  can distinguish numbers and sort appropriately
\item  can ignore punctuation or case differences
\item  can sort files "in place" as well as behaving like a filter
\item  capable of sorting {\it{very large}} files
\end{itemize}
\end{frame}

\begin{frame}[shrink]
\frametitle{\textbf{\tt{sort}}: sort lines}
Some options:


\begin{center}
\begin{tabular}{lll}

  \begin{minipage}{2cm}{\bf{\textbf{\tt{-r}}}} ~\end{minipage}
   & \begin{minipage}{9cm}sort in descending order ({\bf{r}}everse sort)~\end{minipage}
\\[1ex]

  \begin{minipage}{2cm}{\bf{\textbf{\tt{-n}}}} ~\end{minipage}
   & \begin{minipage}{9cm}sort numerically rather than lexicographically~\end{minipage}
\\[1ex]

  \begin{minipage}{2cm}{\bf{\textbf{\tt{-d}}}} ~\end{minipage}
   & \begin{minipage}{9cm}dictionary order: ignore non-letters and non-digits~\end{minipage}
\\[1ex]

  \begin{minipage}{2cm}{\bf{\textbf{\tt{-t'}}$c$\textbf{\tt{'}}}} ~\end{minipage}
   & \begin{minipage}{9cm}use character $c$ to separate columns (default: space)~\end{minipage}
\\[1ex]

  \begin{minipage}{2cm}{\bf{\textbf{\tt{-k}}$n$\textbf{\tt{'}}}} ~\end{minipage}
   & \begin{minipage}{9cm}sort on column $n$~\end{minipage}
\\[1ex]
\end{tabular}
\end{center}

{\small Note: the \textbf{\tt{' '}} around the separator char are usually not necessary,
but are useful to prevent the shell from mis-interpreting shell meta-characters
such as \textbf{\tt{'|'}}.}

{\small Hint: to specify TAB as the field delimiter with an interactive shell like bash, type CTRL-v before pressing the TAB key.}
\end{frame}

\begin{frame}[fragile]
\frametitle{\textbf{\tt{sort}}: sort lines}
Examples:
\begin{verbatim}
    # sort numbers in 3rd column in descending order
    sort -nr -k3 data
    
    # sort the password file based on user name
    sort -t: -k5 /etc/passwd
     
\end{verbatim}

\end{frame}

\begin{frame}[fragile,shrink]
\frametitle{\textbf{\tt{uniq}}: remove or count duplicates}
The \textbf{\tt{uniq}} command by default removes all but one copy of {\em{adjacent}} identical lines.

Some options:


\begin{center}
\begin{tabular}{lll}

  \begin{minipage}{2cm}{\bf{\textbf{\tt{-c}}}} ~\end{minipage}
   & \begin{minipage}{18cm}also print number of times each line is duplicated~\end{minipage}
\\[1ex]

  \begin{minipage}{2cm}{\bf{\textbf{\tt{-d}}}} ~\end{minipage}
   & \begin{minipage}{18cm}only print (one copy of) duplicated lines~\end{minipage}
\\[1ex]

  \begin{minipage}{2cm}{\bf{\textbf{\tt{-u}}}} ~\end{minipage}
   & \begin{minipage}{18cm}only print lines that occur uniquely (once only)~\end{minipage}
\\[1ex]
\end{tabular}
\end{center}


Surprisingly useful tool for summarising data,
typically after extraction by cut.
Always preceded by sort (why?).
\begin{verbatim}
    # extract first field, sort, and tally
    cut -f1 data  |  sort  |  uniq -c
\end{verbatim}

\end{frame}

\begin{frame}[shrink]
\frametitle{\textbf{\tt{join}}: database operator}

\textbf{\tt{join}} merges two files using the values in a field
in each file as a common key.

The key field can be in a different position in each file,
but the files must be ordered on that field.
The default key field is 1.

Some options:


\begin{center}
\begin{tabular}{lll}

  \begin{minipage}{1cm}{\bf{\textbf{\tt{-1}} {\it{k}}}} ~\end{minipage}
   & \begin{minipage}{18cm}key field in first file is {\it{k}}~\end{minipage}
\\[1ex]

  \begin{minipage}{1cm}{\bf{\textbf{\tt{-2}} {\it{k}}}} ~\end{minipage}
   & \begin{minipage}{18cm}key field in second file is {\it{k}}~\end{minipage}
\\[1ex]

  \begin{minipage}{1cm}{\bf{\textbf{\tt{-a}} {\it{N}}}} ~\end{minipage}
   & \begin{minipage}{18cm}print a line for each unpairable line in file {\it{N}} (1 or 2)~\end{minipage}
\\[1ex]

  \begin{minipage}{1cm}{\bf{\textbf{\tt{-i}}}} ~\end{minipage}
   & \begin{minipage}{18cm}ignore case~\end{minipage}
\\[1ex]

  \begin{minipage}{1cm}{\bf{\textbf{\tt{-t}} {\it{c}}}} ~\end{minipage}
   & \begin{minipage}{18cm}tab character is {\it{c}}~\end{minipage}
\\[1ex]
\end{tabular}
\end{center}

\end{frame}

\begin{frame}[fragile,shrink]
\frametitle{\textbf{\tt{join}}: database operator}
Given these two data files (tab-separated fields)
\begin{verbatim}
# data1:                            # data2:
Bugs Bunny      1953            Warners Bugs Bunny
Daffy Duck      1948            Warners Daffy Duck
Donald Duck     1939            Disney  Goofy
Goofy   1952                    Disney  Mickey Mouse
Mickey Mouse    1937            Pixar   Nemo
Nemo    2003
Road Runner     1949
\end{verbatim}
the command \textbf{\tt{join -t'   ' -2 2 -a 1 data1 data2}} gives
\begin{verbatim}
Bugs Bunny      1953    Warners
Daffy Duck      1948    Warners
Donald Duck     1939
Goofy   1952    Disney
Mickey Mouse    1937    Disney
Nemo    2003    Pixar
Road Runner     1949
\end{verbatim}
\end{frame}

\begin{frame}[fragile]
\frametitle{\textbf{\tt{sed}}: stream editor}
The \textbf{\tt{sed}} command provides the power of interactive-style
editing in ``filter-mode''.

Invocation:
\begin{verbatim}
sed -e 'EditCommands' DataFile
sed -f EditCommandFile DataFile
\end{verbatim}

How \textbf{\tt{sed}} works:
\begin{itemize}
\item  read each line of input
\item  check if it matches any patterns or line-ranges
\item  apply related editing commands to the line
\item  write the transformed line to output
\end{itemize}
\end{frame}

\begin{frame}
\frametitle{\textbf{\tt{sed}}: stream editor}
The editing commands are very powerful and subsume the actions of
many of the filters looked at so far.

In addition, \textbf{\tt{sed}} can:
\begin{itemize}
\item  partition lines based on patterns rather than columns
\item  extract ranges of lines based on patterns or line numbers
\end{itemize}
Option \textbf{\tt{-n}} ({\bf{n}}o printing):
\begin{itemize}
\item  applies all editing commands as normal
\item  displays no output, unless \textbf{\tt{p}} appended to edit command
\end{itemize}
\end{frame}

\begin{frame}[shrink]
\frametitle{\textbf{\tt{sed}}: stream editor}
Editing commands:

\begin{center}
\begin{tabular}{lll}

  \begin{minipage}{2cm}{\bf{\textbf{\tt{p}}}} ~\end{minipage}
   & \begin{minipage}{8cm}print the current line~\end{minipage}
\\[1ex]

  \begin{minipage}{2cm}{\bf{\textbf{\tt{d}}}} ~\end{minipage}
   & \begin{minipage}{8cm}delete (don't print) the current line~\end{minipage}
\\[1ex]

  \begin{minipage}{2cm}{\bf{\textbf{\tt{s/}}$RegExp$\textbf{\tt{/}}$Replace$\textbf{\tt{/}}}} ~\end{minipage} \\
   & \begin{minipage}{8cm}substitute first occurrence of string matching $RegExp$ by $Replace$ string~\end{minipage}
\\[1ex]

  \begin{minipage}{2cm}{\bf{\textbf{\tt{s/}}$RegExp$\textbf{\tt{/}}$Replace$\textbf{\tt{/g}}}} ~\end{minipage} \\
   & \begin{minipage}{8cm}substitute all occurrences of string matching $RegExp$ by $Replace$ string~\end{minipage}
\\[1ex]

  \begin{minipage}{2cm}{\bf{\textbf{\tt{q}}}} ~\end{minipage}
   & \begin{minipage}{8cm}terminate execution of \textbf{\tt{sed}}~\end{minipage}
\end{tabular}
\end{center}

\end{frame}

\begin{frame}[shrink]
\frametitle{\textbf{\tt{sed}}: stream editor}
All editing commands can be qualified by line addresses or
line selector patterns to limit lines where command is applied:

\begin{center}
\begin{tabular}{lll}

  \begin{minipage}{2cm}{\bf{$LineNo$}} ~\end{minipage}
   & \begin{minipage}{10cm}selects the specified line~\end{minipage}\\[1ex]
\\[1ex]

  \begin{minipage}{2cm}{\bf{$StartLineNo$\textbf{\tt{,}}$EndLineNo$}} ~\end{minipage} \\
   & \begin{minipage}{10cm}selects all lines between specified line numbers~\end{minipage}
\\[1ex]

  \begin{minipage}{2cm}{\bf{\textbf{\tt{/}}$RegExp$\textbf{\tt{/}}}} ~\end{minipage}
   & \begin{minipage}{10cm}selects all lines that match $RegExp$~\end{minipage}
\\[1ex]

  \begin{minipage}{2cm}{\bf{\textbf{\tt{/}}$RegExp1$\textbf{\tt{/,/}}$RegExp2$\textbf{\tt{/}}}} ~\end{minipage} \\
   & \begin{minipage}{10cm}selects all lines between lines matching reg exps~\end{minipage}
\\[1ex]
\end{tabular}
\end{center}

\end{frame}

\begin{frame}[fragile]
\frametitle{\textbf{\tt{sed}}: stream editor}
Examples:
\begin{verbatim}
# print all lines
sed -n -e 'p' < file

# print the first 10 lines
sed -e '10q' < file
sed -n -e '1,10p' < file

# print lines 81 to 100
sed -n -e '81,100p' < file

# print the last 10 lines of the file?
sed -n -e '$-10,$p' < file  # does NOT work
\end{verbatim}

\end{frame}

\begin{frame}[fragile]
\frametitle{\textbf{\tt{sed}}: stream editor}
More Examples:
\begin{verbatim}
# print only lines containing 'xyz'
sed -n -e '/xyz/p' < file

# print only lines NOT containing 'xyz'
sed -e '/xyz/d' < file

# show the passwd file, displaying only the
# lines from "root" up to "nobody" (i.e. system accounts)
sed -n -e '/^root/,/^nobody/p' /etc/passwd
 
# remove first column from ':'-separated file
sed -e 's/[^:]*://' datafile

# reverse the order of the first two columns
sed -e 's/\([^:]*\):\([^:]*\):\(.*\)$/\2:\1:\3/'
\end{verbatim}

\end{frame}

\begin{frame}[fragile]
\frametitle{\textbf{\tt{find}}: search for files}
The \textbf{\tt{find}} commands allows you to search for files based on
specified properties
~ {\small (a filter for the file system)}
\begin{itemize}
\item  searches an entire directory tree, testing each file for the required property.
\item  takes some action for all "matching" files
        {\small (usually just print the file name)}
\end{itemize}
Invocation:
\begin{verbatim}
    find StartDirectory Tests Actions
\end{verbatim}

where
\begin{itemize}
\item  the $Tests$ examine file properties like name, type, modification date
\item  the $Actions$ can be simply to print the name or execute an arbitrary command on the matched file
\end{itemize}
\end{frame}

\begin{frame}[fragile]
\frametitle{\textbf{\tt{find}}: search for files}
Examples:
\begin{verbatim}
# find all the HTML files below /home/jas/web
find  /home/jas/web  -name '*.html'  -print

# find all your files/dirs changed in the last 2 days
find  ~  -mtime -2  -print

# show info on files changed in the last 2 days
find  ~  -mtime -2  -type f  -exec ls -l {} \;

# show info on directories changed in the last week
find  ~  -mtime -7  -type d  -exec ls -ld {} \;

# find directories either new or with '07' in their name
find  ~  -type d  \(  -name '*07*'  -o  -mtime -1  \)  -print
\end{verbatim}

\end{frame}

\begin{frame}[fragile]
\frametitle{\textbf{\tt{find}}: search for files}
More Examples:
\begin{verbatim}
# find all {\it{new}} HTML files below /home/jas/web
find  /home/jas/web  -name '*.html'  -mtime -1  -print

# find background colours in my HTML files
find  ~/web  -name '*.html'  -exec grep -H 'bgcolor' {} \;

# above could also be accomplished via ...
grep  -r  'bgcolor'  ~/web

# make sure that all HTML files are accessible
find  ~/web  -name '*.html'  -exec chmod 644 {} \;

# remove any really old files ... Danger!
find  /hot/new/stuff  -type f  -mtime +364  -exec rm {} \;
find  /hot/new/stuff  -type f  -mtime +364  -ok rm {} \;
\end{verbatim}

\end{frame}

\begin{frame}[shrink]
\frametitle{Filter summary by type}
\begin{itemize}
\item {\em{Horizontal slicing}} - select subset of lines: \\
	~ ~ ~ \textbf{\tt{cat, head, tail, *grep, sed, uniq}}
\item {\em{Vertical slicing}} - select subset of columns:
	~ \textbf{\tt{cut}}, \textbf{\tt{sed}}
\item {\em{Substitution}}:
	~ \textbf{\tt{tr}}, \textbf{\tt{sed}} 
\item {\em{Aggregation, simple statistics}}:
	~ \textbf{\tt{wc}}, \textbf{\tt{uniq}} 
\item {\em{Assembly}} - combining data sources:
	~ \textbf{\tt{paste}}, \textbf{\tt{join}} 
\item {\em{Reordering}}:
	~ \textbf{\tt{sort}} 
\item {\em{Viewing}} (always end of pipeline):
	~ \textbf{\tt{more}}, \textbf{\tt{less}} 
\item {\em{File system filter}}:
	~ \textbf{\tt{find}} 
\item {\em{Programmable filters}}:
	~ \textbf{\tt{sed}}, (and \textbf{\tt{perl}})
\end{itemize}
\end{frame}

\end{document}
