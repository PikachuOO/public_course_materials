% https://groups.google.com/forum/#!topic/comp.text.tex/s6z9Ult_zds
\makeatletter\let\ifGm@compatii\relax\makeatother

\ifx\notes\undefined
    \documentclass{beamer}
 \else
   \documentclass[handout]{beamer}
    \mode<handout>{
        \usepackage{pgfpages}
    \pgfpagesuselayout{4 on 1}[a4paper ,border shrink=2mm, landscape]
	\pgfpageslogicalpageoptions{1}{border code=\pgfsetlinewidth{1mm}\pgfusepath{stroke}}
	\pgfpageslogicalpageoptions{2}{border code=\pgfsetlinewidth{1mm}\pgfusepath{stroke}}
	\pgfpageslogicalpageoptions{3}{border code=\pgfsetlinewidth{1mm}\pgfusepath{stroke}}
	\pgfpageslogicalpageoptions{4}{border code=\pgfsetlinewidth{1mm}\pgfusepath{stroke}}
%    \setbeamercolor{background canvas}{bg=black!3}
    }
\fi
%http://www.cpt.univ-mrs.fr/~masson/latex/Beamer-appearance-cheat-sheet.pdf
\definecolor{Dandelion}         {RGB}{253, 200, 47}
\setbeamertemplate{frametitle}
{\vspace{1mm}
\begin{centering}
\insertframetitle
\end{centering}
{\color{Dandelion} \rule{\textwidth}{1mm}}
}
\setbeamertemplate{itemize item}{$\bullet$}
\setbeamertemplate{navigation symbols}{}
%\setbeamercolor{title}{fg=Fern}
%\setbeamercolor{frametitle}{fg=Fern}
%\setbeamercolor{normal text}{fg=Charcoal}
%\setbeamercolor{block title}{fg=black,bg=Fern!25!white}
%\setbeamercolor{block body}{fg=black,bg=Fern!25!white}
%\setbeamercolor{alerted text}{fg=AlertColor}
\setbeamercolor{itemize item}{fg=black}

\usepackage[latin1]{inputenc}
\usepackage{epic,eepic,epsf}
%\usepackage{latexsym}
\usepackage{verbatim}
\usepackage{listings}
\usepackage{alltt}
\usepackage{minted}

%% all code examples use "code" environment
\newenvironment{code}{\begin{minted}{c}}{\end{minted}}

%\newcommand{\UnderscoreCommands}{\do\verbatiminput%
%\do\citeNP \do\citeA \do\citeANP \do\citeN \do\shortcite%
%\do\shortciteNP \do\shortciteA \do\shortciteANP \do\shortciteN%
%\do\citeyear \do\citeyearNP%
%}
%\usepackage[strings]{underscore}
%\usetheme{Pittsburgh}

%\makeatletter
%\setbeamercolor{background canvas}{bg=white}
%\setbeamercolor{background}{bg=white}
%\makeatother

\title[COMP1511 Introduction to Programming]{COMP1511}
\author{Andrew Taylor/John Shepherd}
\institute{CSE, UNSW}
\date{\today}

\newcommand{\red}[1]{\textcolor{red}{#1}}
\newcommand{\blue}[1]{\textcolor{blue}{#1}}
\newcommand<>{\RedNote}[2]{%
  \begin{alertblock}#3{#1}
    #2
  \end{alertblock}}
\newcommand<>{\Note}[2]{%
  \begin{exampleblock}#3{#1}
    #2
  \end{exampleblock}}
\newcommand{\NotesFrame}{\frame<handout>[plain]{\frametitle{My Notes}}}
\newcommand{\ttblue}[1]{\texttt{\blue{#1}}}

\usepackage{epstopdf}


% prefer pdf over png if both versions exist
\DeclareGraphicsExtensions{%
    .pdf,.PDF,%
    .png,.PNG,%
    .jpg,.mps,.jpeg,.jbig2,.jb2,.JPG,.JPEG,.JBIG2,.JB2}

\setbeamercolor{block body}{bg=gray!7}

\usemintedstyle{tango}

\newenvironment{C}
 {\VerbatimEnvironment
  \setbeamertemplate{blocks}[rounded][shadow=true]
  \begin{block}{}
  \begin{minted}{c}}
 {\end{minted}
  \end{block}}
  
\newenvironment{sh}
 {\VerbatimEnvironment
  \setbeamertemplate{blocks}[rounded][shadow=true]
  \begin{block}{}
  \begin{minted}{sh}}
 {\end{minted}
  \end{block}}
  
\newenvironment{perl}
 {\VerbatimEnvironment
  \setbeamertemplate{blocks}[rounded][shadow=true]
  \begin{block}{}
  \begin{minted}{perl}}
 {\end{minted}
  \end{block}}
  
\newenvironment{python}
 {\VerbatimEnvironment
  \setbeamertemplate{blocks}[rounded][shadow=true]
  \begin{block}{}
  \begin{minted}{python}}
 {\end{minted}
  \end{block}}
  
\newenvironment{html}
 {\VerbatimEnvironment
  \setbeamertemplate{blocks}[rounded][shadow=true]
  \begin{block}{}
  \begin{minted}{html}}
 {\end{minted}
  \end{block}}
  
\newenvironment{txt}
 {\VerbatimEnvironment
  \setbeamertemplate{blocks}[rounded][shadow=true]
  \begin{block}{}
  \begin{minted}{text}}
 {\end{minted}
  \end{block}}

\begin{document}
\section{Python for Perl Programmers}
\begin{frame}
\frametitle{Python in COMP[29]041}
We will introduce {\bf{briefly}} Python in lectures after Perl
and compare it to Perl.

May be a written exam question on Python.

Challenge part of lab exercises involves Python.

Opportunity for student coping {\bf{well}} with Perl to
teach themselves Python.

Assignments involve Python.
\end{frame}

\begin{frame}
\frametitle{Python}

Developed in the 1990s by Guido van Rossum at CWI (Netherlands) - was at Google - now at Dropbox.

Similar niche to Perl but very different design.

Much simpler syntax than Perl.

No variable interpolation in strings, use \% operator instead

Regular expressions used via functions, not embedded in syntax.

Indenting used to group statements - unlike Perl/C/Java/... 

Python has a more elaborate type system but more straightforward semantics.

Python (like C) does less implicit conversions than Perl, e.g. you have to
convert strings explictly to numbers.

Core python does not have arrays (numpy has arrays).

Python lists can be indexed but do not automatically grow.

\end{frame}

\begin{frame}
\frametitle{Python 2 or Python 3}

Python introduced in 2008 but many users stayed with Python 2 until recently.

Some users still use Python 2 - some useful packages not yet available for Python 3.

Mostly possible to write programs which work in both Python 2 and 3.

Most COMP[29]041 example code will work in both Python 2 and 3.

You may use either Python 2 or Python 3 in COMP[29]041.

\end{frame}

\begin{frame}[fragile]
\frametitle{Syntax Conventions}
Perl uses non-alphabetic characters such as \$, @, \% and \& to associate 
types with names. 

Python associates types with values.

\begin{center}
\begin{tabular}{ll}

{\bf Perl}                         & {\bf Python} \\

\$s = "string"                     & s = "string" \\
@a = (1,2,3)                       & a = [1,2,3] \\
\verb|%a = (1=>'a',2=>'b',3=>'c')| & \verb|a = {1:'a',2:'b',3:'c'}| \\
\$a[42] = 'answer'                 & a[42] = 'answer' \\
\$h\{'answer'\} = 42               & h['answer'] = 42 \\

\end{tabular}
\end{center}

\end{frame}

\begin{frame}[fragile]
\frametitle{Names and Types}

A Python name can be associated with any type.

The {\bf type} function allows introspection.

\begin{python}
>>> a = 42
>>> type(a)
<type 'int'>
>>> a = "String"
>>> type(a)
<type 'str'>
>>> a = [1,2,3]
>>> type(a)
<type 'list'>
>>> a = {'ps':50,'cr':65,'dn':75}
>>> type(a)
<type 'dict'>
\end{python}

\end{frame}

\begin{frame}[fragile]
\frametitle{Python Indexing}

Indexing is generalized in Python.

Python uses [] to access elements of both lists and dicts (hashes).

[] also used for other types (e.g. strings).

[] can be used for user-defined  class if it has {\bf \_\_getitem\_\_} method.

\end{frame}

\begin{frame}[fragile]
\frametitle{Initializing dicts}
In Perl you can assign a value to hash element without having
used the hash previously:

\begin{perl}
    $h{'answer'} = 42
\end{perl}

In Python you must first initialize the dict:
like this:
\begin{python}
    h = {}
    h['answer'] = 42
\end{python}
or you could do this:
\begin{python}
    h = {'answer':42}
\end{python}
\end{frame}

\begin{frame}[fragile]
\frametitle{Python lists}
Perl arrays grow automatically - Python lists do not.
This fails in Python:
\begin{python}
    h = []
    h[0] = 'zero'
\end{python}
This works:
\begin{python}
    h = []
    h.append('zero')
\end{python}
or this works:
\begin{python}
    h = ['zero']
\end{python}

A useful idiom to create a large list of constants is:
\begin{python}
    h = [0] * 100 
\end{python}
\end{frame}
\begin{frame}[fragile]
\frametitle{Variable Interpolation}

Python does not interpolate variables into strings.

There is no difference between single and double quotes.

Python has the \% operator - equivalent to sprintf.

So you can write:
\begin{python}
    print('%8d %s' % (count[word], word))
\end{python}

See also string.Template 
\end{frame}

\begin{frame}[fragile]
\frametitle{Regular expressions}

Regular expressions not built-in to Python syntax

Less concise than Perl but similar functionality.

\begin{perl}
    $x =~ s/a/b/;
\end{perl}

\begin{python}
    x = re.sub('a', 'b', x)
\end{python}

Python r-quotes useful for regex - if a string is prefixed with an r (for raw)
then backslashes have no special meaning (except before quotes).

In Python {\bf r'{\textbackslash}n'} is a 2 character string as is {\bf '{\textbackslash}n'} in Perl.

\end{frame}

\begin{frame}[fragile]
\frametitle{Defining Python Functions}
Python functions have named untyped parameters:

\begin{python}
    def square(x):
        return x*x
\end{python}

Function calls are C-style, e.g:

\begin{python}
    print(square(42))
\end{python}

Default argument values can be indicated:

\begin{python}
    def square(x=42):
            return x*x
\end{python}

Allowing the function to be optionally called without specifying the argument.

\begin{python}
    print(square())
\end{python}

\end{frame}

\begin{frame}[fragile]
\frametitle{Keyword Functions Arguments}

Function arguments can be specified by keyword.

This funcion:

\begin{python}
    def evaluate_poly(a=1, b=2, c=0, x=42):
        return a*x*x+b*x+c
\end{python}

Can be called like this:
         
\begin{python}
    print(evaluate_poly(x=7, c=4))
\end{python}

\end{frame}

\begin{frame}[fragile]
\frametitle{Accessing Function Arguments as List or Dict}

Can also have variable number of non-keywords arguments accesssed by a list.

And a variable number of non-keywords arguments accesssed by a dict.

\begin{python}
    def f(*arguments, **keywords):
        for arg in arguments:
            print(arg)
        for kw in keywords.keys():
            print(kw + " : " + keywords[kw])
\end{python}

Powerful but you don't need for COMP2041.

\end{frame}

\begin{frame}[fragile]
\frametitle{Python imports}

Python module - a file containing  function definitions and other Python.

Access modules using import statements,e.g.

{\bf import random} can access names from random as random.name, e.g {\bf random.choice}

{\bf from  random import shuffle} can access {\bf random.shuffle} as shuffle

{\bf from  random import *} can access all names from random without prefix

{\bf import random as r} access names from random as r.name, e.g {\bf r.choice}


\end{frame}
%%% 
%%% \begin{frame}
%%% \frametitle{Perl-Python Comparison - Devowelling}
%%% 
%%% Perl \href{http://www.cse.unsw.edu.au/~cs2041/tut/perl/devowel.pl}{\tiny http://www.cse.unsw.edu.au/~cs2041/tut/perl/devowel.pl}
%%% {\small \verbatiminput{/home/cs2041/public_html/tut/perl/devowel.pl}}
%%% 
%%% Python \href{http://www.cse.unsw.edu.au/~cs2041/tut/perl/devowel.py}{\tiny http://www.cse.unsw.edu.au/~cs2041/tut/perl/devowel.py}
%%% {\small \verbatiminput{/home/cs2041/public_html/tut/perl/devowel.py}}
%%% \end{frame}
%%% 
%%% \begin{frame}
%%% \frametitle{Perl-Python Comparison: Simple Cut}
%%% 
%%% Perl \href{http://www.cse.unsw.edu.au/~cs2041/tut/perl/simple_cut1.pl}{\tiny http://www.cse.unsw.edu.au/~cs2041/tut/perl/simple_cut1.pl}
%%% {\small \verbatiminput{/home/cs2041/public_html/tut/perl/simple_cut1.pl}}
%%% 
%%% \end{frame}
%%% 
%%% \begin{frame}
%%% \frametitle{Perl-Python Comparison: Simple Cut}
%%% 
%%% 
%%% Python \href{http://www.cse.unsw.edu.au/~cs2041//tut/perl/simple_cut1.py}{\tiny http://www.cse.unsw.edu.au/~cs2041//tut/perl/simple_cut1.py}
%%% {\small \verbatiminput{/home/cs2041/public_html/tut/perl/simple_cut1.py}}
%%% 
%%% \end{frame}

\begin{frame}[fragile]
\frametitle{Perl-Python Comparison: Counting First names}

%%% Perl \href{http://www.cse.unsw.edu.au/~cs2041/code/perl/count_first_names.pl}{\tiny http://www.cse.unsw.edu.au/~cs2041/code/perl/count_first_names.pl}
{\small
\begin{perl}
while ($line = <>) {
    @fields = split /\|/, $line;
    $student_number = $fields[1];
    next if $already_counted{$student_number};
    $already_counted{$student_number} = 1;
    $full_name = $fields[2];
    $full_name =~ /.*,\s+(\S+)/ or next;
    $fname = $1;
    $fn{$fname}++;
}

foreach $fname (sort keys %fn) {
    printf "%2d people with the first name $fname\n", $fn{$fname};
}
\end{perl}
}

\end{frame}

\begin{frame}[fragile]
\frametitle{Perl-Python Comparison: Counting First names}
%%% Python \href{http://www.cse.unsw.edu.au/~cs2041/code/python/count_first_names.py}{\tiny http://www.cse.unsw.edu.au/~cs2041/code/python/count_first_names.py}
{\small
\begin{python}
import fileinput, re
already_counted = {}; fn = {}
for line in fileinput.input():
    fields = line.split('|')
    student_number = fields[1]
    if student_number in already_counted: continue
    already_counted[student_number] = 1
    full_name = fields[2]
    m = re.match(r'.*,\s+(\S+)', full_name)
    if m:
        fname = m.group(1)
        if fname in fn:
            fn[fname] += 1
        else:
            fn[fname] = 1
for fname in sorted(fn.keys()):
    print("%2d people with first name %s"%(fn[fname], fname))
\end{python}
}
\end{frame}

\end{document}
