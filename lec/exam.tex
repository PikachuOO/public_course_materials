% https://groups.google.com/forum/#!topic/comp.text.tex/s6z9Ult_zds
\makeatletter\let\ifGm@compatii\relax\makeatother

\ifx\notes\undefined
    \documentclass{beamer}
 \else
   \documentclass[handout]{beamer}
    \mode<handout>{
        \usepackage{pgfpages}
    \pgfpagesuselayout{4 on 1}[a4paper ,border shrink=2mm, landscape]
	\pgfpageslogicalpageoptions{1}{border code=\pgfsetlinewidth{1mm}\pgfusepath{stroke}}
	\pgfpageslogicalpageoptions{2}{border code=\pgfsetlinewidth{1mm}\pgfusepath{stroke}}
	\pgfpageslogicalpageoptions{3}{border code=\pgfsetlinewidth{1mm}\pgfusepath{stroke}}
	\pgfpageslogicalpageoptions{4}{border code=\pgfsetlinewidth{1mm}\pgfusepath{stroke}}
%    \setbeamercolor{background canvas}{bg=black!3}
    }
\fi
%http://www.cpt.univ-mrs.fr/~masson/latex/Beamer-appearance-cheat-sheet.pdf
\definecolor{Dandelion}         {RGB}{253, 200, 47}
\setbeamertemplate{frametitle}
{\vspace{1mm}
\begin{centering}
\insertframetitle
\end{centering}
{\color{Dandelion} \rule{\textwidth}{1mm}}
}
\setbeamertemplate{itemize item}{$\bullet$}
\setbeamertemplate{navigation symbols}{}
%\setbeamercolor{title}{fg=Fern}
%\setbeamercolor{frametitle}{fg=Fern}
%\setbeamercolor{normal text}{fg=Charcoal}
%\setbeamercolor{block title}{fg=black,bg=Fern!25!white}
%\setbeamercolor{block body}{fg=black,bg=Fern!25!white}
%\setbeamercolor{alerted text}{fg=AlertColor}
\setbeamercolor{itemize item}{fg=black}

\usepackage[latin1]{inputenc}
\usepackage{epic,eepic,epsf}
%\usepackage{latexsym}
\usepackage{verbatim}
\usepackage{listings}
\usepackage{alltt}
\usepackage{minted}

%% all code examples use "code" environment
\newenvironment{code}{\begin{minted}{c}}{\end{minted}}

%\newcommand{\UnderscoreCommands}{\do\verbatiminput%
%\do\citeNP \do\citeA \do\citeANP \do\citeN \do\shortcite%
%\do\shortciteNP \do\shortciteA \do\shortciteANP \do\shortciteN%
%\do\citeyear \do\citeyearNP%
%}
%\usepackage[strings]{underscore}
%\usetheme{Pittsburgh}

%\makeatletter
%\setbeamercolor{background canvas}{bg=white}
%\setbeamercolor{background}{bg=white}
%\makeatother

\title[COMP1511 Introduction to Programming]{COMP1511}
\author{Andrew Taylor/John Shepherd}
\institute{CSE, UNSW}
\date{\today}

\newcommand{\red}[1]{\textcolor{red}{#1}}
\newcommand{\blue}[1]{\textcolor{blue}{#1}}
\newcommand<>{\RedNote}[2]{%
  \begin{alertblock}#3{#1}
    #2
  \end{alertblock}}
\newcommand<>{\Note}[2]{%
  \begin{exampleblock}#3{#1}
    #2
  \end{exampleblock}}
\newcommand{\NotesFrame}{\frame<handout>[plain]{\frametitle{My Notes}}}
\newcommand{\ttblue}[1]{\texttt{\blue{#1}}}

\usepackage{epstopdf}


% prefer pdf over png if both versions exist
\DeclareGraphicsExtensions{%
    .pdf,.PDF,%
    .png,.PNG,%
    .jpg,.mps,.jpeg,.jbig2,.jb2,.JPG,.JPEG,.JBIG2,.JB2}

\setbeamercolor{block body}{bg=gray!7}

\usemintedstyle{tango}

\newenvironment{C}
 {\VerbatimEnvironment
  \setbeamertemplate{blocks}[rounded][shadow=true]
  \begin{block}{}
  \begin{minted}{c}}
 {\end{minted}
  \end{block}}
  
\newenvironment{sh}
 {\VerbatimEnvironment
  \setbeamertemplate{blocks}[rounded][shadow=true]
  \begin{block}{}
  \begin{minted}{sh}}
 {\end{minted}
  \end{block}}
  
\newenvironment{perl}
 {\VerbatimEnvironment
  \setbeamertemplate{blocks}[rounded][shadow=true]
  \begin{block}{}
  \begin{minted}{perl}}
 {\end{minted}
  \end{block}}
  
\newenvironment{python}
 {\VerbatimEnvironment
  \setbeamertemplate{blocks}[rounded][shadow=true]
  \begin{block}{}
  \begin{minted}{python}}
 {\end{minted}
  \end{block}}
  
\newenvironment{html}
 {\VerbatimEnvironment
  \setbeamertemplate{blocks}[rounded][shadow=true]
  \begin{block}{}
  \begin{minted}{html}}
 {\end{minted}
  \end{block}}
  
\newenvironment{txt}
 {\VerbatimEnvironment
  \setbeamertemplate{blocks}[rounded][shadow=true]
  \begin{block}{}
  \begin{minted}{text}}
 {\end{minted}
  \end{block}}

\begin{document}
\section{COMP2041/9041 Final Lecture}
\begin{frame}
\frametitle{Assignment 2 Demo Sessions}

You must attend 1 of 4 assignment demo sessions.

\begin{itemize}
\item Tuesday Nov 7  16:30-19:00 
\item Friday Nov 10  09:30-12:00
\item Thursday Nov 14  17:30-20:00
\item Friday Nov 16 1 17:30-20:00
\end{itemize}

Demo sessions start in a lecture theatre (location TBA) for introduction.
then we move J17 third floor labs

They will be email asking you to indicate which session you will attend.

Not compulsory to attend that session.

\end{frame}

\begin{frame}
\frametitle{Assignment 2 Demo Setup}

When you submit your assignment with give
(from Tuesday on) you will be given a link to
check that your assignment works in the demo setup.

Give early, and check carefully.
\end{frame}

\begin{frame}
\frametitle{Assignment 2 Assessment}

Groups of 6 formed.

You assess 5 other students' assignments.

You demo to the other 5 students.

You must enter assessment of 5 other students assignments.

~20 grades for various attributes to be entered via online form.

Mark will be obtained from median of grades for each question.

1-2 unfair assessors won't change your mark.

Appeal possible to Andrew who will attend all sessions.

\end{frame}

\begin{frame}
\frametitle{Course Aims}
This course aims to explore a range techniques, language and tools 
for the development of software systems.

Hopefully you now:
\begin{itemize}
\item 
understand a wider range of programming languages, tools and techniques
\item 
understand more about how/when to apply these languages, tools and techniques
\end{itemize}
You've been given only an introduction to these topics
but you should be better equipped to:
\begin{itemize}
\item 
given a problem and choose language/tools
\item 
construct a complete software system to meet the problem
\item 
determine how correct/reliable/efficient the system is
\item 
improve correctness/reliability/efficiency of the system
\end{itemize}
\end{frame}

\begin{frame}
\frametitle{Syllabus/Topics}
\begin{itemize}
\item  Perl
\item  Regular expressions
\item  Shell
\item  Unix filters
\item  Generating  web content with Perl (CGI)
\item  Intro Python (for Perl programmers)
\item  Language choice/comparison  - cost, performance, security ..
\item  Performance tuning
\item  Version control, configuration (\textbf{\tt{make}})
\end{itemize}

All topics are potentially examinable but few exam questions on last 3.

\end{frame}

\begin{frame}[fragile]
\frametitle{Assessment}
Assessment summary from the introductory lecture:
\begin{verbatim}
    ass    = mark for assignments     (out of 30)
    labs   = mark for labs            (out of  9)
    test   = mark for weekly tests    (out of  6)
    exam   = mark for exam            (out of 55)
    mark = ass + labs + exam 
   
    okExam = two part 3 questions solved
    
    grade    = HD|DN|CR|PS  if mark >= 50 && okExam
             = FL           if mark < 50 && okExam
             = UF           if !okExam
\end{verbatim}
Note: \textbf{\tt{labs + test + ass1}} marks will be available before the Final Exam.

Check via \textbf{\tt{/home/cs2041/bin/classrun -sturec}}
\end{frame}

\begin{frame}[fragile]
\frametitle{Lab \& Test Marking}
Lab marks from weeks 2..11,13 summed and capped at 9.

Tests were weeks 5..12.

Best 6 of 8 test marks summed to give mark out of 6.

I'll scale either up if its distribution looks low.

\end{frame}

\begin{frame}
\frametitle{Exam}

Internal exam run in CSE labs.

The exam will run in two sessions  Wednesday 09/11

Sessions are 9:15-12:30 \& 12:20-16:10

Students with clashes automatically scheduled to non-clashing session.

Other students can optionally indicate preference for morning/afternoon at
URL email to you at end week 13.

Seating details will appear on class web page 48+ hours prior to actual exam.

Closed book exam - no materials allowed.

Online language cheatsheets \& documentation see: \href{http://www.cse.unsw.edu.au/~cs2041/exam/}{http://www.cse.unsw.edu.au/~cs2041/exam/}

No past exams available,

some past questions released in stuvac.

Some weekly test questions are past exam questions.

Attendance slip will be A4 and most of it available for rough work.

Exam has 3 parts - do all of them!
\end{frame}

\begin{frame}
\frametitle{Exam Part 1}
Must be completed during 1st 30 minutes of 3 hour exam.

No use of computer allowed during this part except to enter answers into 
application and view online documentation,

You can not run Perl or Python or shell or ....

\begin{itemize}
\item Probably about 7 questions
\item Emphasis on reading Shell/Regular expressions/Perl/Python
\item Some questions will ask you to read code and indicate what it does.
\item Questions will mostly be short answer
\end{itemize}

Exact format (skeleton exam) released 24 hours prior to actual exam.
\end{frame}

\begin{frame}
\frametitle{Exam Part 2}
\begin{itemize}
\item Probably about 7 questions
\item Some questions on Shell/Regular expressions/Perl/Python
\item Some questions might ask you to write code
\item Maybe questions on other lecture topics (CGI, git)
\end{itemize}

You  get the part 2 questions at the start of the 3 hours but
some questions may require typing answers into separate file which you won't be able to
do until the first 30 minutes is up.

Exact format (skeleton exam) released 48 hours prior to actual exam.
\end{frame}

\begin{frame}
\frametitle{Exam Part 3}
Probably 6 questions 

You get the part 3 questions at the start of the 3 hours but you
can not run Perl/Python/... or type them in until the first 30 minutes is up.

Almost all students spend the 30 minutes  working on the written questions.

You must perform satisfactorily on the exam to pass the course.

This is defined as solving at least two of the part 3 questions completely.

Exact format (skeleton exam) released by noon day before exam
\end{frame}

\begin{frame}
\frametitle{Part 3 - Question}

\begin{itemize}
\item  Questions will describe a task and ask you to write a program that performs this task.
\item  Questions will usually include examples.
\item  A question may give you some code to start with - most/all will not.
\item  No CGI in part 3.
\item  You may or may not be given test data or other files
\item  1 or more tests may be done on submission.  This does not guarantee any marks.  Do your own testing.
\item  There may be no submission tests for some questions.
\item  It is not sufficient to match any supplied examples.
\item  You may use Shell, Perl or Python to answer any question. 
\end{itemize}
\end{frame}


\begin{frame}
\frametitle{Part 3 - Marking}
\begin{itemize}
\item  Your answers will be run through automatic marking software.
\item  Please follow the input/output format shown exactly.
\item  Please make your program behave exactly as specified.
\item  All answers are also hand marked.  The automatic marking is to assist these markers.
\item  No marks awarded for style or comments.
\item  Use decent formatting so the marker (and you) can read the program.
\item  Comments only necessary if you want to tell the marker something.
\item  Minor errors will result in only a small penalty.
\item  E.g. an answer correct except for a missing semi-colon would receive almost full marks.
\item  No marks will given unless an answer contains a substantial part of a solution (> 33\%).
\item  No marks just for starting a question and writing some code
\end{itemize}
\end{frame}

\begin{frame}
\frametitle{Special Consideration}

By attending the exam, you are saying "I am well enough to sit it".

If you really are sick, stay home and apply for Special Consideration.

Applications for Special Consideration from people who sat the exam
will be ignored.

If you become ill during the exam, ask the supervisor to contact me
and then talk to me.
\end{frame}

\begin{frame}
\frametitle{Provisional Results}

Provisional results will be made available via classrun when marking is complete.

I'll send email announcing this.

Marking usually takes 7-10 days (more if you bug me).

Provisional marks will be emailed to you by Friday 24 

You will be emailed time(s) which you can view your exam and check marking. 

Final results will appear on  uni web pages (myunsw).
\end{frame}

\begin{frame}
\frametitle{Supplementary Assessment}

Most people offered  supplementary exams because they miss original
exam due to illness.

Examiners meeting may also offer students with borderline results
\& good transcripts supplementary assessment. 

I'll offer supplementary assessment to students with borderline results.

See course outline for details.

Please don't plead to be treated specially - I am careful to treat all students equally.

Students who sat the final exam can not increase their mark beyond 50 in the supp.

Students who have passed the subject are not normally offered supplementary assessment.

So if you get a PS but you expected a DN you won't get a supp normally.

\end{frame}

\begin{frame}
\frametitle{Supplementary Exam}

Similar format to final exam (no skeleton released).
 
Supplementary exam likely to be on-or-about

There is no alternative to the supplementary exam - if you miss it
your grade will be FL.

If you think you might be offered supplementary assessment make sure you are available
that week.

Supplementary assessment offers will be sent by email.
\end{frame}

\begin{frame}
\frametitle{COMP2041/9041 - The Bad}

\begin{itemize}
\item
Lecture time not used efficiently.
\item
Late starting with gitlab, git support could have been better
\item
A lot to learn: regex/sh/Python/Perl/CGI/...
\item
Not enough time to cover (so) many things
\item
Labs a lot of work
\end{itemize}
\end{frame}

\begin{frame}
\frametitle{COMP2041/9041 - The Good}

\begin{itemize}
\item
Labs \& tutes (do you agree??)

\item
Weekly Tests (do you agree??)

\item
Tutors

\item
Piazza 

\item
Students
\end{itemize}
\end{frame}

\begin{frame}
\frametitle{And that's all ...}

Good Luck

I hope what you've learnt in this course will be useful.

I hope you get the mark you deserve.
\end{frame}

\end{document}
