% https://groups.google.com/forum/#!topic/comp.text.tex/s6z9Ult_zds
\makeatletter\let\ifGm@compatii\relax\makeatother

\ifx\notes\undefined
    \documentclass{beamer}
 \else
   \documentclass[handout]{beamer}
    \mode<handout>{
        \usepackage{pgfpages}
    \pgfpagesuselayout{4 on 1}[a4paper ,border shrink=2mm, landscape]
	\pgfpageslogicalpageoptions{1}{border code=\pgfsetlinewidth{1mm}\pgfusepath{stroke}}
	\pgfpageslogicalpageoptions{2}{border code=\pgfsetlinewidth{1mm}\pgfusepath{stroke}}
	\pgfpageslogicalpageoptions{3}{border code=\pgfsetlinewidth{1mm}\pgfusepath{stroke}}
	\pgfpageslogicalpageoptions{4}{border code=\pgfsetlinewidth{1mm}\pgfusepath{stroke}}
%    \setbeamercolor{background canvas}{bg=black!3}
    }
\fi
%http://www.cpt.univ-mrs.fr/~masson/latex/Beamer-appearance-cheat-sheet.pdf
\definecolor{Dandelion}         {RGB}{253, 200, 47}
\setbeamertemplate{frametitle}
{\vspace{1mm}
\begin{centering}
\insertframetitle
\end{centering}
{\color{Dandelion} \rule{\textwidth}{1mm}}
}
\setbeamertemplate{itemize item}{$\bullet$}
\setbeamertemplate{navigation symbols}{}
%\setbeamercolor{title}{fg=Fern}
%\setbeamercolor{frametitle}{fg=Fern}
%\setbeamercolor{normal text}{fg=Charcoal}
%\setbeamercolor{block title}{fg=black,bg=Fern!25!white}
%\setbeamercolor{block body}{fg=black,bg=Fern!25!white}
%\setbeamercolor{alerted text}{fg=AlertColor}
\setbeamercolor{itemize item}{fg=black}

\usepackage[latin1]{inputenc}
\usepackage{epic,eepic,epsf}
%\usepackage{latexsym}
\usepackage{verbatim}
\usepackage{listings}
\usepackage{alltt}
\usepackage{minted}

%% all code examples use "code" environment
\newenvironment{code}{\begin{minted}{c}}{\end{minted}}

%\newcommand{\UnderscoreCommands}{\do\verbatiminput%
%\do\citeNP \do\citeA \do\citeANP \do\citeN \do\shortcite%
%\do\shortciteNP \do\shortciteA \do\shortciteANP \do\shortciteN%
%\do\citeyear \do\citeyearNP%
%}
%\usepackage[strings]{underscore}
%\usetheme{Pittsburgh}

%\makeatletter
%\setbeamercolor{background canvas}{bg=white}
%\setbeamercolor{background}{bg=white}
%\makeatother

\title[COMP1511 Introduction to Programming]{COMP1511}
\author{Andrew Taylor/John Shepherd}
\institute{CSE, UNSW}
\date{\today}

\newcommand{\red}[1]{\textcolor{red}{#1}}
\newcommand{\blue}[1]{\textcolor{blue}{#1}}
\newcommand<>{\RedNote}[2]{%
  \begin{alertblock}#3{#1}
    #2
  \end{alertblock}}
\newcommand<>{\Note}[2]{%
  \begin{exampleblock}#3{#1}
    #2
  \end{exampleblock}}
\newcommand{\NotesFrame}{\frame<handout>[plain]{\frametitle{My Notes}}}
\newcommand{\ttblue}[1]{\texttt{\blue{#1}}}

\usepackage{epstopdf}


% prefer pdf over png if both versions exist
\DeclareGraphicsExtensions{%
    .pdf,.PDF,%
    .png,.PNG,%
    .jpg,.mps,.jpeg,.jbig2,.jb2,.JPG,.JPEG,.JBIG2,.JB2}

\setbeamercolor{block body}{bg=gray!7}

\usemintedstyle{tango}

\newenvironment{C}
 {\VerbatimEnvironment
  \setbeamertemplate{blocks}[rounded][shadow=true]
  \begin{block}{}
  \begin{minted}{c}}
 {\end{minted}
  \end{block}}
  
\newenvironment{sh}
 {\VerbatimEnvironment
  \setbeamertemplate{blocks}[rounded][shadow=true]
  \begin{block}{}
  \begin{minted}{sh}}
 {\end{minted}
  \end{block}}
  
\newenvironment{perl}
 {\VerbatimEnvironment
  \setbeamertemplate{blocks}[rounded][shadow=true]
  \begin{block}{}
  \begin{minted}{perl}}
 {\end{minted}
  \end{block}}
  
\newenvironment{python}
 {\VerbatimEnvironment
  \setbeamertemplate{blocks}[rounded][shadow=true]
  \begin{block}{}
  \begin{minted}{python}}
 {\end{minted}
  \end{block}}
  
\newenvironment{html}
 {\VerbatimEnvironment
  \setbeamertemplate{blocks}[rounded][shadow=true]
  \begin{block}{}
  \begin{minted}{html}}
 {\end{minted}
  \end{block}}
  
\newenvironment{txt}
 {\VerbatimEnvironment
  \setbeamertemplate{blocks}[rounded][shadow=true]
  \begin{block}{}
  \begin{minted}{text}}
 {\end{minted}
  \end{block}}

\begin{document}
\section{Perl - Arrays and Hashes}

\begin{frame}[fragile]
\frametitle{Arrays (Lists)}
An {\em{array}} is a sequence of scalars, indexed by position {\small (0,1,2,...)}

The whole array is denoted by ~ \textbf{\tt{@array}}

Individual array elements are denoted by ~
\textbf{\tt{\$}}array\textbf{\tt{[}}index\textbf{\tt{]}}

\textbf{\tt{\$\#}}array gives the {\it{index of the last element}}.

Example:
\begin{perl}
$a[0] = "first string";
$a[1] = "2nd string";
$a[2] = 123;

# or, equivalently,

@a = ("first string", "2nd string", 123);

print "Index of last element is $#a\n";
print "Number of elements is ", $#a+1, "\n";
\end{perl}

\end{frame}

\begin{frame}[fragile]
\frametitle{Arrays (Lists)}
\begin{perl}
@a = ("abc", 123, 'x');

# numeric context ... gives list length
$n = @a;         #  $n == 3

# string context ... gives space-separated elems
$s = "@a";       #  $s eq "abc 123 x"

# scalar context ... gives list length
$t = @a."";      #  $t eq "3"

# print context ... gives joined elems
print @a;        #  displays "abc123x"
\end{perl}

In Perl, interpretation is context-dependent.
\end{frame}

\begin{frame}[fragile]
\frametitle{Arrays (Lists)}
Arrays do not need to be declared, and they grow and shrink as needed.

"Missing" elements are interpolated, e.g.
\begin{perl}
    $abc[0] = "abc";  $abc[2] = "xyz";
    # reference to $abc[1] returns ""
\end{perl}

Can assign {\em{to}} a whole array; can assign {\em{from}} a whole array, e.g.
\begin{perl}
    @numbers = (4, 12, 5, 7, 2, 9);
    ($a, $b, $c, $d) = @numbers;
\end{perl}

Since assignment of list elements happens in parallel ...
\begin{perl}
    ($x, $y) = ($y, $x);   # swaps values of $x, $y}
\end{perl}

\end{frame}

\begin{frame}[fragile]
\frametitle{Arrays (Lists)}
Array {\em{slices}}, e.g.
\begin{perl}
@list = (1, 3, 5, 7, 9);
print "@list[0,2]\n";   # displays "1 5"
print "@list[0..2]\n";  # displays "1 3 5"
print "@list[4,2,3]\n"; # displays "9 5 7"
print "@list[0..9]\n";  # displays "1 3 5 7 9"
\end{perl}

Array values interpolated into array literals:
\begin{perl}
@a = (3, 5, 7);
@b = @a;           # @b = (3,5,7);
@c = (1, @a, 9);   # @c = (1,3,5,7,9);
@a == (@a) == ((@a)) ...
\end{perl}

\end{frame}

\begin{frame}[fragile]
\frametitle{Arrays (Lists)}
Arrays can be accessed element-at-a-time using the \textbf{\tt{for}} loop:
\begin{perl}
@nums = (23, 95, 33, 42, 17, 87);
$sum = 0;
for ($i = 0; $i < @nums; $i++) {   # @nums gives length
    $sum += $nums[$i];
}
$sum = 0;
foreach $num (@nums) { sum += $num; }
\end{perl}

\textbf{\tt{push}} and \textbf{\tt{pop}} act on the "right-hand" end of an array:
\begin{perl}
                   # Value of @a
@a = (1,3,5);      # (1,3,5)
push @a, 7;        # (1,3,5,7)
$x = pop @a;       # (1,3,5,7), $x == 7
$y = pop @a;       # (1,3,5), $y == 5
\end{perl}

\end{frame}

\begin{frame}
\frametitle{Arrays (Lists)}
Other useful operations on arrays:

\begin{center}
\begin{tabular}{lll}

  \textbf{\tt{@b = sort(@a)}} & 
  ~ & 
  returns sorted version of \textbf{\tt{@a}}
\\

  \textbf{\tt{@b = reverse(@a)}} & 
  ~ & 
  returns reversed version of \textbf{\tt{@a}}
\\

  \textbf{\tt{shift(@a)}} & 
  ~ & 
  like \textbf{\tt{pop(@a)}}, but from left-hand end
\\

  \textbf{\tt{unshift(@a,x)}} & 
  ~ & 
  like \textbf{\tt{push(@a,x)}}, but at left-hand end
\\
\end{tabular}
\end{center}
\end{frame}

\begin{frame}[fragile,shrink]
\frametitle{Lists as Strings}
Recall the marks example from earlier on; we used \textbf{\tt{"54,67,88"}} to
effectively hold a list of marks.

Could we turn this into a real list if e.g. we wanted to compute an average?

The {\em{split}} operation allows us to do this:

Syntax: ~ \textbf{\tt{split(/}}$pattern$\textbf{\tt{/,}}$string$\textbf{\tt{)}} ~ returns a list

The {\em{join}} operation allows us to convert from list to string:

Syntax: ~ \textbf{\tt{join(}}$string$\textbf{\tt{,}}$list$\textbf{\tt{)}} ~ returns a string
\\{\small (Don't confuse this with the \textbf{\tt{join}} filter in the shell.
Perl's \textbf{\tt{join}} acts more like \textbf{\tt{paste}}.)}
\end{frame}

\begin{frame}[fragile,shrink]
\frametitle{Lists as Strings}
Examples:
\begin{perl}
$marks = "99,67,85,48,77,84";

@listOfMarks = split(/,/, $marks);
# assigns (99,67,85,48,77,84) to @listOfMarks

$sum = 0;
foreach $m (@listOfMarks) {
    $sum += $m;
}

$newMarks = join(':',@listOfMarks);
# assigns "99:67:85:48:77:84" to $newMarks
\end{perl}


\end{frame}

\begin{frame}[fragile,shrink]
\frametitle{Lists as Strings}
Complex splits can be achieved by using a full regular expression
rather than a single delimiter character.
\\
If part of the regexp is parenthesised,
the corresponding part of each delimiter
is retained in the resulting list.
\begin{perl}
split(/[#@]+/,'ab##@#c#d@@e');  #gives (ab,c,d,e)
split(/([#@]+)/,'ab##@#c#d@@e');#gives (ab,##@#,c,#,d,@@,e)
split(/([#@])+/,'ab##@#c#d@@e');#gives (ab,#,c,#,d,@,e)
\end{perl}


And as a specially useful case, the empty regexp
is treated as if it matched between every character,
splitting the string into a list of single characters:

\begin{perl}
    split(//, 'hello'); # gives (h, e, l, l, o)
\end{perl}


\end{frame}
\begin{frame}[fragile,shrink]
\frametitle{Associative Arrays (Hashes)}
As well as arrays indexed by numbers, Perl supports arrays indexed by
strings: {\em{hashes}}.

Conceptually, as hash is a set {\small (not list)} of $(key,value)$ pairs.

We can deal with an entire hash at a time via \textbf{\tt{\%}}$hashName$, e.g.
\begin{perl}
         #  Key      Value

 %days = ( "Sun" => "Sunday",
           "Mon" => "Monday",
           "Tue" => "Tuesday",
           "Wed" => "Wednesday",
           "Thu" => "Thursday",
           "Fri" => "Friday",
           "Sat" => "Saturday" );
\end{perl}

\end{frame}

\begin{frame}[fragile,shrink]
\frametitle{Associative Arrays (Hashes)}
Individual components of a hash are accessed via \textbf{\tt{\$}}$hashName$\textbf{\tt{\{}}$keyString$\textbf{\tt{\}}}

Examples:
\begin{perl}
$days{"Sun"}   # returns "Sunday"
$days{"Fri"}   # returns "Friday"
$days{"dog"}   # is undefined (interpreted as "")
$days{0}       # is undefined (interpreted as "")

# inserts a new (key,value)
$days{dog} = "Dog Day Afternoon";   # bareword OK as key

# replaces value for key "Sun"
$days{"Sun"} = Soonday;             # bareword OK as value
\end{perl}

\end{frame}

\begin{frame}[fragile,shrink]
\frametitle{Associative Arrays (Hashes)}
Consider the following two assignments:
\begin{perl}
@f = ("John", "blue", "Anne", "red", "Tim", "pink");
%g = ("John" => "blue", "Anne" => "red", "Tim" => "pink");
\end{perl}


The first produces an array of strings that can be accessed via position,
such as \textbf{\tt{\$f[0]}}

The second produces a lookup table of names and colours, e.g. \textbf{\tt{\$g\{"Tim"\}}}.
\\(In fact the symbols \textbf{\tt{=>}} and comma have identical meaning in a list,
so either right-hand side could have been used.
However, always use the arrow form exclusively for hashes.)
\end{frame}

\begin{frame}[fragile,shrink]
\frametitle{Associative Arrays (Hashes)}
Consider iterating over each of these data structures:

\begin{center}

\begin{center}
\begin{tabular}{lll}

\begin{minipage}{4cm}
\begin{perl}
foreach $x (@f) {
    print "$x\n";
}

John
blue
Anne
red
Tim
pink
\end{perl}

 ~\end{minipage}
 & \begin{minipage}{19cm}
\begin{perl}
foreach $x (keys %g) {
   print "$x => $g{$x}\n";
}

Anne => red
Tim => pink
John => blue



\end{perl}

~\end{minipage}
\\[1ex]
\end{tabular}
\end{center}
\end{center}
The data comes out of the hash in a fixed but arbitrary order
(due to the hash function).
\end{frame}

\begin{frame}[fragile,shrink]
\frametitle{Associative Arrays (Hashes)}
There are several ways to examine the $(key,value)$ pairs
in a hash:
\begin{perl}
foreach $key (keys %myHash) {
    print "($key, $myHash{$key})\n";
}
\end{perl}


or, if you just want the values without the keys
\begin{perl}
foreach $val (values %myHash) {
    print "(?, $val)\n";
}
\end{perl}


or, if you want them both together
\begin{perl}
while (($key,$val) = each %myHash) {
    print "($key, $val)\n";
}
\end{perl}


Note that each method produces the keys/values in the same order.
It's illegal to change the hash within these loops.
\end{frame}

\begin{frame}[fragile,shrink]
\frametitle{Associative Arrays (Hashes)}
Example (collecting marks for each student):
\begin{itemize}
\item  a data file of $(name,mark)$ pairs, space-separated, one per line
\item  out should be $(name,marksList)$, with comma-separated marks
\end{itemize}
\begin{perl}
while (<>) {
    chomp;    # remove newline
    ($name, $mark) = split;    # separate data fields
    $marks{$name} .= ",$mark"; # accumulate marks
}
foreach $name (keys %marks) {
    $marks{$name} =~ s/,//;    # remove comma prefix
    print "$name $marks{$name}\n";
}
\end{perl}

\end{frame}

\begin{frame}[fragile,shrink]
\frametitle{Associative Arrays (Hashes)}
The \textbf{\tt{delete}} function removes an entry (or entries) from an associative array.

To remove a single pair:
\begin{perl}
delete $days{"Mon"};  # "I don't like Mondays"
\end{perl}


To remove multiple pairs:
\begin{perl}
delete @days{ ("Sat","Sun") };  # Oh noes - no weekend!
\end{perl}


To clean out the entire hash:
\begin{perl}
foreach $d (keys %days) { delete $days{$d}; }
# or, more simply
%days = ();
\end{perl}

\end{frame}
\end{document}
