\input{include.tex}
\begin{document}
\section{CGI Scripts}

\begin{frame}
\frametitle{CGI Scripts}
CGI scripts can be written in {\em{most}} languages.

The better CGI languages:
\begin{itemize}
\item  are good at manipulating character strings
\item  make it easy to produce HTML
\end{itemize}
Perl satisfies both of these criteria ok on its own.

Libraries like \textbf{\tt{CGI.pm}} make Perl even better for CGI.
\end{frame}

\begin{frame}[fragile]
\frametitle{CGI at CSE}
On CSE machines, users typically place CGI scripts in:

{\tt /home/{\it{UserName}}/public\_html/cgi-bin}

And access them via:

{\tt http://cgi.cse.unsw.edu.au/\textasciitilde{\it{Username}}/cgi-bin/{\it{Script}}}

Nowadays, you can place CGI scripts
\begin{itemize}
\item  anywhere under your \textbf{\tt{public\_html}} directory
\item  provided that they have a \textbf{\tt{.cgi}} or suffix
\end{itemize}
and access them via e.g.

    {\tt http://cgi.cse.unsw.edu.au/\textasciitilde{\it{UserName}}/path/to/script.cgi}

The CSE web server will automatically forward them to
the CGI server for execution.
\end{frame}

\begin{frame}[fragile]
\frametitle{CGI at CSE}
A note on file/directory protections and security ...
\begin{itemize}
\item  files under \textbf{\tt{public\_html}} need to be readable
\item  directories under \textbf{\tt{public\_html}} need to be executable
\end{itemize}
so that at least the Web server can access them.

A special command:
\begin{perl}
    priv webonly FileOrDirecctory
\end{perl}

makes files/dirs readable only to you and the web server.
\end{frame}

\begin{frame}
\frametitle{CGI and Security}
Putting up a CGI scripts means that
\begin{itemize}
\item  anyone, anywhere can execute your script
\item  they can give it any data they like
\end{itemize}
If you are not careful how data is used ...

Many prople run Perl CGI scripts in ``taint'' mode
\begin{itemize}
\item  generates an error if tainted data used unsafely
\end{itemize}
Tainted data = any CGI parameter

Unsafely = in system-type operations (e.g. \textbf{\tt{`...`}})
\end{frame}

\begin{frame}
\frametitle{CGI.pm}
{\tt CGI.pm} is a Perl module to simplify CGI scripts.

It provides functions/methods that make it easy
\begin{itemize}
\item  to access parameters and other data for CGI scripts
\item  to produce HTML output from the script
\end{itemize}
CGI.pm supports two styles of programming:

\begin{itemize}
\item  object-oriented, with CGI objects and methods 
\item  function-oriented, with function calls (single implicit CGI object)
\end{itemize}

We'll use simpler function-oriented style in this course.
\end{frame}

\begin{frame}
\frametitle{CGI.pm}
CGI.pm has a range of methods/functions for:
\begin{itemize}
\item  producing HTML ~ {\small (several flavours, including browser-specific)},
\item  building HTML forms ~ {\small (overall wrapping, plus all form elements)}
\item  CGI handling ~ {\small (manipulating parameters, managing state)}
\end{itemize}
HTML and form building methods typically
\begin{itemize}
\item  accept a collection of string arguments
\item  return a string that contains a fragment of HTML
\end{itemize}

A dynamic Web page is produced by
\begin{itemize}
\item  printing a collection of such HTML fragments
\end{itemize}
\end{frame}

\begin{frame}[fragile]
\frametitle{Example CGI.pm}
Consider a data collection form (\textbf{\tt{SayHello.html}}):
\begin{perl}
    <form name="Hello" action="HelloScript.cgi">
    Your name: <input name="UserName" type="text">
    <input type=submit value="Say Hello">
    </form>
\end{perl}

And consider that we type \textbf{\tt{John}} into the input box.
\end{frame}

\begin{frame}[fragile]
\frametitle{Example CGI.pm}
An OO-style script (\textbf{\tt{HelloScript.cgi}})
\begin{perl}
use CGI;
$cgi = new CGI;
$name = $cgi->param("UserName");
print $cgi->header(), $cgi->start_html(),
      $cgi->p("Hello there, $name"),
      $cgi->end_html();
\end{perl}

Output of script (sent to browser):
\begin{perl}
Content-type: text/html

<!DOCTYPE HTML PUBLIC "-//IETF//DTD HTML//EN">
<HTML><HEAD><TITLE>Untitled Document</TITLE>
</HEAD><BODY><P>Hello there, John</P></BODY></HTML>
\end{perl}

\end{frame}

\begin{frame}[fragile]
\frametitle{Example CGI.pm}
A function-style script (\textbf{\tt{HelloScript.cgi}})
\begin{perl}
use CGI qw/:standard/;
$name = param("UserName");
print header(), start_html(),
      p("Hello there, $name"),
      end_html();
\end{perl}

Output of script (sent to browser):
\begin{perl}
Content-type: text/html

<!DOCTYPE HTML PUBLIC "-//IETF//DTD HTML//EN">
<HTML><HEAD><TITLE>Untitled Document</TITLE>
</HEAD><BODY><P>Hello there, John</P></BODY></HTML>
\end{perl}

\end{frame}

\begin{frame}[fragile]
\frametitle{Calling CGI.pm Methods}
CGI.pm methods often accept many (optional) parameters.

Special method-call syntax available throughout CGI.pm:
\begin{perl}
    MethodName(-ArgName1=>Value1,
               -ArgName2=>Value2,
               -ArgName3=>Value3,
               ...
               -ArgNamen=>Valuen,
               );
\end{perl}

Example:
\begin{perl}
    print header(-type=>'image/gif',-expires=>'+3d');
\end{perl}

Argument names are case-insensitive; args can be supplied in any order.
\end{frame}

\begin{frame}[fragile]
\frametitle{Calling CGI.pm Methods}
CGI.pm doesn't explicitly define methods for all HMTL tags.

Instead, constructs them on-the-fly using rules about arguments.

This allows you to include arbitrary attributes in HTML tags
\begin{perl}
    MethodName(-AttrName=>Value,..., OtherArgs, ...);
\end{perl}

If first argument is an associative array, it is converted into tag attributes.

Other unnamed string arguments are concatenated space-separated.

Methods that behave like this are called {\em HTML shortcuts}.
\end{frame}

\begin{frame}
\frametitle{Calling CGI.pm Methods}
Examples of HTML shortcuts:

\begin{center}
\begin{tabular}{|l|l|}
\hline
\textbf{\tt{h1()}} ~ or ~ \textbf{\tt{h1}}
&
\textbf{\tt{{\textless}H1>}}
\\ \hline
\textbf{\tt{h1('some','contents')}}
&
\textbf{\tt{{\textless}H1>some contents{\textless}/H1>}}
\\ \hline
\textbf{\tt{h1(\{-align=>left\})}}
&
\textbf{\tt{{\textless}H1 ALIGN="left">}}
\\ \hline
\textbf{\tt{h1(\{-align=>left\},'Head')}}
&
\textbf{\tt{{\textless}H1 ALIGN="left">Head{\textless}/H1>}}
\\ \hline
\textbf{\tt{p()}} ~ or ~ \textbf{\tt{p}}
&
\textbf{\tt{{\textless}P>}}
\\ \hline
\textbf{\tt{p('how's',"this","now")}}
&
\textbf{\tt{{\textless}P>how's this now{\textless}/P>}}
\\ \hline
\textbf{\tt{p(\{-align=>center\},'Now!')}} & \textbf{\tt{{\textless}P ALIGN="center">Now!{\textless}/P>}} \\ \hline\hline
\end{tabular}
\end{center}

\end{frame}

\begin{frame}[fragile]
\frametitle{Accessing Data Items}
The \textbf{\tt{{\bf{{\em{param}}}}}} method provides access to CGI parameters.
\begin{itemize}
\item  can get a list of names for all parameters
\item  can get value for a single named parameter
\item  can modify the values of individual parameters
\end{itemize}
Examples:
\begin{perl}
# get a list of names of all parameters
@names = param();
# get value of parameter "name"
$name = param('name');
# get values of parameter "choices"
@list = param('choices');
# set value of "colour" parameter to 'red"
param('colour','red');
param(-name=>'colour', -value='red');
\end{perl}

\end{frame}

\begin{frame}[fragile]
\frametitle{Accessing Data Items}
Example - dump a table of CGI params:
\begin{perl}
#!/usr/bin/perl
use CGI ':standard';
@params = param();
print header, "<html><body>";

foreach $p (@params) {
    $v = param($p);
    $rows .= "<tr><td>$p</td><td>$v ";
}

print "<center><table border=1>
<tr><th>Param<th>Value
$rows
</table>
</body></center></html>";
\end{perl}

\end{frame}

\begin{frame}[fragile]
\frametitle{Accessing Data Items}
Example - dump a table of CGI params, using shortcuts:
\begin{perl}
#!/usr/bin/perl
use CGI ':standard';
@params = param();
print header, start_html;

foreach $p (@params) {
   $rows .= Tr(td([$p, param($p)]));
}
print center(
         table({-border=>1},
            Tr(th(['Param','Value'])),
            $rows
         )
      ),
      end_html;
\end{perl}

\end{frame}

\begin{frame}
\frametitle{Generating Forms}
CGI.pm has methods to assist in generating forms dynamically:

\begin{center}
\begin{tabular}{lll}

  \begin{minipage}{2.5cm}\textbf{\tt{start\_form}} ~\end{minipage}
   & \begin{minipage}{8.5cm}generates a \textbf{\tt{{\textless}form>}} tag with \\
    optional params for \textbf{\tt{action}},...~\end{minipage}
\\[1ex]

  \begin{minipage}{2.5cm}\textbf{\tt{end\_form}} ~\end{minipage}
   & \begin{minipage}{8.5cm}generates a \textbf{\tt{{\textless}/form>}} tag~\end{minipage}
\\[1ex]
\end{tabular}
\end{center}

Plus methods for each different kind of data collection element \\
\begin{itemize}
\item  \textbf{\tt{textfield}}, ~ \textbf{\tt{textarea}}, ~ \textbf{\tt{password\_field}}
\item  \textbf{\tt{popup\_menu}}, ~ \textbf{\tt{scrolling\_list}}
\item  \textbf{\tt{checkbox}}, ~ \textbf{\tt{radio\_group}}, ~ \textbf{\tt{checkbox\_group}}
\item  \textbf{\tt{submit}}, ~ \textbf{\tt{reset}}, ~ \textbf{\tt{button}}, ~ \textbf{\tt{hidden}}
\end{itemize}
\end{frame}

\begin{frame}[fragile]
\frametitle{Self-invoking Form}

\begin{perl}
use CGI qw/:standard/;  # qw/X/ == 'X'
print header,
   start_html('A Simple Example'),
   h1(font({-color=>'blue'},'A Simple Example')),
   start_form,
   "What's your name? ",textfield('name'),p,
   "What's your favorite color? ",
   popup_menu(-name=>'color',
            -values=>['red','green','blue','yellow'])
   p,
   submit,
   end_form;
if (param()) {
   print "Your name is ",em(param('name')),p,
      "Your favorite color is ",em(param('color')),
      hr;
}
\end{perl}

\end{frame}

\begin{frame}
\frametitle{CGI Script Structure}
CGI scripts {\it{can}} interleave computation and output.

Arbitrary interleaving is not generally effective \\
{\small (e.g. produce some output and then encounter an error in middle of table)}

Useful structure for (large) scripts:
\begin{itemize}
\item  collect and check parameters; handle errors
\item  use parameters to compute result data structures
\item  convert results into HTML string
\item  output entire well-formed HTML string
\end{itemize}
\end{frame}

\begin{frame}
\frametitle{Multi-page (State-based) Scripts}
Often, a Web-based transaction goes through several stages.

Sometimes useful to implement all stages by a single script.

Such scripts are
\begin{itemize}
\item  structured as a collection of cases, distinguished by a "state" variable
\item  each state sets parameter to pass to next invocation of same script
\item  new invocation produces a new state (different value of "state" variable)
\end{itemize}
Overall effect: a single script produces many different Web pages.
\end{frame}

\begin{frame}[fragile]
\frametitle{Multi-page (State-based) Scripts}
Example (state-based script schema):
\begin{perl}
    $state = param("state");
    if ($state eq "") {
       do processing for initial state
       set up form to invoke next state
    } elsif ($state == Value1) {
       do processing for state 1
       set up form to invoke next state
    } elsif ($state == Value2) {
       do processing for state 2
       set up form to invoke next state
    } elsif ($state == Value3) {
       do processing for state 3
       set up form to invoke next state
    }
    ...
\end{perl}

\end{frame}

\begin{frame}[fragile]
\frametitle{Cookies}
Web applications often need to maintain state (variables)
between execution of their CGI script(s).

Hidden input fields are useful for the one "session".

Cookies provide more persistant storage.

Cookies are strings sent to web clients in the response headers.

Clients (browsers) store these strings in a file and send them back in the header
when they subsequently access the site. For example:

{\tiny
\begin{sh}
$ ./webget.pl  http://www.amazon.com/
HTTP/1.1 200 OK
Date: Thu, 19 May 2011 00:54:27 GMT
Server: Server
Set-Cookie: skin=noskin; path=/;domain=.amazon.com;expires=Thu, 19-May-2011 00:54:27 GMT
Set-cookie: session-id-time=208567201l;path=/;domain=.amazon.com;expires=Tue Jan 01 08:00:01 2036 GMT
Set-cookie: session-id=191-0575084-2685655;path=/;domain=.amazon.com;expires=Tue Jan 01 08:00:01 2036 GMT
\end{sh}
}
Web clients send the cookie strings back next time they fetch pages from Amazon.
\end{frame}

\begin{frame}[fragile]
\frametitle{Storing a Hash}
The Storable module provides an easy way to store a hash, e.g:

\begin{perl}
use Storable;
$cache_file = "./.cache";
%h = %{retrieve($cache_file)} if -r $cache_file;
$h{COUNT}++;
print "This script has now been run $h{COUNT} times\n";
store(\%h, $cache_file);
\end{perl}

{\tiny Source:  \href{http://cgi.cse.unsw.edu.au/~cs2041/code/web/persistent.pl}{http://cgi.cse.unsw.edu.au/{\textasciitilde}cs2041/code/web/persistent.pl}}

\begin{sh}
$ persistent.pl 
This script has now been run 1 times
$ persistent.pl 
This script has now been run 2 times
$ persistent.pl 
This script has now been run 3 times
...
\end{sh}
\end{frame}

\begin{frame}[fragile]
\frametitle{A Web Client with Cookies}

We can  add code to our simple web client to store cookies it receives using Storable.

\begin{perl}
use Storable;
$cookies_db = "./.cookies";
%cookies = %{retrieve($cookies_db)} if -r $cookies_db;
...
while (<$c>) {
  last if /^\s*$/;
  next if !/^Set-Cookie:/i;
  my ($name,$value, %v) = /([^=;\s]+)=([^=;\s]+)/g;
  my $domain = $v{'domain'} || $host;
  my $path = $v{'path'} || $path;
  $cookies{$domain}{$path}{$name} = $value;
  print "Received cookie $domain $path $name=$value\n"
}
\end{perl}

{\tiny Source:  \href{http://cgi.cse.unsw.edu.au/~cs2041/code/web/webget-cookies.pl}{http://cgi.cse.unsw.edu.au/{\textasciitilde}cs2041/code/web/webget-cookies.pl}}

\end{frame}

\begin{frame}[fragile]
\frametitle{A Web Client with Cookies}

And add code to send cookies when making requests:

\begin{small}
\begin{perl}
use Storable;
$cookies_db = "./.cookies";
%cookies = %{retrieve($cookies_db)} if -r $cookies_db;
...
foreach $domain (keys %cookies) {
  next if $host !~ /$domain$/;
  foreach $cookie_path (keys %{$cookies{$domain}}) {
    next if $path !~ /^$cookie_path/;
    foreach $name (keys %{$cookies{$domain}{$path}}) {
      my $cookie = $cookies{$domain}{$path}{$name}
      print $c "Cookie: $cookie\n";
    }
  }
}
\end{perl}
\end{small}

{\tiny Source:  \href{http://cgi.cse.unsw.edu.au/~cs2041/code/web/webget-cookies.pl}{http://cgi.cse.unsw.edu.au/{\textasciitilde}cs2041/code/web/webget-cookies.pl}}
\end{frame}

\begin{frame}[fragile]
\frametitle{A Web Client with Cookies}

In action:

\begin{small}
\begin{sh}
$ webget-cookies.pl http://www.amazon.com/
Received cookie .amazon.com / skin=noskin
Received cookie .amazon.com / session-id-time=2092797201l
Received cookie .amazon.com / session-id=192-8901109-68109
$ webget-cookies.pl http://www.amazon.com/
Sent cookie skin=noskin
Sent cookie session-id-time=2092797201l
Sent cookie session-id=192-8901109-6810988
Received cookie .amazon.com / skin=noskin
Received cookie .amazon.com / ubid-main=198-1199999-11869
Received cookie .amazon.com / session-id-time=2092797201l
Received cookie .amazon.com / session-id=192-8901109-68109
\end{sh}
\end{small}
\end{frame}

\begin{frame}[fragile]
\frametitle{CGI Script Setting Cookie Directly}

This crude script puts a cookie in the header directly.

And retrieves a cookie from the HTTP\_COOKIE environment variable.

\begin{perl}
$x = 0;
if ($ENV{HTTP_COOKIE} =~ /\bx=(\d+)/) {
    $x = $1 + 1;
}
print "Content-type: text/html
Set-Cookie: x=$x;

<html><head></head><body>
x=$x
</body></html>";
\end{perl}

{\tiny Source:  \href{http://cgi.cse.unsw.edu.au/~cs2041/code/web/simple_cookie.cgi}{http://cgi.cse.unsw.edu.au/{\textasciitilde}cs2041/code/web/simple\_cookie.cgi}}
\end{frame}

\begin{frame}[fragile,shrink]
\frametitle{Using CGI.pm to Set a Cookie}

CGI.pm provides more convenient access to cookies.

\begin{perl}
use CGI qw/:all/;
use CGI::Cookie;

%cookies = fetch CGI::Cookie;
$x = 0;
$x = $cookies{'x'}->value if $cookies{'x'};
$x++;
print header(-cookie=>"x=$x");
print start_html('Cookie Example')
print "x=$x\n"
print end_html;
\end{perl}

{\tiny Source:  \href{http://cgi.cse.unsw.edu.au/~cs2041/code/web/simple_cookie.cgipm.cgi}{http://cgi.cse.unsw.edu.au/{\textasciitilde}cs2041/code/web/simple\_cookie.cgipm.cgi}}
\end{frame}

\begin{frame}[fragile]
\frametitle{CGI Security Vulnerability - Input Parameter length}
CGI script may expect a parameter containing a few bytes, e.g. user name.

But a malicious user may supply instead megabytes.

This may the first step in a buffer overflow or denial of service attack 

Always check/limit length of input parameters.
\begin{perl}
$user = param('user');
$user = substr $user, 0, 64;
\end{perl}

\end{frame}

\begin{frame}[fragile]
\frametitle{CGI Security Vulnerability - Absolute Pathname and ..}
CGI script may use a parameter has a filename. 

A malicious user may supply an input containing / or ..

This will allow read and/or write access to other files on system.

Always santitize of input parameters.

Safest to remove all but necessary characters, e.g.:

\begin{perl}
$name = param('name');
$name = s/[^a-z0-9]//g; 
\end{perl}

\end{frame}

\begin{frame}
\frametitle{CGI Security Vulnerability - Perl's Two Argument Open}
The 2 argument version of Perl's open treats {\textgreater} and |
as special characters.

A malicious user may supply an input containing thgese characters

This will allow files to be written and arbitrary programs to be run.

Always santitize input parameters.

Safest to use 3 argument form of open.
\end{frame}

\begin{frame}
\frametitle{CGI Security Vulnerability - User Input \& External Programs.}

A CGI script may pass user input as arguments to an external program.
External program are often run via a shell, e.g. Perl's system and back quotes.

A malicious user may supply input containing shell metacharacters such as | or ;

This will allow arbitrary programs to be run.

Always santitize input parameters.

Safest to run external programs directly (not via shell).
\end{frame}

\begin{frame}
\frametitle{CGI Security Vulnerability - SQL Injection}

A CGI script may incorporate user input in SQL commands.

A malicious user may supply input containing SQL metacharacters such as '

This may allow the user to circumvent authentication.

Remove or quote SQL metacharacters before using them in queries.

Safest to run query via PREPARE.
\end{frame}

\begin{frame}
\frametitle{CGI Security Vulnerability - Cross-site Scripting (XSS)}

A CGI script may incorporate user input into web pages shown to other users.

A malicious user may supply input containing HTML particularly Javascript.

This Javascript can redirect links, steal information etc.

Remove {\textless}, \textgreater, \& characters from input before incorporating in web pages.

In other contexts, e.g. within script tags, other characters unsafe.
\end{frame}

\begin{frame}[fragile]
\frametitle{Further Information ...}
Comprehensive documentation attached to course Web page:

\href{http://perldoc.perl.org/CGI.html}{http://perldoc.perl.org/CGI.html}

Most Perl books have some material on CGI.pm.
\end{frame}

\end{document}
