% https://groups.google.com/forum/#!topic/comp.text.tex/s6z9Ult_zds
\makeatletter\let\ifGm@compatii\relax\makeatother

\ifx\notes\undefined
    \documentclass{beamer}
 \else
   \documentclass[handout]{beamer}
    \mode<handout>{
        \usepackage{pgfpages}
    \pgfpagesuselayout{4 on 1}[a4paper ,border shrink=2mm, landscape]
	\pgfpageslogicalpageoptions{1}{border code=\pgfsetlinewidth{1mm}\pgfusepath{stroke}}
	\pgfpageslogicalpageoptions{2}{border code=\pgfsetlinewidth{1mm}\pgfusepath{stroke}}
	\pgfpageslogicalpageoptions{3}{border code=\pgfsetlinewidth{1mm}\pgfusepath{stroke}}
	\pgfpageslogicalpageoptions{4}{border code=\pgfsetlinewidth{1mm}\pgfusepath{stroke}}
%    \setbeamercolor{background canvas}{bg=black!3}
    }
\fi
%http://www.cpt.univ-mrs.fr/~masson/latex/Beamer-appearance-cheat-sheet.pdf
\definecolor{Dandelion}         {RGB}{253, 200, 47}
\setbeamertemplate{frametitle}
{\vspace{1mm}
\begin{centering}
\insertframetitle
\end{centering}
{\color{Dandelion} \rule{\textwidth}{1mm}}
}
\setbeamertemplate{itemize item}{$\bullet$}
\setbeamertemplate{navigation symbols}{}
%\setbeamercolor{title}{fg=Fern}
%\setbeamercolor{frametitle}{fg=Fern}
%\setbeamercolor{normal text}{fg=Charcoal}
%\setbeamercolor{block title}{fg=black,bg=Fern!25!white}
%\setbeamercolor{block body}{fg=black,bg=Fern!25!white}
%\setbeamercolor{alerted text}{fg=AlertColor}
\setbeamercolor{itemize item}{fg=black}

\usepackage[latin1]{inputenc}
\usepackage{epic,eepic,epsf}
%\usepackage{latexsym}
\usepackage{verbatim}
\usepackage{listings}
\usepackage{alltt}
\usepackage{minted}

%% all code examples use "code" environment
\newenvironment{code}{\begin{minted}{c}}{\end{minted}}

%\newcommand{\UnderscoreCommands}{\do\verbatiminput%
%\do\citeNP \do\citeA \do\citeANP \do\citeN \do\shortcite%
%\do\shortciteNP \do\shortciteA \do\shortciteANP \do\shortciteN%
%\do\citeyear \do\citeyearNP%
%}
%\usepackage[strings]{underscore}
%\usetheme{Pittsburgh}

%\makeatletter
%\setbeamercolor{background canvas}{bg=white}
%\setbeamercolor{background}{bg=white}
%\makeatother

\title[COMP1511 Introduction to Programming]{COMP1511}
\author{Andrew Taylor/John Shepherd}
\institute{CSE, UNSW}
\date{\today}

\newcommand{\red}[1]{\textcolor{red}{#1}}
\newcommand{\blue}[1]{\textcolor{blue}{#1}}
\newcommand<>{\RedNote}[2]{%
  \begin{alertblock}#3{#1}
    #2
  \end{alertblock}}
\newcommand<>{\Note}[2]{%
  \begin{exampleblock}#3{#1}
    #2
  \end{exampleblock}}
\newcommand{\NotesFrame}{\frame<handout>[plain]{\frametitle{My Notes}}}
\newcommand{\ttblue}[1]{\texttt{\blue{#1}}}

\usepackage{epstopdf}


% prefer pdf over png if both versions exist
\DeclareGraphicsExtensions{%
    .pdf,.PDF,%
    .png,.PNG,%
    .jpg,.mps,.jpeg,.jbig2,.jb2,.JPG,.JPEG,.JBIG2,.JB2}

\setbeamercolor{block body}{bg=gray!7}

\usemintedstyle{tango}

\newenvironment{C}
 {\VerbatimEnvironment
  \setbeamertemplate{blocks}[rounded][shadow=true]
  \begin{block}{}
  \begin{minted}{c}}
 {\end{minted}
  \end{block}}
  
\newenvironment{sh}
 {\VerbatimEnvironment
  \setbeamertemplate{blocks}[rounded][shadow=true]
  \begin{block}{}
  \begin{minted}{sh}}
 {\end{minted}
  \end{block}}
  
\newenvironment{perl}
 {\VerbatimEnvironment
  \setbeamertemplate{blocks}[rounded][shadow=true]
  \begin{block}{}
  \begin{minted}{perl}}
 {\end{minted}
  \end{block}}
  
\newenvironment{python}
 {\VerbatimEnvironment
  \setbeamertemplate{blocks}[rounded][shadow=true]
  \begin{block}{}
  \begin{minted}{python}}
 {\end{minted}
  \end{block}}
  
\newenvironment{html}
 {\VerbatimEnvironment
  \setbeamertemplate{blocks}[rounded][shadow=true]
  \begin{block}{}
  \begin{minted}{html}}
 {\end{minted}
  \end{block}}
  
\newenvironment{txt}
 {\VerbatimEnvironment
  \setbeamertemplate{blocks}[rounded][shadow=true]
  \begin{block}{}
  \begin{minted}{text}}
 {\end{minted}
  \end{block}}

\begin{document}
\section{Perl - Functions}


\begin{frame}[fragile,shrink]
\frametitle{Perl Library Functions}
Perl has literally hundreds of functions for all kinds of purposes:
\begin{itemize}
\item  file manipulation, database access, network programming, etc. etc.
\end{itemize}
It has an especially rich collection of functions for strings.

E.g. lc, uc, length.

Consult on-line Perl manuals, reference books, example programs for further information.
\end{frame}

\begin{frame}[fragile,shrink]
\frametitle{Defining Functions}
Perl functions (or subroutines) are defined via \textbf{\tt{sub}}, e.g.
\begin{perl}
sub sayHello {
    print "Hello!\n";
}
\end{perl}


And used by calling, with or without \textbf{\tt{\&}}, e.g.
\begin{perl}
&sayHello;  # arg list optional with &
sayHello(); # more common, show empty arg list explicitly
\end{perl}

\end{frame}

\begin{frame}[fragile,shrink]
\frametitle{Defining Functions}
Function arguments are passed via a list variable \textbf{\tt{@\_}}, e.g.
\begin{perl}
sub mySub {
    @args = @_;
    print "I got ",@#args+1," args\n";
    print "They are (@args)\n";
}
\end{perl}


Note that @args is a global variable.

To make it local, precede by \textbf{\tt{my}}, e.g.
\begin{perl}
        my @args = @_;
\end{perl}

\end{frame}

\begin{frame}[fragile,shrink]
\frametitle{Defining Functions}
Can achieve similar effect to the C function
\begin{perl}
    int f(int x, int y, int z) {
        int  res;
        ...
        return res;
    }
\end{perl}


by using array assignment in Perl
\begin{perl}
    sub f {
        my ($x, $y, $z) = @_;
        my $res;
        ...
        return $res;
    }
\end{perl}

\end{frame}

\begin{frame}[fragile,shrink]
\frametitle{Defining Functions}
Lists (arrays and hashes) with any scalar arguments
to produce a single  argument list.

This in effect means you can only pass a single array or hash to
a Perl function and it must be the last argument.

\begin{perl}
    sub good {
        my ($x, $y, @list) = @_;
\end{perl}


This will not work (x and y will be undefined):

\begin{perl}
    sub bad {
        my (@list, $x, $y) = @_;
\end{perl}


And this will not work (list2 will be undefined):
\begin{perl}
    sub bad {
        my (@list1, @list2) = @_;
\end{perl}

\end{frame}

\begin{frame}[fragile,shrink]
\frametitle{References}
References
\begin{itemize}
\item  are like C pointers {\small (refer to some other objects)}
\item  can be assigned to scalar variables
\item  are dereferenced by evaluating the variable
\end{itemize}
Example:
\begin{perl}
    $aref = [1,2,3,4];
    print @$aref;    # displays whole array
    ... $$aref[0];   # access the first element
    ... ${$aref}[1]; # access the second element
    ... $aref->[2];  # access the third element
\end{perl}


\end{frame}

\begin{frame}[fragile,shrink]
\frametitle{Parameter Passing}
Scalar variable are aliased to the corresponding element of @\_.
This means a function can changed them. E.g. this code sets x to 42.
\begin{perl}
    sub assign {
            $_[0] = $_[1];
    }
    assign($x, 42);
\end{perl}

Arrays \& hashes are passed by value.

If a function needs to change an
array/hash pass a reference.

Also use references if you need to pass multiple hashes or arrays.

\begin{perl}
    %h = (jas=>100,arun=>95,eric=>50);
    @x = (1..10)
    
    mySub(3, \%h, \@x);
    mysub(2, \%h, [1,2,3,4,5]);
    mysub(5, {a=>1,b=>2}, [1,2,3]);
\end{perl}


Notation:
\begin{itemize}
\item  \textbf{\tt{[1,2,3]}} gives a reference to \textbf{\tt{(1,2,3)}}
\item  \textbf{\tt{\{a=>1,b=>2\}}} gives a reference to \textbf{\tt{(a=>1,b=>2)}}
\end{itemize}
\end{frame}

\begin{frame}[fragile,shrink]
\frametitle{Perl Prototypes}
Perl prototypes declare the expected parameter structure
for a function.

Unlike other language (e.g. C) Perl prototype's main purpose
is not type checking.

Perl prototypes' main purpose is to allow more convenient calling of  functions.

You also get some error checking - sometimes useful, sometimes less so.

Some programmers recommend against using prototypes.

Use in COMP2041/9041 optional.

You can use prototypes to define functions that can be called like builtins.
\end{frame}


\begin{frame}[fragile,shrink]
\frametitle{Perl Prototypes}

Prototypes can  cause a reference to
be passed when an array is given as a parameter.
If we define our version of push like this:

\begin{perl}
    sub mypush {
            my ($array_ref,@elements) = @_;
            if (@elements) {
                    @$array_ref = (@$array_ref, @elements);
            } else {
                    @$array_ref = (@$array_ref, $_);
            }
\end{perl}

It has to be called like this:

\begin{perl}
    mypush(\@array, $x);
\end{perl}

But if we add this prototype:

\begin{perl}
    sub mypush2(\@@)
\end{perl}

It can be called just like the builtin push:

\begin{perl}
    mypush @array, $x;
\end{perl}
\end{frame}

\begin{frame}[fragile,shrink]
\frametitle{Recursive example}
\begin{perl}
    sub fac {
        my ($n) = @_;
    
        return 1 if $n < 1;
    
        return $n * fac($n-1);
    }
\end{perl}

which behaves as
\begin{perl}
    print fac(3);   # displays 6
    print fac(4);   # displays 24
    print fac(10);  # displays 3628800
    print fac(20);  # displays 2.43290200817664e+18
\end{perl}
\end{frame}

\begin{frame}[fragile,shrink]
\frametitle{Eval}

The Perl builtin function eval evaluates (executes) a supplied string 
as Perl.

For example, this Perl  will print 43:
\begin{perl}
    $perl = '$answer = 6 * 7;';
    eval $perl;
    print "$answer\n";
\end{perl}

and this Perl  will print 55:
\begin{perl}
    @numbers = 1..10;
    $perl = join("+", @numbers);
    print eval $perl, "\n";
\end{perl}

and this Perl also prints 55:
\begin{perl}
    $perl = '$sum=0; $i=1; while ($i <= 10) {$sum+=$i++}';
    eval $perl;
    print "$sum\n";
\end{perl}

and this Perl could do anything:
\begin{perl}
    $input_line = <STDIN>;
    eval $input_line;
\end{perl}
\end{frame}

\begin{frame}
\frametitle{Pragmas and Modules}

Perl provides a way of controlling some aspects of the interpreter's behaviour
(through {\em{pragmas}}) and is an extensible language through the
use of compiled modules, of which there is a large number.
Both are introduced by the {\em{use}} keyword.

\begin{itemize}
\item 
\textbf{\tt{use English;}} - allow names for built-in vars, e.g.,
\textbf{\tt{\$NF}} = \textbf{\tt{\$.}} and \textbf{\tt{\$ARG}} = \textbf{\tt{\$\_}}.
\item 
\textbf{\tt{use integer;}} - truncate all arithmetic operations to integer,
effective to the end of the enclosing block.
\item 
\textbf{\tt{use strict 'vars';}} - insist on all variables declared using \textbf{\tt{my}}.
\end{itemize}

\end{frame}

\begin{frame}
\frametitle{Standard modules}
These are several thousand stndard Perl modules into packages.
The package name is prefixed with \textbf{\tt{::}}


Examples:
\begin{itemize}
\item 
\textbf{\tt{use DB\_File;}} - functions for maintaining an external hash
\item 
\textbf{\tt{use Getopt::Std;}} - functions for processing command-line flags
\item 
\textbf{\tt{use File::Find;}} - find function like the shell's find
\item 
\textbf{\tt{use Math::BigInt;}} - unlimited-precision arithmetic
\item 
\textbf{\tt{use CGI;}} - see next week's lecture.
\end{itemize}

\end{frame}


\begin{frame}
\frametitle{CPAN}

Comprehensive Perl Archive Network (CPAN) is an archive of 150,000+ Perl packages.

Hundreds of mirrors, including http://mirror.cse.unsw.edu.au/pub/CPAN/

Command line tools to quickly install packages from CPAN.

\end{frame}


\begin{frame}
\frametitle{CPAN}

Comprehensive Perl Archive Network (CPAN) is an archive of 150,000+ Perl packages.

Hundreds of mirrors, including http://mirror.cse.unsw.edu.au/pub/CPAN/

Command line tools to quickly install packages from CPAN.

\end{frame}

\end{document}
