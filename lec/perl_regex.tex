% https://groups.google.com/forum/#!topic/comp.text.tex/s6z9Ult_zds
\makeatletter\let\ifGm@compatii\relax\makeatother

\ifx\notes\undefined
    \documentclass{beamer}
 \else
   \documentclass[handout]{beamer}
    \mode<handout>{
        \usepackage{pgfpages}
    \pgfpagesuselayout{4 on 1}[a4paper ,border shrink=2mm, landscape]
	\pgfpageslogicalpageoptions{1}{border code=\pgfsetlinewidth{1mm}\pgfusepath{stroke}}
	\pgfpageslogicalpageoptions{2}{border code=\pgfsetlinewidth{1mm}\pgfusepath{stroke}}
	\pgfpageslogicalpageoptions{3}{border code=\pgfsetlinewidth{1mm}\pgfusepath{stroke}}
	\pgfpageslogicalpageoptions{4}{border code=\pgfsetlinewidth{1mm}\pgfusepath{stroke}}
%    \setbeamercolor{background canvas}{bg=black!3}
    }
\fi
%http://www.cpt.univ-mrs.fr/~masson/latex/Beamer-appearance-cheat-sheet.pdf
\definecolor{Dandelion}         {RGB}{253, 200, 47}
\setbeamertemplate{frametitle}
{\vspace{1mm}
\begin{centering}
\insertframetitle
\end{centering}
{\color{Dandelion} \rule{\textwidth}{1mm}}
}
\setbeamertemplate{itemize item}{$\bullet$}
\setbeamertemplate{navigation symbols}{}
%\setbeamercolor{title}{fg=Fern}
%\setbeamercolor{frametitle}{fg=Fern}
%\setbeamercolor{normal text}{fg=Charcoal}
%\setbeamercolor{block title}{fg=black,bg=Fern!25!white}
%\setbeamercolor{block body}{fg=black,bg=Fern!25!white}
%\setbeamercolor{alerted text}{fg=AlertColor}
\setbeamercolor{itemize item}{fg=black}

\usepackage[latin1]{inputenc}
\usepackage{epic,eepic,epsf}
%\usepackage{latexsym}
\usepackage{verbatim}
\usepackage{listings}
\usepackage{alltt}
\usepackage{minted}

%% all code examples use "code" environment
\newenvironment{code}{\begin{minted}{c}}{\end{minted}}

%\newcommand{\UnderscoreCommands}{\do\verbatiminput%
%\do\citeNP \do\citeA \do\citeANP \do\citeN \do\shortcite%
%\do\shortciteNP \do\shortciteA \do\shortciteANP \do\shortciteN%
%\do\citeyear \do\citeyearNP%
%}
%\usepackage[strings]{underscore}
%\usetheme{Pittsburgh}

%\makeatletter
%\setbeamercolor{background canvas}{bg=white}
%\setbeamercolor{background}{bg=white}
%\makeatother

\title[COMP1511 Introduction to Programming]{COMP1511}
\author{Andrew Taylor/John Shepherd}
\institute{CSE, UNSW}
\date{\today}

\newcommand{\red}[1]{\textcolor{red}{#1}}
\newcommand{\blue}[1]{\textcolor{blue}{#1}}
\newcommand<>{\RedNote}[2]{%
  \begin{alertblock}#3{#1}
    #2
  \end{alertblock}}
\newcommand<>{\Note}[2]{%
  \begin{exampleblock}#3{#1}
    #2
  \end{exampleblock}}
\newcommand{\NotesFrame}{\frame<handout>[plain]{\frametitle{My Notes}}}
\newcommand{\ttblue}[1]{\texttt{\blue{#1}}}

\usepackage{epstopdf}


% prefer pdf over png if both versions exist
\DeclareGraphicsExtensions{%
    .pdf,.PDF,%
    .png,.PNG,%
    .jpg,.mps,.jpeg,.jbig2,.jb2,.JPG,.JPEG,.JBIG2,.JB2}

\setbeamercolor{block body}{bg=gray!7}

\usemintedstyle{tango}

\newenvironment{C}
 {\VerbatimEnvironment
  \setbeamertemplate{blocks}[rounded][shadow=true]
  \begin{block}{}
  \begin{minted}{c}}
 {\end{minted}
  \end{block}}
  
\newenvironment{sh}
 {\VerbatimEnvironment
  \setbeamertemplate{blocks}[rounded][shadow=true]
  \begin{block}{}
  \begin{minted}{sh}}
 {\end{minted}
  \end{block}}
  
\newenvironment{perl}
 {\VerbatimEnvironment
  \setbeamertemplate{blocks}[rounded][shadow=true]
  \begin{block}{}
  \begin{minted}{perl}}
 {\end{minted}
  \end{block}}
  
\newenvironment{python}
 {\VerbatimEnvironment
  \setbeamertemplate{blocks}[rounded][shadow=true]
  \begin{block}{}
  \begin{minted}{python}}
 {\end{minted}
  \end{block}}
  
\newenvironment{html}
 {\VerbatimEnvironment
  \setbeamertemplate{blocks}[rounded][shadow=true]
  \begin{block}{}
  \begin{minted}{html}}
 {\end{minted}
  \end{block}}
  
\newenvironment{txt}
 {\VerbatimEnvironment
  \setbeamertemplate{blocks}[rounded][shadow=true]
  \begin{block}{}
  \begin{minted}{text}}
 {\end{minted}
  \end{block}}

\begin{document}
\section{Perl Programming}

\begin{frame}[fragile,shrink]
\frametitle{Perl - Regular Expressions}
Because Perl makes heavy use of strings, regular expressions
are a very important component of the language.

They can be used:
\begin{itemize}
\item  in conditional expressions to test whether a string matches a pattern

e.g. ~ checking the contents of a string
\begin{verbatim}
    if ($name =~ /[0-9]/) { print "name contains digit\n"; }
\end{verbatim}

\item  in assignments to modify the value of a string

e.g. ~ convert McDonald to MacDonald
\begin{verbatim}
    $name =~ s/Mc/Mac/;
\end{verbatim}

e.g. ~ convert to upper case
\begin{verbatim}
    $string =~ tr/a-z/A-Z/;
\end{verbatim}


\end{itemize}
\end{frame}

\begin{frame}[fragile]
\frametitle{Perl Regular Expressions}
Because Perl makes heavy use of strings, regular expressions
are a very important component of the language.

They can be used:
\begin{itemize}
\item  in conditional expressions to test whether a string matches a pattern

e.g. ~ checking the contents of a string
\begin{verbatim}
if ($name =~ /[0-9]/) { print "name contains digit\n"; }
\end{verbatim}

\item  in assignments to modify the value of a string

e.g. ~ convert McDonald to MacDonald
\begin{verbatim}
$name =~ s/Mc/Mac/;
\end{verbatim}

\end{itemize}
\end{frame}

\begin{frame}[fragile,shrink]
\frametitle{Perl Regular Expressions}
Perl extends {\small POSIX} regular expressions with some shorthand:

\begin{center}
\begin{tabular}{lll}

  \begin{minipage}{1cm}\textbf{{\textbackslash}\tt{d}} ~\end{minipage}
   & \begin{minipage}{18cm}matches any digit, i.e. \textbf{\tt{[0-9]}}~\end{minipage}
\\[1ex]

  \begin{minipage}{1cm}\textbf{{\textbackslash}\tt{D}} ~\end{minipage}
   & \begin{minipage}{18cm}matches any non-digit, i.e. \textbf{\tt{[{\textasciicircum}0-9]}}~\end{minipage}
\\[1ex]

  \begin{minipage}{1cm}\textbf{{\textbackslash}\tt{w}} ~\end{minipage}
   & \begin{minipage}{18cm}matches any "word" char, i.e. \textbf{\tt{[a-zA-Z\_0-9]}}~\end{minipage}
\\[1ex]

  \begin{minipage}{1cm}\textbf{{\textbackslash}\tt{W}} ~\end{minipage}
   & \begin{minipage}{18cm}matches any non "word" char, i.e. \textbf{\tt{[{\textasciicircum}a-zA-Z\_0-9]}}~\end{minipage}
\\[1ex]

  \begin{minipage}{1cm}\textbf{{\textbackslash}\tt{s}} ~\end{minipage}
   & \begin{minipage}{18cm}matches any whitespace, i.e. \textbf{\tt{[ {\textbackslash}t{\textbackslash}n{\textbackslash}r{\textbackslash}f]}}~\end{minipage}
\\[1ex]

  \begin{minipage}{1cm}\textbf{{\textbackslash}\tt{S}} ~\end{minipage}
   & \begin{minipage}{18cm}matches any non-whitespace, i.e. \textbf{\tt{[{\textasciicircum} {\textbackslash}t{\textbackslash}n{\textbackslash}r{\textbackslash}f]}}~\end{minipage}
\\[1ex]
\end{tabular}
\end{center}

\end{frame}

\begin{frame}[fragile,shrink]
\frametitle{Perl Regular Expressions}
Perl also adds some new anchors to regexps:

\begin{center}
\begin{tabular}{lll}

  \begin{minipage}{1cm}\textbf{\tt{{\textbackslash}b}} ~\end{minipage}
   & \begin{minipage}{18cm}matches at a word boundary~\end{minipage}
\\[1ex]

  \begin{minipage}{1cm}\textbf{\tt{{\textbackslash}B}} ~\end{minipage}
   & \begin{minipage}{18cm}matches except at a word boundary~\end{minipage}
\\[1ex]
\end{tabular}
\end{center}

And generalises the repetition operators:

\begin{center}
\begin{tabular}{lll}

  \begin{minipage}{1cm}$patt$\textbf{\tt{*}} ~\end{minipage}
   & \begin{minipage}{18cm}matches 0 or more occurences of $patt$~\end{minipage}
\\[1ex]

  \begin{minipage}{1cm}$patt$\textbf{\tt{+}} ~\end{minipage}
   & \begin{minipage}{18cm}matches 1 or more occurences of $patt$~\end{minipage}
\\[1ex]

  \begin{minipage}{1cm}$patt$\textbf{\tt{?}} ~\end{minipage}
   & \begin{minipage}{18cm}matches 0 or 1 occurence of $patt$~\end{minipage}
\\[1ex]

  \begin{minipage}{1cm}$patt$\textbf{\tt{\{}}$n$\textbf{\tt{,}}$m$\textbf{\tt{\}}} ~\end{minipage}
   & \begin{minipage}{18cm}matches between $n$ and $m$ occurences of $patt$~\end{minipage}
\\[1ex]
\end{tabular}
\end{center}

\end{frame}

\begin{frame}[fragile,shrink]
\frametitle{Perl Regular Expressions}
The default semantics for pattern matching is "first, then largest".

E.g. \textbf{\tt{/ab+/}} matches ~ \textbf{\tt{{\em{abbb}}abbbb}} ~
	not ~ \textbf{\tt{{\em{ab}}bbabbbb}} ~ or
	~ \textbf{\tt{abbb{\em{abbbb}}}}

A pattern can also be qualified so that it looks for the shortest match.

If the repetition operator is followed by \textbf{\tt{?}} the "first, then shortest"
string that matches the pattern is chosen.

E.g. \textbf{\tt{/ab+?/}} would match ~ \textbf{\tt{{\em{ab}}bbabbbb}}
\end{frame}

\begin{frame}[fragile,shrink]
\frametitle{Perl Regular Expressions}
Regular expressions can be formed by interpolating strings in between \textbf{\tt{/.../}}.

Example:
\begin{verbatim}
   $pattern = "ab+";
   $replace = "Yod";
   $text = "abba";

   $text =~ s/$pattern/$replace/;

   # converts "abba" to "Yoda"
\end{verbatim}

{\small 
Note: Perl doesn't confuse the use of \textbf{\tt{\$}} in \textbf{\tt{\$var}} and \textbf{\tt{abc\$}},
because the anchor occurs at the end.
}
\end{frame}

\begin{frame}[fragile,shrink]
\frametitle{Using Matching Results}
In a scalar context matching \& substitute operators
return how many times the match/substitute succeeded.

This allows them to be used as the controlling expression
in if/while statements.

For example:

\begin{verbatim}
print "Destroy the file system? "
$answer = <STDIN>;
if ($answer =~ /yes||ok|affirmative/i) {
   system "rm -r /";
}
\end{verbatim}


\begin{verbatim}
s/[aeiou]//g or die "now vowels to replace";
\end{verbatim}

\end{frame}

\begin{frame}[fragile,shrink]
\frametitle{Using Matching Results}
In a list context the matching operators
returns a list of the matched strings.

For example:

\begin{verbatim}
$string = "-5==10zzz200_";
@numbers = $string =~ /\d+/g;
print join(",", @numbers), "\n";
# prints 5,10,200
\end{verbatim}


If the regex contains ()s only the captured text is returned

\begin{verbatim}
$string = "Bradley, Marion Zimmer";
($family_name, $given_name) = $string =~ /([^,]*), (\S+)/;
print "$given_name $family_name\n";
# prints Marion Bradley
\end{verbatim}

\end{frame}

\begin{frame}[fragile,shrink]
\frametitle{Pattern Matcher}
A Perl script to accept a pattern and a string and show the match (if any):
\begin{verbatim}
#!/usr/bin/perl

$pattern = $ARGV[0];     print "pattern=/$pattern/\n";

$string = $ARGV[1];      print "string =\"$string\"\n";

$string =~ /$pattern/;   print "match  =\"$&\"\n";
\end{verbatim}

{\small You might find this a useful tool to test out your understanding of regular expressions.}
\end{frame}
\end{document}
