% https://groups.google.com/forum/#!topic/comp.text.tex/s6z9Ult_zds
\makeatletter\let\ifGm@compatii\relax\makeatother

\ifx\notes\undefined
    \documentclass{beamer}
 \else
   \documentclass[handout]{beamer}
    \mode<handout>{
        \usepackage{pgfpages}
    \pgfpagesuselayout{4 on 1}[a4paper ,border shrink=2mm, landscape]
	\pgfpageslogicalpageoptions{1}{border code=\pgfsetlinewidth{1mm}\pgfusepath{stroke}}
	\pgfpageslogicalpageoptions{2}{border code=\pgfsetlinewidth{1mm}\pgfusepath{stroke}}
	\pgfpageslogicalpageoptions{3}{border code=\pgfsetlinewidth{1mm}\pgfusepath{stroke}}
	\pgfpageslogicalpageoptions{4}{border code=\pgfsetlinewidth{1mm}\pgfusepath{stroke}}
%    \setbeamercolor{background canvas}{bg=black!3}
    }
\fi
%http://www.cpt.univ-mrs.fr/~masson/latex/Beamer-appearance-cheat-sheet.pdf
\definecolor{Dandelion}         {RGB}{253, 200, 47}
\setbeamertemplate{frametitle}
{\vspace{1mm}
\begin{centering}
\insertframetitle
\end{centering}
{\color{Dandelion} \rule{\textwidth}{1mm}}
}
\setbeamertemplate{itemize item}{$\bullet$}
\setbeamertemplate{navigation symbols}{}
%\setbeamercolor{title}{fg=Fern}
%\setbeamercolor{frametitle}{fg=Fern}
%\setbeamercolor{normal text}{fg=Charcoal}
%\setbeamercolor{block title}{fg=black,bg=Fern!25!white}
%\setbeamercolor{block body}{fg=black,bg=Fern!25!white}
%\setbeamercolor{alerted text}{fg=AlertColor}
\setbeamercolor{itemize item}{fg=black}

\usepackage[latin1]{inputenc}
\usepackage{epic,eepic,epsf}
%\usepackage{latexsym}
\usepackage{verbatim}
\usepackage{listings}
\usepackage{alltt}
\usepackage{minted}

%% all code examples use "code" environment
\newenvironment{code}{\begin{minted}{c}}{\end{minted}}

%\newcommand{\UnderscoreCommands}{\do\verbatiminput%
%\do\citeNP \do\citeA \do\citeANP \do\citeN \do\shortcite%
%\do\shortciteNP \do\shortciteA \do\shortciteANP \do\shortciteN%
%\do\citeyear \do\citeyearNP%
%}
%\usepackage[strings]{underscore}
%\usetheme{Pittsburgh}

%\makeatletter
%\setbeamercolor{background canvas}{bg=white}
%\setbeamercolor{background}{bg=white}
%\makeatother

\title[COMP1511 Introduction to Programming]{COMP1511}
\author{Andrew Taylor/John Shepherd}
\institute{CSE, UNSW}
\date{\today}

\newcommand{\red}[1]{\textcolor{red}{#1}}
\newcommand{\blue}[1]{\textcolor{blue}{#1}}
\newcommand<>{\RedNote}[2]{%
  \begin{alertblock}#3{#1}
    #2
  \end{alertblock}}
\newcommand<>{\Note}[2]{%
  \begin{exampleblock}#3{#1}
    #2
  \end{exampleblock}}
\newcommand{\NotesFrame}{\frame<handout>[plain]{\frametitle{My Notes}}}
\newcommand{\ttblue}[1]{\texttt{\blue{#1}}}

\usepackage{epstopdf}


% prefer pdf over png if both versions exist
\DeclareGraphicsExtensions{%
    .pdf,.PDF,%
    .png,.PNG,%
    .jpg,.mps,.jpeg,.jbig2,.jb2,.JPG,.JPEG,.JBIG2,.JB2}

\setbeamercolor{block body}{bg=gray!7}

\usemintedstyle{tango}

\newenvironment{C}
 {\VerbatimEnvironment
  \setbeamertemplate{blocks}[rounded][shadow=true]
  \begin{block}{}
  \begin{minted}{c}}
 {\end{minted}
  \end{block}}
  
\newenvironment{sh}
 {\VerbatimEnvironment
  \setbeamertemplate{blocks}[rounded][shadow=true]
  \begin{block}{}
  \begin{minted}{sh}}
 {\end{minted}
  \end{block}}
  
\newenvironment{perl}
 {\VerbatimEnvironment
  \setbeamertemplate{blocks}[rounded][shadow=true]
  \begin{block}{}
  \begin{minted}{perl}}
 {\end{minted}
  \end{block}}
  
\newenvironment{python}
 {\VerbatimEnvironment
  \setbeamertemplate{blocks}[rounded][shadow=true]
  \begin{block}{}
  \begin{minted}{python}}
 {\end{minted}
  \end{block}}
  
\newenvironment{html}
 {\VerbatimEnvironment
  \setbeamertemplate{blocks}[rounded][shadow=true]
  \begin{block}{}
  \begin{minted}{html}}
 {\end{minted}
  \end{block}}
  
\newenvironment{txt}
 {\VerbatimEnvironment
  \setbeamertemplate{blocks}[rounded][shadow=true]
  \begin{block}{}
  \begin{minted}{text}}
 {\end{minted}
  \end{block}}

\begin{document}
\section{Shells and Scripting}
\begin{frame}
\frametitle{Command Interpreters}
A {\em{command interpreter}} is a program that executes other programs.

Aim: allow users to execute the commands provided on a computer system.

Command interpreters come in two flavours:
\begin{itemize}
\item  graphical ~ (e.g. Windows or Mac desktop)
\begin{itemize}
\item  advantage: easy for naive users to start using system
\end{itemize}
\item  command-line ~ (e.g. Unix shell)
\begin{itemize}
\item  advantage: programmable, powerful tool for expert users
\end{itemize}
\end{itemize}
On Unix/Linux, \textbf{\tt{bash}} has become defacto standard shell.
\end{frame}

\begin{frame}[fragile]
\frametitle{What Shells Do}
All Unix shells have the same basic mode of operation:
\begin{verbatim}
    loop
       if (interactive) print a prompt
       read a line of user input
       apply transformations to line
       split line into words (/\s+/)
       use first word in line as command name
       execute that command,
          using other words as arguments
    end loop
\end{verbatim}

Note that "\textbf{\tt{line of user input}}" could be a line from a file.

In that case, the shell is reading a "script" of commands
and acting as a kind of programming language interpreter.
\end{frame}

\begin{frame}
\frametitle{What Shells Do}
The "\textbf{\tt{transformations}}" applied to input lines include:
\begin{itemize}
\item  variable expansion ... e.g. ~ \textbf{\tt{\$1  \$\{x-20\}}}
\item  file name expansion ... e.g. ~ \textbf{\tt{*.c  enr.07s?}}
\end{itemize}
To "\textbf{\tt{execute that command}}" the shell needs to:
\begin{itemize}
\item  find file containing named program ~ ({\em{\textbf{\tt{PATH}}}})
\item  start new process for execution of program
\end{itemize}
\end{frame}

\begin{frame}[shrink,fragile]
\frametitle{Command Search PATH}

If we have a script called {\tt{bling}} in the current directory,
we might be able to execute it with any of these:

\begin{sh}
$ sh bling   # file need not be executable
$ ./bling    # file must be executable
$ bling      # file must be executable and . in PATH
\end{sh}

Shell searches for programs to run
using the colon-separated list of directories in the variable {\tt{PATH}}.
Beware: only append the current directory to end of your path, e.g:

\begin{sh}
$ PATH=.:$PATH
$ cat >cat <<eof
#!/bin/sh
echo miaou
eof
$ chmod 755 cat
$ cat /home/cs2041/public_html/index.html
miaou
$
\end{sh}


Note ./cat is being run rather /bin/cat

Much hard to discover if it happens with
another shell script which runs cat.  Safer still: don't put  .  in your PATH.
\end{frame}

\begin{frame}
\frametitle{Unix Processes}
A Unix process executes in this environment


	\begin{figure}
    	\centering
    	\includegraphics[width = 0.9\textwidth]{Pic/unixproc}
    	% \caption{Caption}
  	\end{figure}
\end{frame}

\begin{frame}
\frametitle{Unix Processes: C Program View}
Components of process environment {\small (C programmer's view)}:
\begin{itemize}
\item  \textbf{\tt{char *argv[]}} - command line "words"
\item  \textbf{\tt{int argc}} - size of \textbf{\tt{argv[]}}
\item  \textbf{\tt{char *env[]}} - \textbf{\tt{name=value}} pairs from shell
\item  \textbf{\tt{FILE *stdin}} - input byte-stream, e.g. \textbf{\tt{getchar()}}
\item  \textbf{\tt{FILE *stdout}} - output byte-stream, e.g. \textbf{\tt{putchar()}}
\item  \textbf{\tt{FILE *stderr}} - output byte-stream, e.g. \textbf{\tt{fputc(c, stderr)}}
\item  \textbf{\tt{exit(int)}} - terminate program, set exit status
\item  \textbf{\tt{return int}} - terminate \textbf{\tt{main()}}, set exit status
\end{itemize}
\end{frame}

\begin{frame}
\frametitle{Shell as Interpreter}
The shell can be viewed as a programming language interpreter.

As with all interpreters, the shell has:
\begin{itemize}
\item  a state ~ (collection of variables and their values)
\item  control ~ (current location, execution flow)
\end{itemize}
Different to most interpreters, the shell:
\begin{itemize}
\item  modifies the program code before finally executing it
\item  has an infinitely extendible set of basic operations
\end{itemize}
\end{frame}

\begin{frame}[fragile]
\frametitle{Shell as Interpreter}
Basic operations in shell scripts are a sequence of {\em{words}}.
\begin{verbatim}
    CommandName  Arg1  Arg2  Arg3  ...
\end{verbatim}


A {\em{word}} is defined to be any sequence of:
\begin{itemize}
\item non-whitespace characters (e.g. \textbf{\tt{x, y1, aVeryLongWord}})
\item characters enclosed in double-quotes (e.g. \textbf{\tt{"abc", "a b c"}})
\item characters enclosed in single-quotes (e.g. \textbf{\tt{'abc', 'a b c'}})
\end{itemize}
We discuss the different kinds of quote later.
\end{frame}

\begin{frame}[fragile]
\frametitle{Shell Scripts}
Consider a file called "\textbf{\tt{hello}}" containing
\begin{sh}
#!/bin/sh

echo Hello, World
\end{sh}

How do we execute it?

\begin{sh}
$ sh hello           # execute the script
\end{sh}

or

\begin{sh}
$ chmod +x hello     # make the file executable
$ ./hello            # execute the script
\end{sh}

\end{frame}

\begin{frame}[fragile]
\frametitle{Shell Scripts}
The next simplest shell program: "Hello, {\it{YourName}}"
\begin{sh}
#!/bin/sh

echo -n "Enter your name: "
read name
echo Hello, $name
\end{sh}

Shell variables:

\begin{sh}
$ read x     # read a value into variable x
$ y=John     # assign a value to variable y
$ echo $x    # display the VALUE of variable x
$ z="$y $y"  # assign two copies of y to variable z
\end{sh}


{\small Note: spaces matter ... do {\it{not}} put spaces around the \textbf{\tt{=}} symbol.}
\end{frame}

\begin{frame}[fragile]
\frametitle{Shell Variables}
More on shell variables:

\begin{itemize}
\item  no need to declare shell variables; simply use them
\item  are local to the current execution of the shell.
\item  all variables have type string
\item  initial value of variable = empty string
\item  note that ~ \textbf{\tt{x=1}} ~ is equivalent to ~ \textbf{\tt{x="1"}}
\end{itemize}

Examples:

\begin{sh}
$ x=5
$ y="6"
$ z=abc
$ echo $(( $x + $y ))
11
$ echo $(( $x + $z ))
5
\end{sh}

\end{frame}

\begin{frame}[shrink,fragile]
\frametitle{Shell Variables}
\begin{sh}
$ x=1
$ y=fred
$ echo $x$y
1fred
$ echo $xy        # the aim is to display "1y"

$ echo "$x"y
1y
$ echo ${x}y
1y
$ echo ${j-10}    # give value of j or 10 if no value
10
$ echo ${j=33}    # set j to 33 if no value (and give $j)
33
$ echo ${x:?No Value}   # display "No Value" if $x not set
1
$ echo ${xx:?No Value}  # display "No Value" if $xx not set
-bash: xx: No Value
\end{sh}

\end{frame}

\begin{frame}
\frametitle{Shell Scripts}
Some shell built-in variables with pre-assigned values:
\begin{center}
\begin{tabular}{ll}
 \textbf{\tt{\$0}}  &   the name of the command  \\
 \textbf{\tt{\$1}}  &   the first command-line argument  \\
 \textbf{\tt{\$2}}  &   the second command-line argument  \\
 \textbf{\tt{\$3}}  &   the third command-line argument  \\
 \textbf{\tt{\$\#}}  &   count of command-line arguments  \\
 \textbf{\tt{\$*}}  &   all of the command-line arguments (together) \\
 \textbf{\tt{\$@}}  &   all of the command-line arguments (separately) \\
 \textbf{\tt{\$?}}  &   exit status of the most recent command \\
 \textbf{\tt{\$\$}}  &   process ID of this shell \\
\end{tabular}
\end{center}
The last one is useful for generating unique filenames.
\end{frame}

\begin{frame}
\frametitle{Shell Pathname Expansion}

If a string contains any of  \textbf{\tt{* ? []}}
it is  matched against pathnames.

\begin{itemize}
\item  \textbf{\tt{*}} matches zero or more of any character
\item  \textbf{\tt{?}} matches  any  one character
\item  \textbf{\tt{[]}} matches  any  one character between the []
\end{itemize}

The string is replaced with all matching pathnames.

If no matches the string is left unchanged (usually configurable)
\end{frame}

\begin{frame}
\frametitle{Shell Scripts}
Tip: debugging for shell scripts
\begin{itemize}
\item  the shell transforms commands before executing
\item  can be useful to know what commands are executed
\item  can be useful to know what transformations produced
\item  \textbf{\tt{set -x}} shows each command after transformation
\end{itemize}
i.e. execution trace
\end{frame}

\begin{frame}[shrink]
\frametitle{Quoting}
Quoting can be used for three purposes in the shell:
\begin{itemize}
\item  to group a sequence of words into a single "word"
\item  to control the kinds of transformations that are performed
\item  to capture the output of commands (back-quotes)
\end{itemize}
The three different kinds of quotes have three different effects:
\begin{center}

\begin{center}
\begin{tabular}{lll}

  \begin{minipage}{3cm}single-quote (\textbf{\tt{'}}) ~\end{minipage}
   & \begin{minipage}{17cm}grouping, turns off all transformations~\end{minipage}
\\[1ex]

  \begin{minipage}{3cm}double-quote (\textbf{\tt{"}}) ~\end{minipage}
   & \begin{minipage}{17cm}grouping, no transformations except \textbf{\tt{\$}} and \textbf{\tt{`}}~\end{minipage}
\\[1ex]

  \begin{minipage}{3cm}backquote ({\em{\textbf{\tt{`}}}}) ~\end{minipage}
   & \begin{minipage}{17cm}no grouping, capture command results~\end{minipage}
\\[1ex]
\end{tabular}
\end{center}

\end{center}
\end{frame}

\begin{frame}
\frametitle{Quoting}
Single-quotes are useful to pass shell meta-characters in args:
e.g. ~ \textbf{\tt{grep 'S.*[0-9]+\$' < myfile}}

Use double-quotes to
\begin{itemize}
\item  construct strings using the values of shell variables 

{\small e.g. ~ \textbf{\tt{"x=\$x, y=\$y"}} ~~ like Java's ~~ \textbf{\tt{("x=" + x + ", y=" + y)}}}
\item  prevent empty variables from "vanishing" \\

{\small e.g. ~ use ~ \textbf{\tt{test "\$x" = "abc"}} ~ rather than ~ \textbf{\tt{test \$x = "abc"}} ~ in case \textbf{\tt{\$x}} is empty}
\item  for values obtained from the command line or a user

{\small e.g. ~ use ~ \textbf{\tt{test -f "\$1" }} ~ rather than ~ \textbf{\tt{test -f \$1}} ~ in case \\
	\textbf{\tt{\$1}} contains a path with spaces e.g. \textbf{\tt{C:/Program Files/app/data}}}
\end{itemize}
\end{frame}

\begin{frame}
\frametitle{Back-quotes}
Back-quotes capture the output of commands as shell values.

For \textbf{\tt{`}}{\it{Command}}\textbf{\tt{`}}, the shell:
\begin{enumerate}
\item  performs variable-substitution (as for double-quotes)
\item  executes the resulting command and arguments
\item  captures the standard output from the command
\item  converts it to a single string
\item  uses this string as the value of the expression
\end{enumerate}
\end{frame}

\begin{frame}[fragile]
\frametitle{Back-quotes}
Example: convert GIF files to PNG format.

Original and converted files share the same prefix \\
(e.g. \textbf{\tt{/x/y/abc.gif}} is converted to \textbf{\tt{/x/y/abc.png}})
\begin{sh}
#!/bin/sh
# ungif - convert gifs to PNG format

for f in "$@"
do
    dir=`dirname "$f"`
    prefix=`basename "$f" .gif`
    outfile="$dir/$prefix.png"
    giftopnm "$f" | pnmtopng > "$outfile"
done
\end{sh}

\end{frame}

\begin{frame}[shrink]
\frametitle{Connecting Commands}
The shell provides {\em{I/O redirection}} to allow us to change
where processes read from and write to.


\begin{center}
\begin{tabular}{lll}

   \begin{minipage}{3cm}\textbf{\tt{< infile}} ~\end{minipage}
    & \begin{minipage}{18cm}connect \textbf{\tt{stdin}} to the file \textbf{\tt{infile}}~\end{minipage}
\\[1ex]

   \begin{minipage}{3cm}\textbf{\tt{> outfile}} ~\end{minipage}
    & \begin{minipage}{18cm}connect \textbf{\tt{stdout}} to the file \textbf{\tt{outfile}}~\end{minipage}
\\[1ex]

   \begin{minipage}{3cm}\textbf{\tt{>> outfile}} ~\end{minipage}
    & \begin{minipage}{18cm}apppend \textbf{\tt{stdout}} to the file \textbf{\tt{outfile}}~\end{minipage}
\\[1ex]

   \begin{minipage}{3cm}\textbf{\tt{2> outfile}} ~\end{minipage}
    & \begin{minipage}{18cm}connect \textbf{\tt{stderr}} to the file \textbf{\tt{outfile}}~\end{minipage}
\\[1ex]

   \begin{minipage}{3cm}\textbf{\tt{2>\&1 > outfile}} ~\end{minipage}
    & \begin{minipage}{18cm}connect \textbf{\tt{stderr}}+\textbf{\tt{stdout}} to \textbf{\tt{outfile}} ~\end{minipage}
\\[1ex]
\end{tabular}
\end{center}

Beware: \textbf{\tt{>}} truncates file before executing command.

Always have backups!
\end{frame}

\begin{frame}
\frametitle{Connecting Commands}
Many commands accept list of input files:

E.g. ~ \textbf{\tt{cat file1 file2 file3}} \\

These commands also typically adopt the conventions:
\begin{itemize}
\item  read contents of \textbf{\tt{stdin}} if no filename arguments
\item  treat the filename \textbf{\tt{-}} as meaning \textbf{\tt{stdin}}
\end{itemize}

E.g. ~ \textbf{\tt{cat -n < file}} ~ and ~ \textbf{\tt{cat a - b - c}} \\

If a command does not allow the use of multiple files as input, use:

E.g. ~ \textbf{\tt{cat file1 file2 file3 | Command}}
\end{frame}

\begin{frame}
\frametitle{Connecting Commands}
The shell sets up the environment for each command in a pipeline and
connects them together:


	\begin{figure}
    	\centering
    	\includegraphics[width = 0.9\textwidth]{Pic/pipeline}
    	% \caption{Caption}
  	\end{figure}
\end{frame}

\begin{frame}
\frametitle{Exit Status and Control}
Process exit status is used for control in shell scripts:
\begin{itemize}
\item  zero exit status means command successful $\rightarrow$ true
\item  non-zero exit status means error occurred $\rightarrow$ false
\end{itemize}
Mostly, exit status is simply ignored ~ {\small (e.g. when interactive)}

One application of exit statuses:
\begin{itemize}
\item  AND lists ~~ $cmd_{1}$
	\textbf{\tt{\&\&}}
	$cmd_{2}$
	\textbf{\tt{\&\& ... \&\&}}
	$cmd_{n}$ \\
	{\small ($cmd_{i+1}$ is executed only if $cmd_{i}$ succeeds (zero exit status))}
\item  OR lists ~~ $cmd_{1}$ \textbf{\tt{||}} $cmd_{2}$ \textbf{\tt{|| ... ||}} $cmd_{n}$ \\
	{\small ($cmd_{i+1}$ is executed only if $cmd_{i}$ fails (non-zero exit status))}
\end{itemize}
\end{frame}

\begin{frame}
\frametitle{Testing}
The \textbf{\tt{test}} command performs a test or combination of tests and:
\begin{itemize}
\item  returns a zero exit status if the test succeeds
\item  returns a non-zero exit status if the test fails
\end{itemize}
Provides a variety of useful testing features:
\begin{itemize}
\item  string comparison ~ ( \textbf{\tt{=  !=}} )
\item  numeric comparison ~ ( \textbf{\tt{-eq  -ne  -lt}} )
\item  checks on files ~ ( \textbf{\tt{-f  -x  -r}} )
\item  boolean operators ~ ( \textbf{\tt{-a  -o  !}} )
\end{itemize}
\end{frame}

\begin{frame}[fragile,shrink]
\frametitle{Testing}
Examples:
\begin{sh}
# does the variable msg have the value "Hello"?
test "$msg" = "Hello"

# does x contain a numeric value larger than y?
test "$x" -gt "$y"

# Error: expands to "test hello there = Hello"?
msg="hello there"
test $msg = Hello

# is the value of x in range 10..20?
test "$x" -ge 10 -a "$x" -le 20

# is the file xyz a readable directory?
test -r xyz -a -d xyz

# alternative syntax; requires closing ]
[ -r xyz -a -d xyz ]
\end{sh}

{\small Note: use of quotes, spaces around values/operators}
\end{frame}

\begin{frame}
\frametitle{Sequential Execution}
Combine commands in pipelines and AND and OR lists.

Commands executed sequentially if separated by semicolon or newline.

$cmd_{1}$ ; $cmd_{2}$ ; ... ; $cmd_{n}$

$cmd_{1}$

$cmd_{2}$

...

$cmd_{n}$
\end{frame}

\begin{frame}[shrink,fragile]
\frametitle{Grouping}
Commands can be grouped using \textbf{\tt{ ( ... ) }} or \textbf{\tt{ \{ ... \} }}

\textbf{\tt{(}}$cmd_{1}$ \textbf{\tt{; ...}} $cmd_{n}$) \textbf{\tt{}} are executed in a new sub-shell.

\textbf{\tt{\{}}$cmd_{1}$ \textbf{\tt{; ...}} $cmd_{n}$ \textbf{\tt{\}}} are executed in the current shell.

Exit status of group is exit status of last command.

Beware: state of sub-shell (e.g. \textbf{\tt{\$PWD}}, other variables) is lost after \textbf{\tt{(...)}},
hence
\begin{sh}
$ cd /usr/share
$ x=123
$ ( cd $HOME;  x=abc; )
$ echo $PWD $x
/usr/share 123
$ { cd $HOME;  x=abc; }
$ echo $PWD $x
/home/cs2041 abc
\end{sh}


\end{frame}

\begin{frame}[fragile]
\frametitle{If Command}
The \textbf{\tt{if-then-else}} construct allows conditional execution:
\begin{sh}
if testList{1}
then
   commandList{1}
elif testList{2}
then
   commandList{2}
   ...
else
   commandList{n}
fi
\end{sh}

Keywords \textbf{\tt{if}}, \textbf{\tt{else}} etc, are only recognised
at the start of a command (after newline or semicolon).
\end{frame}

\begin{frame}[fragile]
\frametitle{If Command}
Examples:
\begin{sh}
# Check whether a file is readable

if [ -r $HOME ]     # neater than:  if test -r $HOME
then
   echo "$0: $HOME is readable"
fi

# Test whether a user exists in passwd file

if grep "^$user" /etc/passwd > /dev/null
then
    # do something if they do exist ...
else
   echo "$0: $user does not exist"
fi
\end{sh}

\end{frame}

\begin{frame}[fragile]
\frametitle{Case command}
\textbf{\tt{case}} provides multi-way choice based on patterns:
\begin{sh}
case word in
pattern{1})  commandList{1} ;;
pattern{2}-2)  commandList{2}-2 ;;
...
pattern{n})  commandList{n} ;;
esac
\end{sh}

The $word$ is compared to each $pattern_{i}$ in turn.

For the first matching pattern, corresponding $commandList_{i}$
is executed and the statement finishes.

Patterns are those used in filename expansion ~ ( \textbf{\tt{* ? []}} ).
\end{frame}

\begin{frame}[fragile,shrink]
\frametitle{Case command}
Examples:
\begin{sh}
# Checking number of command line args

case $# in
0)  echo "You forgot to supply the argument" ;;
1)  ... process the argument ...  ;;
*)  echo "You supplied too many arguments" ;;
esac

# Classifying a file via its name

case "$file" in
*.c)  echo "$file looks like a C source-code file" ;;
*.h)  echo "$file looks like a C header file" ;;
*.o)  echo "$file looks like a an object file" ;;
...
?)    echo "$file's name is too short to classify" ;;
*)    echo "I have no idea what $file is" ;;
esac         
\end{sh}
\end{frame}

\begin{frame}[fragile]
\frametitle{Loop commands}
\textbf{\tt{while}} loops iterate based on a test command list:
\begin{sh}
while testList
do
   commandList
done
\end{sh}

\textbf{\tt{for}} loops set a variable to successive words from a list:
\begin{sh}
for var in wordList
do
   commandList  # ... generally involving var
done
\end{sh}

\end{frame}

\begin{frame}[fragile]
\frametitle{Loop commands}
Examples of \textbf{\tt{while}}:
\begin{sh}
# Check the system status every ten minutes

while true
do
   uptime ; sleep 600
done

# Interactively prompt the user to process files

echo -n "Next file: "
while read filename
do
   process < "$filename" >> results
   echo -n "Next file: "
done
\end{sh}

\end{frame}

\begin{frame}[fragile,shrink]
\frametitle{Loop commands}
Examples of \textbf{\tt{for}}:
\begin{sh}
# Compute sum of a list of numbers from command line

sum=0
for n in "$@"    # use "$@" to preserve args
do
   sum=`expr $sum + "$n"`
done

# Process files in $PWD, asking for confirmation

for file in *
do
   echo -n "Process $file? "
   read answer
   case "$answer" in
      [yY]*) process < $file >> results ;;
      *)     ;;
   esac
done
\end{sh}

\end{frame}

\end{document}
