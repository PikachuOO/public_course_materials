% https://groups.google.com/forum/#!topic/comp.text.tex/s6z9Ult_zds
\makeatletter\let\ifGm@compatii\relax\makeatother

\ifx\notes\undefined
    \documentclass{beamer}
 \else
   \documentclass[handout]{beamer}
    \mode<handout>{
        \usepackage{pgfpages}
    \pgfpagesuselayout{4 on 1}[a4paper ,border shrink=2mm, landscape]
	\pgfpageslogicalpageoptions{1}{border code=\pgfsetlinewidth{1mm}\pgfusepath{stroke}}
	\pgfpageslogicalpageoptions{2}{border code=\pgfsetlinewidth{1mm}\pgfusepath{stroke}}
	\pgfpageslogicalpageoptions{3}{border code=\pgfsetlinewidth{1mm}\pgfusepath{stroke}}
	\pgfpageslogicalpageoptions{4}{border code=\pgfsetlinewidth{1mm}\pgfusepath{stroke}}
%    \setbeamercolor{background canvas}{bg=black!3}
    }
\fi
%http://www.cpt.univ-mrs.fr/~masson/latex/Beamer-appearance-cheat-sheet.pdf
\definecolor{Dandelion}         {RGB}{253, 200, 47}
\setbeamertemplate{frametitle}
{\vspace{1mm}
\begin{centering}
\insertframetitle
\end{centering}
{\color{Dandelion} \rule{\textwidth}{1mm}}
}
\setbeamertemplate{itemize item}{$\bullet$}
\setbeamertemplate{navigation symbols}{}
%\setbeamercolor{title}{fg=Fern}
%\setbeamercolor{frametitle}{fg=Fern}
%\setbeamercolor{normal text}{fg=Charcoal}
%\setbeamercolor{block title}{fg=black,bg=Fern!25!white}
%\setbeamercolor{block body}{fg=black,bg=Fern!25!white}
%\setbeamercolor{alerted text}{fg=AlertColor}
\setbeamercolor{itemize item}{fg=black}

\usepackage[latin1]{inputenc}
\usepackage{epic,eepic,epsf}
%\usepackage{latexsym}
\usepackage{verbatim}
\usepackage{listings}
\usepackage{alltt}
\usepackage{minted}

%% all code examples use "code" environment
\newenvironment{code}{\begin{minted}{c}}{\end{minted}}

%\newcommand{\UnderscoreCommands}{\do\verbatiminput%
%\do\citeNP \do\citeA \do\citeANP \do\citeN \do\shortcite%
%\do\shortciteNP \do\shortciteA \do\shortciteANP \do\shortciteN%
%\do\citeyear \do\citeyearNP%
%}
%\usepackage[strings]{underscore}
%\usetheme{Pittsburgh}

%\makeatletter
%\setbeamercolor{background canvas}{bg=white}
%\setbeamercolor{background}{bg=white}
%\makeatother

\title[COMP1511 Introduction to Programming]{COMP1511}
\author{Andrew Taylor/John Shepherd}
\institute{CSE, UNSW}
\date{\today}

\newcommand{\red}[1]{\textcolor{red}{#1}}
\newcommand{\blue}[1]{\textcolor{blue}{#1}}
\newcommand<>{\RedNote}[2]{%
  \begin{alertblock}#3{#1}
    #2
  \end{alertblock}}
\newcommand<>{\Note}[2]{%
  \begin{exampleblock}#3{#1}
    #2
  \end{exampleblock}}
\newcommand{\NotesFrame}{\frame<handout>[plain]{\frametitle{My Notes}}}
\newcommand{\ttblue}[1]{\texttt{\blue{#1}}}

\usepackage{epstopdf}


% prefer pdf over png if both versions exist
\DeclareGraphicsExtensions{%
    .pdf,.PDF,%
    .png,.PNG,%
    .jpg,.mps,.jpeg,.jbig2,.jb2,.JPG,.JPEG,.JBIG2,.JB2}

\setbeamercolor{block body}{bg=gray!7}

\usemintedstyle{tango}

\newenvironment{C}
 {\VerbatimEnvironment
  \setbeamertemplate{blocks}[rounded][shadow=true]
  \begin{block}{}
  \begin{minted}{c}}
 {\end{minted}
  \end{block}}
  
\newenvironment{sh}
 {\VerbatimEnvironment
  \setbeamertemplate{blocks}[rounded][shadow=true]
  \begin{block}{}
  \begin{minted}{sh}}
 {\end{minted}
  \end{block}}
  
\newenvironment{perl}
 {\VerbatimEnvironment
  \setbeamertemplate{blocks}[rounded][shadow=true]
  \begin{block}{}
  \begin{minted}{perl}}
 {\end{minted}
  \end{block}}
  
\newenvironment{python}
 {\VerbatimEnvironment
  \setbeamertemplate{blocks}[rounded][shadow=true]
  \begin{block}{}
  \begin{minted}{python}}
 {\end{minted}
  \end{block}}
  
\newenvironment{html}
 {\VerbatimEnvironment
  \setbeamertemplate{blocks}[rounded][shadow=true]
  \begin{block}{}
  \begin{minted}{html}}
 {\end{minted}
  \end{block}}
  
\newenvironment{txt}
 {\VerbatimEnvironment
  \setbeamertemplate{blocks}[rounded][shadow=true]
  \begin{block}{}
  \begin{minted}{text}}
 {\end{minted}
  \end{block}}

\begin{document}
\section{COMP[29]041 Introduction 17s2}
\begin{frame}
~
\frametitle{COMP[29]041 - Software Construction 17s2}


\begin{center}
Lecturer/Admin: ~ Andrew Taylor, ~ {\small andrewt{\makeatletter@\makeatother}cse.unsw.edu.au} \\


~ http://www.cse.unsw.edu.au/{\textasciitilde}cs2041/
\end{center}
\end{frame}

\begin{frame}
\frametitle{Course Goals}
Overview: to expand your knowledge of programming.

First year deals with ...
\begin{itemize}
\item  some aspects of programming {\small (e.g. basics, correctness)}
\item  on relatively small tightly-specified examples
\end{itemize}
COMP[29]041 deals with ...
\begin{itemize}
\item  other aspects of programming {\small (e.g. testing, performance)}
\item  using a wider range of tools {\small (e.g. filters, Perl)}
\item  on larger (less small) less specified examples
\end{itemize}
\end{frame}

\begin{frame}
\frametitle{Course Goals}
Introduce you to:
\begin{itemize}
\item  building software systems from components
\item  treating software as an object of experimental study
\end{itemize}
Develop skills in:
\begin{itemize}
\item  using software development tools {\small (e.g.  make, gprof, svn)}
\item  building reliable, efficient, maintainable, portable software
\end{itemize}
{\small 
Ultimately: get you to the point where you could build some
software, put it on sourceforge/google code/github, have people use it and have
it rated well.
}
\end{frame}

\begin{frame}
\frametitle{Inputs}
At the start of this course you should be able to:
\begin{itemize}
\item  produce a correct procedural program from a spec
\item  understand fundamental data structures + algorithms \\
	{\small (char, int, float, array, struct, pointers, sorting, searching)}
\item  appreciate the use of abstraction in computing
\end{itemize}
\end{frame}

\begin{frame}
\frametitle{Outputs}
At the end of this course you should be able to:
\begin{itemize}
\item  understand the capabilities of many programming tools
\item  choose an appropriate tool to solve a given problem
\item  apply that tool to develop a software solution
\item  use appropriate tools to assist in the development 
\item  show that your solution is reliable, efficient, portable
\end{itemize}
\end{frame}

\begin{frame}[shrink]
\frametitle{Syllabus Overview}
\begin{enumerate}
\item  Qualities of software systems
{\small 
\begin{itemize}
\item  Correctness, clarity, reliability, efficiency, portability, ...
\end{itemize}
}
\item  Techniques for software construction
{\small 
\begin{itemize}
\item  Analysis, design, coding, testing, debugging, tuning
\item  Interface design, documentation, configuration
\end{itemize}
}
\item  Tools for software construction
{\small 
\begin{itemize}
\item  Filters ~ ({\em{grep, sed, cut, sort, uniq, tr,}}...)
\item  Scripting languages ~ ( {\em{shell}}, {\em{Perl}}, some Python)
\item  Intro to Programming for the we
\item  Analysis/configuration/documentation tools ~ (git, gprof, make, ...)
\end{itemize}
}
\end{enumerate}
\end{frame}

\begin{frame}
\frametitle{Lectures}
Lectures will:
\begin{itemize}
\item  present a brief overview of theory
\item  give practical demonstrations of tools
\item  demonstrate problem-solving methods
\end{itemize}
Lecture notes available on the Web before each lecture.

Feel free to ask questions, but otherwise {\em{Keep Quiet}}.
\end{frame}

\begin{frame}
\frametitle{Tutorials}
Tutorials aim to:
\begin{itemize}
\item  clarify any problems with lecture material
\item  work through problems related to lecture topics
\item  give practice with design skills ~ {\small {\em{(think before coding)}}}
\end{itemize}

Tutorials start in week 2.

Tutorial questions available on the web the week before.

Tutorial answers available on the web after the week's last tutorial.

Use tutorials to discuss {\em{how}} solutions were reached.
\end{frame}

\begin{frame}
\frametitle{Tutorials}

To get the best out of tutorials

\begin{itemize}
\item  attempt the problems yourself
\item  if you understand OK and get feasible answer, fine
\end{itemize}

Ask your tutor if ...

\begin{itemize}
\item  if you aren't sure your answer is feasible/correct
\item  if you don't know {\em{how}} the solution was reached
\item  if you don't understand a question or how to solve it
\end{itemize}

Do {\em{not}} keep quiet in tutorials ... talk, discuss, ...

Your tutor may ask for your attempt to start a discussion.

\end{frame}

\begin{frame}
\frametitle{Lab Classes}
Each tutorial is followed by a two-hour lab class.

Lab exercises aim to build skills that will help you to

\begin{itemize}
\item  complete the assignments
\item  pass the final exam
\end{itemize}

Lab classes give you experience applying tools/techniques.

Each lab exercise is a small implementation/analysis task.

Labs often includes challenge exercise(s)

{\em{Do them yourself!}} 
\end{frame}

\begin{frame}
\frametitle{Lab Classes}
Lab exercises contribute 9\% to overall mark.

In order to get a marks most lab exercises:

\begin{itemize}
\item submitted via \textbf{\tt{give}} before Sunday midnight
\item automarked (partial marks if fails autotests)
\item tutors will give feedback separately
\end{itemize}

Some lab exercises must be completed and/or assessed during lab class.

Most labs will contain challenge exercises.

More marks available for lab exercises than needed for full marks for lab component.

Can miss 1 week and some challenge exercises and still get full marks.

Any submission of work not your own - zero for lab component.

\end{frame}

\begin{frame}
\frametitle{Lab Exemption}
COMP9041 students can apply for lab exemption if they have:

\begin{itemize}
\item  full-time work or other commitments
\item  previous  programming, particularly scripting, experience
\item  good marks (70+) in previous courses
\end{itemize}

If granted - final mark calculated without lab component.

Strongly recommended they still  complete labs.

\end{frame}

\begin{frame}
\frametitle{Weekly Programming Tests}
Programming tests in weeks 5-12 contribute 6\% to overall mark.

Immediate reality-check on your progress.

Done in your own time under self-enforced exam conditions.

Each test specifies conditions, typically:

\begin{itemize}
\item No assistance from any person.
\item Time limit of 1 hour
\item No access to materials (written or online) \\
except language documentation or man pages.

\item  automarked:

\begin{itemize}
\item 100\% passes all autotests
\item 75\% passes most autotests (single bug)
\item 50\% otherwise
\end{itemize}
\item  best 6 of 8 tests used to calculate the 6%
\item  any violation of the test conditions, zero for whole component
\end{itemize}

\end{frame}

\begin{frame}
\frametitle{Assignments}
Assignments give you experience applying tools/techniques \\
{\small (but to larger programming problems than the lab exercises)}

Assignments will be carried out individually.

They always take longer than you expect.

Don't leave them to the last minute.

There are late penalties applied to maximum assignment marks,
typically:
\begin{itemize}
\item  2\%/hour
\end{itemize}
Organising your time $\Rightarrow$ no penalty.
\end{frame}

\begin{frame}
\frametitle{Plagiarism}
Labs and Assignments must be entirely your own work.

No group work in this course.

{\em{{\bf{Plagiarism}}}} = {\small submitting someone else's work as your own.}

Plagiarism will be checked for and {\em{penalized}}.

Plagiarism may result in suspension from UNSW .

International students may lose their visa.

Supplying your work to any another person  may result in 
loss of all your marks for the lab/assignment.

Assignments may allow use of code snippets ($<$ 10 lines)
with {\bf attribution}

For example, can use 3 lines of code from Stack Overflow if clear you are not author
\end{frame}

\begin{frame}
\frametitle{Final Exam}
3-hour closed-book exam during the exam period.

The exam contributes 55\% to total mark for the course.

Some multiple-choice/short answer questions - similar to  tut questions.

Six (probably) implementation questions  - similar to lab exercises.

COMP[29]041 hurdle requirement:

\begin{itemize}
\item you must successfully complete two implementation questions
\end{itemize}

A question still regarded as complete if it contains a minor bug.

You can not pass COMP[29]041 unless you meet the hurdle.

If you fail to meet  hurdle and get 50+, you receive a UF grade.
\end{frame}

\begin{frame}
\frametitle{Final Exam}
Format:
Held in the CSE Labs ~ {\small (must know lab environment)}

\begin{itemize}
\item  limited on-line language documentation available
\item  we give you  programming task
\item  most tasks you choose the language from (sh,perl,python,C) \\
a task might specify particular language
\item  similar in style/difficulty to labs
\end{itemize}
\end{frame}


\begin{frame}
\frametitle{Supplementary Exams}
Supplementary exam for students who miss the exam due a serious documented reason (e.g illness)

Supplementary exam also offered to students:
\begin{itemize}
\item final mark 40+
\item attended 9+ tutorials
\item attempted 9+ labs
\item attempted 6+ quizes
\item attempted both assignments (achieving 50+%)
\end{itemize}

If you are close to passing and have been taking the course seriously,
we'll give you a second chance.

Includes students who don't meet hurdle requirement on final exam.

Supplementary exam tentatively scheduled for  Dec 6 

You responsibility to be in Sydney around that date and available.
\end{frame}


\begin{frame}
\frametitle{How to Pass this Course}
Coding is a {\em{skill}} that improves with practice.

The more you practise, the easier you will find assignments/exams.

Don't restrict practice to lab times and two days before assignments due.

It also helps to pay attention in lectures and tutorials.
\end{frame}

\begin{frame}
\frametitle{Reading Material}
{\bf{General References:}}
\begin{itemize}
\item  {\em{Kernighan \& Pike}},
	The Practice of Programming,
	\\{\small Addison-Wesley, 1998.
	\\(Inspiration for 2041 - philosophy and some tool details)}
\item  {\em{McConnell}},
	Code Complete (2ed),
	\\{\small Microsoft Press, 2004.
	\\(Many interesting case studies and practical ideas)}
\end{itemize}
\end{frame}

\begin{frame}[shrink]
\frametitle{Reading Material}
{\bf{Perl Reference Books:}}
\begin{itemize}
\item  {\em{Wall, Christiansen \& Orwant }},
	Programming Perl (3ed),
	\\{\small O'Reilly, 2000.
		~ (Original \& best Perl reference manual)}
\item  {\em{Schwartz, Phoenix \& Foy}},
	Learning Perl (5ed),
	\\{\small O'Reilly, 2008.
		~ (gentle \&  careful introduction to Perl)}
\item  {\em{Christiansen \& Torkington}},
	Perl Cookbook (2ed),
	\\{\small O'Reilly, 2003.
		~ (Lots and lots of interesting Perl examples)}
\item  {\em{Schwartz \& Phoenix}},
	Learning Perl Objects, References, and Modules (2ed),
	\\{\small O'Reilly, 2003.
		~ (gentle \&  careful introduction to parts of Perl mostly  not covered in this course)}
\item  {\em{Schwartz, Phoenix \& Foy}},
	Intermediate Perl  (2ed),
	\\{\small O'Reilly, 2008.
		~ (good book to read after 2041 - starts where this course finishes)}
\item  {\em{Sebesta}},
	A Little Book on Perl,
	\\{\small Prentice Hall, 1999.
		~ (Modern, concise introduction to Perl)}
\item  {\em{Orwant, Hietaniemi, MacDonald}},
	Mastering Algorithms with Perl,
	\\{\small O'Reilly, 1999.
		~ (Algorithms and data structures via Perl)}
\end{itemize}
\end{frame}

\begin{frame}
\frametitle{Reading Material}
{\bf{Shell Programming:}}
\begin{itemize}
\item  {\em{Kochgan \& Wood 2003}},
	Unix® Shell Programming,
	\\{\small Sams Publishing 2003
		~ (Careful intoduction to Shell Programming)} 
\item  {\em{Peek, O'Reilly, Loukides}},
	Bash Cookbook,
	\\{\small O'Reilly, 2007.
		~ (Recipe(example) based intro to Shell programming)}
\end{itemize}
\end{frame}

\begin{frame}
\frametitle{Reading Material}
{\bf{Unix Tools Reference Books:}}
\begin{itemize}
\item  {\em{Powers, Peek, O'Reilly, Loukides}},
	Unix Power Tools (3ed),
	\\{\small O'Reilly, 2003.
		~ (Comprehensive guide to common Unix tools)}
\item  {\em{Loukides \& Oram}},
	Programming with GNU Software,
	\\{\small O'Reilly, 1997.
		~ (Tutorial on the GNU programming tools (gcc,gdb,...))} 
\item  {\em{Robbins}},
	Unix in a Nutshell (4ed),
	\\{\small O'Reilly, 2006.
		~ (Concise guide to Unix and its toolset)}
\item  {\em{Kernighan \& Pike}},
	The Unix Programming Environment,
	\\{\small Prentice Hall, 1984.
		~ (Pre-cursor to the textbook, intro to Unix tools)}
\end{itemize}
\end{frame}

\begin{frame}
\frametitle{Reading Material}
All tools in the course have extensive on-line documentation.

Links to this material are available in the course Web pages.

You are expected to master these systems largely by reading the manuals.

However ...
\begin{itemize}
\item  we will also give introductory lectures on them 
\item  the lab exercises will give practice in using them
\end{itemize}
{\em{Note:}} {\it{"The ability to read software manuals is an invaluable skill"}} ~ {\small (jas,1999)}
\end{frame}

\begin{frame}
\frametitle{Home Computing}
All of the tools in the course are available under Unix and Linux.

Most have also been ported to {\small } MS Windows. \\
{\small (generally via the CygWin project)}

All tools should be available on Mac. \\
{\small (given that Mac OS X is based on FreeBSD Unix)}

Links to downloads will be placed on course Web site.
\end{frame}

\begin{frame}
\frametitle{Home Computing}
There may be minor incompatibilities between Unix, Windows and Mac
versions of tools.

Therefore ... {\em{test your assignments at CSE}} before you submit them.

{\small 
{\em{Note:}} we expect that any software that {\em{you}}
produce will be portable to all platforms.

This is accomplished by adhering to {\it{standards}}.
}
\end{frame}

\begin{frame}
\frametitle{Conclusion}
The goal is for you to become a better programmer
\begin{itemize}
\item  more confident in your own ability
\item  producing a better end-product
\item  ultimately, enjoying the programming process
\end{itemize}
\end{frame}

\end{document}
