\input{include.tex}
\begin{document}
\section{Version Control}
\begin{frame}
\frametitle{Version Control}

Any large, useful software system ...

\begin{itemize}
\item  will undergo many changes in its lifetime
\item  multiple programmers making changes
\item  who may work on the code concurrently and independently
\end{itemize}

The process of code change needs to be managed so that

\begin{itemize}
\item  changes produce "consistent" versions of the system
\item  many programmers can easily work simultaneously
\item  we can roll back to earlier version if needed
\item  documentation of when, who, \& why changes made
\item  multiple versions of system can be distributed, tested, merged
\end{itemize}

\end{frame}

\begin{frame}
\frametitle{Version Control}
Consider the following simple scenario:
\begin{itemize}
\item  a software system contains a source code file \textbf{\tt{x.c}}
\item  system is worked on by several teams of programmers
\item  Ann in Sydney adds a new feature in her copy of \textbf{\tt{x.c}}
\item  Bob in Singapore fixes a bug in his copy of \textbf{\tt{x.c}}
\end{itemize}
Ultimately, we need to ensure that
\begin{itemize}
\item  all changes are properly recorded ~ {\small (when, who, why)}
\item  both the new feature and the bug fix are in the next release
\item  if we later find bugs in old release, they can be fixed
\end{itemize}
\end{frame}

\begin{frame}
\frametitle{Human Version Control}
Manual solution to above problems:
\begin{itemize}
\item  Ann requests/receives a copy of \textbf{\tt{x.c}} v2.1 from Manager
\item  Bob requests/receives a copy of \textbf{\tt{x.c}} v2.1 from Manager
\item  Ann sends her new version of \textbf{\tt{x.c}} back to Manager
\item  Bob sends his new version of \textbf{\tt{x.c}} back to Manager
\item  Manager ensures that all changes are incorporated
{\small 
\begin{itemize}
\item  might need to ask Ann and Bob for help if changes conflict
\end{itemize}
}
\item  Manager sets up the merged version as v2.2 of \textbf{\tt{x.c}}
\item  Col requests/receives a copy of \textbf{\tt{x.c}} v2.2 from Manager
\end{itemize}
Problem: eventually Manager not available 24x7 \& as system scales
will be overwhelmed with requests.
\end{frame}

\begin{frame}
\frametitle{Version Control Systems}
{\em{Version control systems}} aim to solve the above problems.

Version control systems (VCSs) are also called ...
\begin{itemize}
\item  revision control systems
\item  source (code) control systems
\item  (source) code management systems
\end{itemize}
There are various approaches to solving the problems, \\
leading to different families of version control systems.

While VCSs could be used for all kinds of documents, \\
we focus on their use for managing source code files.
\end{frame}

\begin{frame}
\frametitle{Version Control Systems}
A {\em{version control system}} allows software developers to:
\begin{itemize}
\item  share development work on a system
\item  recreate old versions of a system when needed
\item  identify the current versions of source code files
\item  restrict who is allowed to modify each source code file
\end{itemize}
This allows change to be managed/controlled in a systematic way.

VCSs also try to minimise resource use in maintaining multiple versions.
\end{frame}

\begin{frame}
\frametitle{Labelling Versions}
How to name an important version? ~ {\small (unique identifier)}

Common approach: file name + version "number" (e.g. Perl 5.8.1)

No "standard" for $a.b.c$ version numbers, but typically:
\begin{itemize}
\item  $a$ is major version number {\small (changes when functionality/API changes)}
\item  $b$ is minor version number {\small (changes when internals change)}
\item  $c$ is patch level {\small (changes after each set of bug fixes are added)}
\end{itemize}
{\small 
Examples: ~ Oracle 7.3.2, 8.0.5, 8.1.5, ... 
}
\end{frame}

\begin{frame}
\frametitle{Labelling Versions}
How to store multiple versions (e.g. v$3.2.3$ and v$3.2.4$)?

We {\em{could}} simply store one complete file for each version.

Alternative approach:
\begin{itemize}
\item  store complete file for version $3.2.3$
\item  store differences between $3.2.3$ and $3.2.4$
\end{itemize}
A {\em{delta}} is the set of differences between two successive
versions of a file.

Many VCSs store a combination of complete versions and deltas for each file.

\end{frame}

\begin{frame}[fragile]
\frametitle{Labelling Versions}
Creating version $N$ of file F (F$_{N}$) from a collection of
\begin{itemize}
\item  complete copies of F whose versions $< N$
\item  deltas for all versions in between complete copies
\end{itemize}
is acheived via:
\begin{sh}
    get list of complete copies of F
    choose highest complete version V << N
    f = copy of F{V}
    foreach delta between V .. N {
       f = f + delta
    }
    # f == F{N} (i.e. version N of F)
\end{sh}

Programs like \textbf{\tt{patch}} can apply deltas.
\end{frame}

\begin{frame}
\frametitle{Delta Bandwidth Efficiency}
An example of why deltas are useful:
{\small 
\begin{itemize}
\item  Google Chrome for Windows upgrades in the background almost daily
\item  Google Chrome is ~10MB
\item  bsdiff delta = 700KB
\item  google custom delta (Courgette) = 80KB
\item  200 full upgrades/year = 2GB/year
\item  200 bsdiff upgrades/year = 140Mb/year
\item  200 Courgette upgrades/year = 16Mb/year
\end{itemize}
}
\end{frame}

\begin{frame}
\frametitle{Unix VCS - Generation 1 (Unix)}
1970's ... {\em{SCCS}} (source code control system)
\begin{itemize}
\item  first version control system
\item  centralized VCS - single central repository
\item  introduced idea of multiple versions via delta's
\item  single user model: lock - modify - unlock
\item  only one user working on a file at a time 
\end{itemize}

1980's ... {\em{RCS}} (revision control system)
\begin{itemize}
\item  similar functionality to SCCS \\
    {\small (essentially a clean open-source re-write of SCCS)}
\item  centralized VCS - single central repository
\item  single user model: lock - modify - unlock
\item  only one user working on a file at a time 
\item  still available and in use 
\end{itemize}
\end{frame}

\begin{frame}
\frametitle{Unix VCS - Generation 2 (Unix)}
~1990 ... {\em{CVS}} (concurrent version system)
\begin{itemize}
\item  centralized VCS - single central repository
\item  locked check-out replaced by copy-modify-merge model
\item  users can work simultaneously and later merge changes
\item  allows remote development essential for open source projects
\item  web-accessible interface promoted wide-dist projects
\item  poor handling of file metadata, renames, links 
\end{itemize}

Early 2000's ... {\em{Subversion}} (svn)
\begin{itemize}
\item  depicted as "CVS done right"
\item  many cvs weakness fixed 
\item  solid, well documented, widely used system
\item  but essentially the same model as CVS
\item  centralized VCS - single central repository
\item  svn is suitable for  assignments/small-medium projects
\item  easier to understand than distributed VCSs, well supported
\item  but Andrew recommends git
\end{itemize}
\end{frame}

\begin{frame}
\frametitle{Unix VCS - Generation 3}
Early 2000s... {\em{Bitkeeper}} 
\begin{itemize}
\item  distributed VCS - multiple repositories, no "master"
\item  every user has their own repository
\item  written by Larry McVoy
\item  Commercial system but allowed limited use for Linux kernel until dispute over  licensing issues
\item  Linus Torvalds + others  then wrote GIT open source distributed VCS
\item  Other open source distributed VCS's appeared, e.g: bazaar {\small (Canonical!)}, darcs {\small (Haskell!)}, Mercurial
\end{itemize}
\end{frame}

\begin{frame}
\frametitle{Git}
\begin{itemize}
\item  distributed VCS - multiple repositories, no "master"
\item  every user has their own repository
\item  created by Linux Torvalds for Linux kernel
\item  external revisions imported as new branches
\item  flexible handling of branching
\item  various auto-merging algorithms
\item  Andrew recommends you use git unless good reason not to
\item  not better than competitors but better supported/more widely used (e.g. github/bitbucket)
\item  at first stick with a small subset of commands
\item  substantial time investment to learn to use Git's  full power
\end{itemize}
\end{frame}

\begin{frame}
\frametitle{Repository}
Many VCSs use the notion of a {\em{repository}}

\begin{itemize}
\item  store all versions of all objects (files) managed by VCS
\item  may be  single file, directory tree, database,...
\item  possibly accessed by filesystem, http, ssh or custom protocol
\item  possibly structured as a collection of {\em{projects}}
\end{itemize}


\end{frame}


\begin{frame}
\frametitle{Git Repository}
Git uses the sub-directory {\tt .git} to store the repository.

Inside  {\tt .git} there are:

\begin{itemize}
\item  {\bf blobs} file contents identified by SHA-1 hash
\item  {\bf tree objects} links blobs to info about directories, link, permssions (limited)
\item  {\bf commit objects} links trees objects with info about parents, time, log message
\end{itemize}

\begin{itemize}
\item  Create repository {\bf git init}
\item  Copy exiting repository {\bf git clone}
\end{itemize}
\end{frame}

\begin{frame}
\frametitle{Tracking a Project with Git}

\begin{itemize}
\item  Project must be in single directory tree.

\item Usually don't want to track all files in directory tree

\item Don't track binaries, derived files, temporary files, large static files

\item Use {\bf .gitignore} files  to indicate files never want to track

\item Use {\bf git add {\it file}} to indicate you want to track {\it file}

\item Careful: {\bf git add {\it directory}} will every file in {\it file} and sub-directories
\end{itemize}

\end{frame}

\begin{frame}
\frametitle{Git Commit}

\begin{itemize}
\item A git commit is a snapshot of all the files in the project.

\item Can return the project to this state using {\bf git checkout}

\item Beware if you  accidentally  add a file with confidential info 
to git - need to remove it from all commits.

\item {\bf git add} copies file to staging area for next commit

\item {\bf git commit -a} if you want commit current versions of all files being tracked

\item commits have parent commit(s), most have 1 parent

\item merge produce commit with 2 parents, first commit has no parent

\item merge commits labelled with SHA-1 hash
\end{itemize}
\end{frame}


\begin{frame}
\frametitle{Git Branch}

\begin{itemize}
\item Git branch is pointer to a commit

\item Allows you to name a series of commits

\item Provides convenient tracking of versions of parallel version of projects

\item New features can be developed in a branch and eventually merged with other branches

\item Default branch is called master. 

\item {\bf HEAD} is a reference to the last commit in the current branch

\item  {\bf git branch {\it name}} creates branch {\it name}

\item  {\bf git checkout {\it name}} changes all project files to their version on this branch
\end{itemize}
\end{frame}

\begin{frame}
\frametitle{Git Merge}

\begin{itemize}
\item merges branches
\item git mergetool shows conflicts
\item configure your own mergetool - many choices kdiff3, meld, p4merge
\item kaleidoscope popular on OSX 
\end{itemize}

\end{frame}

\begin{frame}
\frametitle{Git Push}

\begin{itemize}
\item {\bf git push {\it repository-name} {\it branch}} adds commits from your {\it branch} to remote repository {\it repository-name} 

\item Can set defaults, e.g. {\bf git push -u origin master} then run {\bf git push}

\item {\bf git remotel} lets you give names to other repositories

\item Note {\bf git clone} sets {\tt origin} to be name for cloned repository
\end{itemize}
\end{frame}

\begin{frame}
\frametitle{Git Fetch/Pull}

\begin{itemize}
\item {\bf git fetch {\it repository-name} {\it branch}} adds commits from {\it branch} in remote repository {\it repository-name} 

\item Usually  {\bf git pull} - combines fetch and merge
\end{itemize}
\end{frame}

\begin{frame}[fragile]
\frametitle{Example: Git \& conflicts}

\begin{sh}
$ git init /home/cs2041
Initialized empty Git repository in /home/cs2041/.git/
$ git add main.c 
$ git commit main.c 
Aborting commit due to empty commit message.
$ git commit main.c -m initial
[master (root-commit) 8c7d287] initial
 1 files changed, 1 insertions(+), 0 deletions(-)
 create mode 100644 main.c
\end{sh}
\end{frame}

\begin{frame}[fragile]
\frametitle{Example: Git \& conflicts}
 
Suppose Fred does:

\begin{sh}
$ cd /home/fred
$ git clone /home/cs2041/.git /home/fred
Cloning into /home/fred...
done.
$ echo >fred.c
$ git commit -m 'created fred.c'
\end{sh}
\end{frame}

\begin{frame}[fragile]
\frametitle{Example: Git \& conflicts}
 
Suppose Jane does: 
\begin{sh}
$ cd /home/jane
$ git clone /home/cs2041/.git /home/jane
Cloning into /home/jane...
done.
$ echo '/* Jane Rules */' >>main.c
$ git commit -a  -m 'did some documentation'
[master 1eb8d32] did some documentation
 1 files changed, 1 insertions(+), 0 deletions(-)
\end{sh}
\end{frame}

\begin{frame}[fragile]
\frametitle{Example: Git \& conflicts}
 
Fred can now get Jane's work like this:

\begin{sh}
$ git pull /home/jane/.git
remote: Counting objects: 5, done.
remote: Total 3 (delta 0), reused 0 (delta 0)
Unpacking objects: 100% (3/3), done.
From /home/jane/.git
 * branch            HEAD       -> FETCH_HEAD
Merge made by recursive.
 main.c |    1 +
 1 files changed, 1 insertions(+), 0 deletions(-)
\end{sh}
\end{frame}

\begin{frame}[fragile]
\frametitle{Example: Git \& conflicts}
 
And Jane can now get Fred's work like this:

\begin{sh}
$  git pull /home/fred/.git
remote: Counting objects: 7, done.
remote: Compressing objects: 100% (4/4), done.
remote: Total 5 (delta 0), reused 0 (delta 0)
Unpacking objects: 100% (5/5), done.
From /home/fred/.git
 * branch            HEAD       -> FETCH_HEAD
Updating 1eb8d32..63af286
Fast-forward
 fred.c |    1 +
 1 files changed, 1 insertions(+), 0 deletions(-)
 create mode 100644 fred.c
\end{sh}
\end{frame}

\begin{frame}[shrink,fragile]
\frametitle{Example: Git \& conflicts}
 
But if Fred  does this:

\begin{sh}
$ echo '// Fred Rules' >fred.c
$ git commit -a  -m 'added documentation'
\end{sh}

And Jane does this:

\begin{sh}
$ echo '// Jane Rules' >fred.c
$ git commit -a  -m 'inserted comments'
\end{sh}
\end{frame}

\begin{frame}[shrink,fragile]
\frametitle{Example: Git \& conflicts}
 

When  Fred tries to get Jane's work:

\begin{sh}
$ git pull /home/jane/.git
remote: Counting objects: 5, done.
remote: Compressing objects: 100% (2/2), done.
remote: Total 3 (delta 0), reused 0 (delta 0)
Unpacking objects: 100% (3/3), done.
From ../jane/
 * branch            HEAD       -> FETCH_HEAD
Auto-merging fred.c
CONFLICT (content): Merge conflict in fred.c
Automatic merge failed; fix conflicts and then commit the result.
$ git  mergetool
\end{sh}
\end{frame}

\begin{frame}[fragile]
\frametitle{Example: making Git Repository Public via Github}
Github popular repo hosting site (see competitors e.g. bitbucket)

Github free for small number of public repos 

Github and competitors also let you setup collaborators, wiki, web pages, issue tracking

Web access to git repo e.g. https://github.com/mirrors/linux
\end{frame}

\begin{frame}[fragile]
\frametitle{Example: making Git Repository Public via Github}

Its a week after the 2041 assignment was due and you want to publish your code to the world.

Create github account - assume you choose {\it{2041rocks}} as your login

Create a repository - assume you choose {\it{my\_code}} for the repo name

Add your ssh key ({\tt{.ssh/id\_rsa.pub}}) to github (Account Settings - SSH Public Keys - Add another public key)

\begin{sh}
$ cd ~/cs2041/ass2
$ git remote add origin git@github.com:2041rocks/my_code.git
$ git push -u origin master
\end{sh}

Now anyone anywhere can clone your repository by 

\begin{sh}
git clone git@github.com:2041rocks/my_code.git
\end{sh}
\end{frame}

\begin{frame}[fragile,shrink]
\frametitle{Simple Version Control System}

You can implement in a small amount of  Perl the  basic operations of a VCS.
We'll use these utility functions:

\begin{small}
\begin{perl}
sub read_file {
  my ($file) = @_;
  open(my $f, '<', $file) or die "Can noyt read $file: $!\n";
  return do {local $/; <$f>}
}

sub write_file {
  my ($file, $contents) = @_;
  open my $f, '>', $file or die "Can not write $file: $!\n";
  print $f $contents;
}

sub read_dir {
  my ($dir) = @_;
  my @files = glob "$dir/*";
  s/.*\/// foreach @files;
  return  @files;
}
\end{perl}
\end{small}

\end{frame}

\begin{frame}
\frametitle{Repository}

Git uses a directory tree in {\tt .git} by default to store the repository.

Our simple VCS (sit.pl) uses {\tt .sit}, storing each version
in a separate numbered subdirectory.
\end{frame}

\begin{frame}[fragile]
\frametitle{Creating an empty repository:}
\begin{itemize}
\item  git init
\item  sit.pl init
\end{itemize}

Our simple VCS just needs to create a directory and initialize a file.

\begin{small}
\begin{perl}
die "repository already exists\n" if -d $repository_dir;
mkdir $repository_dir or
    die "$0: can not create $repository_dir:$!";
write_file($commit_file, 0);
print "Created empty repository\n";
\end{perl}
\end{small}
\end{frame}

\begin{frame}[fragile]
\frametitle{Placing Files under Version Control}
\begin{itemize}
\item  git add FILES
\item  sit.pl add FILES
\end{itemize}

Our simple VCS copies the files into a sub-directory of the repository (the stage area)

Git does also stages the files for commit.

\begin{small}
\begin{perl}
-d $staging_area or mkdir $staging_area or
    die "$0: can not create $staging_area: $!\n";
write_file("$staging_area/$_",read_file($_)) foreach @ARGV; 
my @staged_files = read_dir("$staging_area");
print "Files now staged: @staged_files\n";
\end{perl}
\end{small}
\end{frame}

\begin{frame}[fragile]
\frametitle{Saving a Version}
\begin{itemize}
\item  git commit  -m LOG\_MESSAGE
\item  sit.pl commit LOG\_MESSAGE
\end{itemize}

Git labels commits with a SHA-1 hash (160 bit 40 hex digits). 

Our simple VCS labels each commit with consecutive integer.

It moves the files from its staging area into a sub-directory named with this number and writes the log message.

\begin{small}
\begin{perl}
die "No files staged for commit\n" if !-d $staging_area;
my $last_commit = read_file($commit_file);
my $commit = $last_commit + 1;
write_file($commit_file,  $commit);
$version_dir = "$repository_dir/$commit";
rename $staging_area, $version_dir;
write_file("$repository_dir/log_message.$commit", "@ARGV"); 
my @commited_files = read_dir("$repository_dir/$commit");
print "Commited as commit #$commit: @commited_files\n";
\end{perl}
\end{small}
\end{frame}

\begin{frame}[fragile,shrink]
\frametitle{Checking out a version}
Common to want to examine a particular version of a file.
\begin{itemize}
\item  git show file  [branch]
\item  sit.pl show file  [version]
\end{itemize}

Our simple VCS has to work backwards through versions
from the requested version to find the appropriate version of the file.

\begin{small}
\begin{perl}
my $file =  shift @ARGV or die "Usage: $0 <file>\n";
my $checkout_commit = shift @ARGV or read_file($commit_file);
# go through commits in reverse order
# finding last commit of each file
foreach $commit (reverse 1..$checkout_commit) {
    my $path =  "$repository_dir/$commit/$file";
    next if !-r $path;
    print STDERR "Commit #$commit: $file\n";
    print read_file($path);
    exit(0);
}
die "$0: no commit of $file\n";
\end{perl}
\end{small}

\end{frame}

\begin{frame}[fragile,shrink]
\frametitle{Reviewing commits}
\begin{itemize}
\item  git log [huge number of options]
\item  sit.pl log
\end{itemize}

Our simple VCS just prints the log message for each commit and
which files it contained.

\begin{small}
\begin{perl}
my $last_commit = read_file($commit_file);
foreach $commit (1..$last_commit) {
}
\end{perl}
\end{small}
\end{frame}

\begin{frame}[fragile,shrink]
\frametitle{Checking Status of Files}
\begin{itemize}
\item  git status 
\item  sit.pl log
\end{itemize}

Our simple VCS indicates files in the current directory
not in the repository or which are different to the latest version in the repository.

\begin{small}
\begin{perl}
my $last_commit = read_file($commit_file);
FILE: foreach $file (glob "*") {
    my $path =  "$staging_area/$file";
    if (-r $path) {
        if (read_file($file) ne read_file($path)) {
           print "Modified since staged: $file\n";
        } else {
           print "Staged for commit: $file\n";
        }
    } else {
        foreach $commit (reverse 1..$last_commit) {
            my $path =  "$repository_dir/$commit/$file";
            next if !-r $path;
            if (read_file($file) ne read_file($path)) {
               print "Modified and unstaged: $file\n";
            } else {
               print "Unmodified since committed: $file\n";
            }
            next FILE;
        }
        print "Untracked: $file\n";
    }
}
\end{perl}
\end{small}
\end{frame}

\begin{frame}
\frametitle{Inspecting file differences}
\begin{itemize}
\item  git diff [many options]
\item  sit.pl unimplemented
\end{itemize}
\end{frame}

\begin{frame}[fragile]
\frametitle{Updating your copy from repository}
\begin{itemize}
\item  git pull   (or fetch)
\item  sit.pl unimplemented
\end{itemize}
\end{frame}


\begin{frame}[shrink,fragile]
\frametitle{Using our simple Simple Version Control System}
\begin{sh}
$ sit.pl init
Created empty repository
$ sit.pl add *.[ch]
Files staged are: world.c world.h graphics.c graphics.h main.c
$ sit.pl commit 'initial import'
Commited as commit #1: world.c world.h graphics.c graphics.h main.c
$ echo >>world.c
$ echo >>graphics.c
$ sit.pl add world.c
$ echo >>world.c
$  sit.pl status
Untracked: Makefile
Unstaged: graphics.c
Modified: world.c
$ sit.pl add world.c graphics Makefile
Files staged are: world.c graphics.c Makefile
$ sit.pl commit 'Minor changes'
Commited as commit #2: world.c graphics.c Makefile
$ sit.pl log
Commit #1
Log message: initial import
Files: world.c world.h graphics.c graphics.h main.c
Commit #2
Log message: Minor changes
Files: world.c graphics.c Makefile
\end{sh}
\end{frame}


\end{document}
