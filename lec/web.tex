% https://groups.google.com/forum/#!topic/comp.text.tex/s6z9Ult_zds
\makeatletter\let\ifGm@compatii\relax\makeatother

\ifx\notes\undefined
    \documentclass{beamer}
 \else
   \documentclass[handout]{beamer}
    \mode<handout>{
        \usepackage{pgfpages}
    \pgfpagesuselayout{4 on 1}[a4paper ,border shrink=2mm, landscape]
	\pgfpageslogicalpageoptions{1}{border code=\pgfsetlinewidth{1mm}\pgfusepath{stroke}}
	\pgfpageslogicalpageoptions{2}{border code=\pgfsetlinewidth{1mm}\pgfusepath{stroke}}
	\pgfpageslogicalpageoptions{3}{border code=\pgfsetlinewidth{1mm}\pgfusepath{stroke}}
	\pgfpageslogicalpageoptions{4}{border code=\pgfsetlinewidth{1mm}\pgfusepath{stroke}}
%    \setbeamercolor{background canvas}{bg=black!3}
    }
\fi
%http://www.cpt.univ-mrs.fr/~masson/latex/Beamer-appearance-cheat-sheet.pdf
\definecolor{Dandelion}         {RGB}{253, 200, 47}
\setbeamertemplate{frametitle}
{\vspace{1mm}
\begin{centering}
\insertframetitle
\end{centering}
{\color{Dandelion} \rule{\textwidth}{1mm}}
}
\setbeamertemplate{itemize item}{$\bullet$}
\setbeamertemplate{navigation symbols}{}
%\setbeamercolor{title}{fg=Fern}
%\setbeamercolor{frametitle}{fg=Fern}
%\setbeamercolor{normal text}{fg=Charcoal}
%\setbeamercolor{block title}{fg=black,bg=Fern!25!white}
%\setbeamercolor{block body}{fg=black,bg=Fern!25!white}
%\setbeamercolor{alerted text}{fg=AlertColor}
\setbeamercolor{itemize item}{fg=black}

\usepackage[latin1]{inputenc}
\usepackage{epic,eepic,epsf}
%\usepackage{latexsym}
\usepackage{verbatim}
\usepackage{listings}
\usepackage{alltt}
\usepackage{minted}

%% all code examples use "code" environment
\newenvironment{code}{\begin{minted}{c}}{\end{minted}}

%\newcommand{\UnderscoreCommands}{\do\verbatiminput%
%\do\citeNP \do\citeA \do\citeANP \do\citeN \do\shortcite%
%\do\shortciteNP \do\shortciteA \do\shortciteANP \do\shortciteN%
%\do\citeyear \do\citeyearNP%
%}
%\usepackage[strings]{underscore}
%\usetheme{Pittsburgh}

%\makeatletter
%\setbeamercolor{background canvas}{bg=white}
%\setbeamercolor{background}{bg=white}
%\makeatother

\title[COMP1511 Introduction to Programming]{COMP1511}
\author{Andrew Taylor/John Shepherd}
\institute{CSE, UNSW}
\date{\today}

\newcommand{\red}[1]{\textcolor{red}{#1}}
\newcommand{\blue}[1]{\textcolor{blue}{#1}}
\newcommand<>{\RedNote}[2]{%
  \begin{alertblock}#3{#1}
    #2
  \end{alertblock}}
\newcommand<>{\Note}[2]{%
  \begin{exampleblock}#3{#1}
    #2
  \end{exampleblock}}
\newcommand{\NotesFrame}{\frame<handout>[plain]{\frametitle{My Notes}}}
\newcommand{\ttblue}[1]{\texttt{\blue{#1}}}

\usepackage{epstopdf}


% prefer pdf over png if both versions exist
\DeclareGraphicsExtensions{%
    .pdf,.PDF,%
    .png,.PNG,%
    .jpg,.mps,.jpeg,.jbig2,.jb2,.JPG,.JPEG,.JBIG2,.JB2}

\setbeamercolor{block body}{bg=gray!7}

\usemintedstyle{tango}

\newenvironment{C}
 {\VerbatimEnvironment
  \setbeamertemplate{blocks}[rounded][shadow=true]
  \begin{block}{}
  \begin{minted}{c}}
 {\end{minted}
  \end{block}}
  
\newenvironment{sh}
 {\VerbatimEnvironment
  \setbeamertemplate{blocks}[rounded][shadow=true]
  \begin{block}{}
  \begin{minted}{sh}}
 {\end{minted}
  \end{block}}
  
\newenvironment{perl}
 {\VerbatimEnvironment
  \setbeamertemplate{blocks}[rounded][shadow=true]
  \begin{block}{}
  \begin{minted}{perl}}
 {\end{minted}
  \end{block}}
  
\newenvironment{python}
 {\VerbatimEnvironment
  \setbeamertemplate{blocks}[rounded][shadow=true]
  \begin{block}{}
  \begin{minted}{python}}
 {\end{minted}
  \end{block}}
  
\newenvironment{html}
 {\VerbatimEnvironment
  \setbeamertemplate{blocks}[rounded][shadow=true]
  \begin{block}{}
  \begin{minted}{html}}
 {\end{minted}
  \end{block}}
  
\newenvironment{txt}
 {\VerbatimEnvironment
  \setbeamertemplate{blocks}[rounded][shadow=true]
  \begin{block}{}
  \begin{minted}{text}}
 {\end{minted}
  \end{block}}

\begin{document}
\section{Perl/CGI Scripting}
\begin{frame}[fragile]
\frametitle{TCP/IP Intro}

TCP/IP beyond scope of this course - take COMP[39]331.

But easier to understand  CGI using TCP/IP from Perl

Easy to establish a TCP/IP connection.

Server running on host williams.cse.unsw.edu.au does this:
\begin{perl}
use IO::Socket;
$server = IO::Socket::INET->new(LocalPort => 1234,
               Listen => SOMAXCONN) or die;
$c = $server->accept()
\end{perl}

Client running anywhere on internet does this:

\begin{perl}
use IO::Socket;
$host = "williams.cse.unsw.edu.au";
$c = IO::Socket::INET->new(PeerAddr=>$host,
                           PeerPort=>1234) or die;
\end{perl}

Then \textbf{\tt{\$c}} effectively a bidirectional file handle.
\end{frame}

\begin{frame}[fragile]
\frametitle{Time Server}

A simple TCP/IP server which supplies the current time as an ASCII string.

\begin{perl}
use IO::Socket;
$server = IO::Socket::INET->new(LocalPort => 4242,
               Listen => SOMAXCONN) or die;
while ($c = $server->accept()) {
    printf "[Connection from %s]\n", $c->peerhost;
    print $c scalar localtime,"\n";
    close $c;
}
\end{perl}

{\tiny Source:  \href{http://cgi.cse.unsw.edu.au/~cs2041/code/web/timeserver.pl}{http://cgi.cse.unsw.edu.au/{\textasciitilde}cs2041/code/web/timeserver.pl}}
\end{frame}

\begin{frame}[fragile]
\frametitle{Time Client}

Simple client which gets the time from the server on host \textbf{\tt{\$ARGV[0]}}
and prints it.

See NTP for how to seriously distribute time across networks.

\begin{perl}
use IO::Socket;
$server_host =  $ARGV[0] || 'localhost';
$server_port = 4242;
$c = IO::Socket::INET->new(PeerAddr => $server_host,
                PeerPort  => $server_port) or die;
$time = <$c>;
close $c;
print "Time is $time\n";
\end{perl}

{\tiny Source:  \href{http://cgi.cse.unsw.edu.au/~cs2041/code/web/timeclient.pl}{http://cgi.cse.unsw.edu.au/{\textasciitilde}cs2041/code/web/timeclient.pl}}

\end{frame}

\begin{frame}
\frametitle{Well-known TCP/IP ports}
To connect via TCP/IP you need to know the port.
Particular services often listen to a standard TCP/IP port
on the machine they are running. For example:

\begin{itemize}
\item  21 ftp
\item  22 ssh (Secure shell)
\item  23 telnet
\item  25 SMTP (e-mail)
\item  80 HTTP (Hypertext Transfer Protocol)
\item  123 NTP (Network Time Protocol)
\item  443 HTTPS (Hypertext Transfer Protocol over SSL/TLS)
\end{itemize}

So web server normally listens to port 80 (http) or 443 (https).

\end{frame}

\begin{frame}[fragile]
\frametitle{Uniform Resource Locator (URL)}

Familiar  syntax:

\begin{txt}
scheme://domain:port/path?query_string#fragment_id
\end{txt}

For example:

\begin{txt}
http://en.wikipedia.org/wiki/URI_scheme#Generic_syntax
http://www.google.com.au/search?q=COMP2041&hl=en
\end{txt}

Given a http URL a web browser extracts the hostname from the URL
and connects to port 80 (unless another port is specified).

It then sends the remainder of the URL to the server.

The HTTP syntax of such a request is simple: \\[2ex]

{\tt GET {\it{path}} HTTP/{\it{version}}} \\[2ex]

We can do this easily in Perl
\end{frame}

\begin{frame}[fragile]
\frametitle{Simple Web Client in Perl}
A very simple web client - doesn't render the HTML, no GUI, no ... -
see HTTP::Request::Common for a more general solution

\begin{perl}
use IO::Socket;
foreach $url (@ARGV) {
    $url =~ /http:\/\/([^\/]+)(:(\d+))?(.*)/ or die;
    $c = IO::Socket::INET->new(PeerAddr => $1,
            PeerPort => $2 || 80) or die;
    # send request for web page to server
    print $c "GET $4 HTTP/1.0\n\n";
    # read what the server returns
    my @webpage = <$c>; 
    close $c;
    print "GET $url =>\n", @webpage, "\n";
}
\end{perl}

{\tiny Source: \href{http://cgi.cse.unsw.edu.au/~cs2041/code/web/webget.pl}{http://cgi.cse.unsw.edu.au/{\textasciitilde}cs2041/code/web/webget.pl}}
\end{frame}

\begin{frame}[fragile]
\frametitle{Simple Web Client in Perl}

\begin{sh}
$ cd /home/cs2041/public_html/lec/cgi/examples
$ ./webget.pl http://cgi.cse.unsw.edu.au/
GET http://cgi.cse.unsw.edu.au/ =>
HTTP/1.1 200 OK
Date: Sun, 21 Sep 2014 23:40:41 GMT
Set-Cookie: JSESSIONID=CF09BE9CADA20036D93F39B04329DB
Last-Modified: Sun, 21 Sep 2014 23:40:41 GMT
Content-Type: text/html;charset=UTF-8
Content-Length: 35811
Connection: close

<!DOCTYPE html>
<html lang='en'>
  <head>
...
\end{sh}

Notice the web server returns some header lines and then data.
\end{frame}

\begin{frame}[fragile]
\frametitle{Web server in Perl - getting started}

This Perl web server just prints details of incoming requests \&
always returns a 404 (not found) status.
\begin{perl}
use IO::Socket;
$server = IO::Socket::INET->new(LocalPort => 2041,
          ReuseAddr => 1, Listen => SOMAXCONN) or die;
while ($c = $server->accept()) {
    printf "HTTP request from %s =>\n\n", $c->peerhost;
    while ($request_line = <$c>) {
        print "$request_line";
        last if $request_line !~ /\S/;
    }
    print $c "HTTP/1.0 404 This server always 404s\n";
    close $c;
}
\end{perl}

{\tiny \href{http://cgi.cse.unsw.edu.au/~cs2041/code/web/webserver-404.pl}{http://cgi.cse.unsw.edu.au/~cs2041/code/web/webserver-404.pl}}
\end{frame}

\begin{frame}[fragile]
\frametitle{Web server in Perl - getting started}
\vspace{-7mm}
\begin{small} 
\begin{perl}
use IO::Socket;
$server = IO::Socket::INET->new(LocalPort => 2041,
          ReuseAddr => 1, Listen => SOMAXCONN) or die;
$content = "Everything is OK - you love COMP[29]041.\n";
while ($c = $server->accept()) {
    printf "HTTP request from %s =>\n\n", $c->peerhost;
    while ($request_line = <$c>) {
        print "$request_line";
        last if $request_line !~ /\S/;
    }
    my $request = <$c>;
    print "Connection from ", $c->peerhost, ": $request";
    $request =~ /^GET (.+) HTTP\/1.[01]\s*$/;
    print "Sending back /home/cs2041/public_html/$1\n";
    open my $f, '<', "/home/cs2041/public_html/$1";
    $content = join "", <$f>;
    print $c "HTTP/1.0 200 OK\nContent-Type: text/html\n";
    print $c "Content-Length: ",length($content),"\n";
    print $c $content;
    close $c;
}
\end{perl}
\end{small} 

{\tiny \href{http://cgi.cse.unsw.edu.au/~cs2041/code/web/webserver-200.pl}{http://cgi.cse.unsw.edu.au/~cs2041/code/web/webserver-debug.pl}}
\end{frame}

\begin{frame}[fragile]
\frametitle{Web server in Perl - too simple}
A simple web server in Perl.

Does fundamental job of serving web pages but
has  bugs, securtity holes and huge limitations.

\begin{perl}
while ($c = $server->accept()) {
    my $request = <$c>;
    $request =~ /^GET (.+) HTTP\/1.[01]\s*$/;
    open F, "</home/cs2041/public_html/$1";
    $content = join "", <F>;
    print $c "HTTP/1.0 200 OK\n";
    print $c "Content-Type: text/html\n";
    print $c "Content-Length: ",length($content),"\n";
    print $c $content;
    close $c;
}
\end{perl}

{\tiny Source:  \href{http://cgi.cse.unsw.edu.au/~cs2041/code/web/webserver-too-simple.pl}{http://cgi.cse.unsw.edu.au/{\textasciitilde}cs2041/code/web/webserver-too-simple.pl}}
\end{frame}

\begin{frame}[fragile]
\frametitle{Web server in Perl - mime-types}
Web servers typically determine a file's type from its extension (suffix)
and pass this back in a header line.

ON Unix-like systems  file /etc/mime.types contains lines mapping extensions to
mime-types, e.g.:

\begin{txt}
    application/pdf             pdf
    image/jpeg                  jpeg jpg jpe
    text/html                   html htm shtml
\end{txt}

May also be configured within web-server e.g cs2041's .htaccess file contains:

\begin{txt}
AddType text/plain pl py sh c cgi
\end{txt}

\end{frame}

\begin{frame}[fragile]
\frametitle{Web server in Perl - mime-types}

Easy to read /etc/mime.types specifications into a hash:

\begin{perl}
open MT, '<', "/etc/mime.types") or die;
while ($line = <MT>) {
    $line =~ s/#.*//;
    my ($mime_type, @extensions) = split /\s+/, $line;
    foreach $extension (@extensions) {
    	$mime_type{$extension} = $mime_type;
    }
}
\end{perl}
\end{frame}

\begin{frame}[fragile]
\frametitle{Web server in Perl - mime-types}

Previous simple web server with code added to
use the {\tt mime\_type} hash to return the appropriate {\tt Content-type}:

\begin{small}
\begin{perl}
$url =~ s/(^|\/)\.\.(\/|$)//g;
my $file = "/home/cs2041/public_html/$url";
# prevent access outside 2041 directory
$file =~ s/(^|\/)..(\/|$)//g;
$file .= "/index.html" if -d $file;
if (open my $f, '<', $file) {
    my ($extension) = $file =~ /\.(\w+)$/;
    print $c "HTTP/1.0 200 OK\n";
    if ($extension && $mime_type{$extension}) {
        print $c "Content-Type: $mime_type{$extension}\n";
    }
    print $c <my $f>;
}
\end{perl}
\end{small}

{\tiny Source:  \href{http://cgi.cse.unsw.edu.au/~cs2041/code/web/webserver-mime-types.pl}{http://cgi.cse.unsw.edu.au/{\textasciitilde}cs2041/code/web/webserver-mime-types.pl}}
\end{frame}

\begin{frame}[fragile]
\frametitle{Web server in Perl - multi-processing}

Previous web server scripts serve only one request at a time.

They can not handle a high volume of requests.

And slow client can deny access for others to the web server, e.g 
our previous web client with a 1 hour sleep added:

\begin{perl}
$url =~ /http:\/\/([^\/]+)(:(\d+))?(.*)/ or die;
$c = IO::Socket::INET->new(PeerAddr => $1,
      PeerPort => $2 || 80) or die;
sleep 3600;
print $c "GET $4 HTTP/1.0\n\n";
\end{perl}

{\tiny Source:  \href{http://cgi.cse.unsw.edu.au/~cs2041/code/web/webget-slow.pl}{http://cgi.cse.unsw.edu.au/{\textasciitilde}cs2041/code/web/webget-slow.pl}}

Simple solution is to process each request in a separate process.

The Perl subroutine fork duplicates a running program.

Returns 0 in new process (child) and process id of child in
original process (parent).
\end{frame}

\begin{frame}[fragile]
\frametitle{Web server in Perl - multi-processing}

We can add this easily to our previous webserver:
\begin{perl}
while ($c = $server->accept()) {
    if (fork() != 0) {
        # parent process goes to waiting for next request
        close($c);
        next;
    }
    # child processes request
    my $request = <$c>;
    ...
    close $c;
    # child must terminate here otherwise
    # it would compete with parent for requests
    exit 0;
}
\end{perl}

{\tiny Source:  \href{http://cgi.cse.unsw.edu.au/~cs2041/code/web/webserver-parallel.pl}{http://cgi.cse.unsw.edu.au/{\textasciitilde}cs2041/code/web/webserver-parallel.pl}}
\end{frame}

\begin{frame}[fragile]
\frametitle{Web server - Simple CGI}
Web servers allow dynamic content to be generated via CGI (and other ways).

Typically they can be configure to execute programs for certain URIS.

for example cs2041's .htaccess file indicates files with suffix {\tt .cgi} should be executed.

\begin{txt}
<Files *.cgi>
SetHandler application/x-setuid-cgi
</Files>
\end{txt}
\end{frame}

\begin{frame}[fragile]
\frametitle{Web server - Simple CGI}

We can add this to our simple web-server:
\begin{perl}
if ($url =~ /^(.*\.cgi)(\?(.*))?$/) {
    my $cgi_script = "/home/cs2041/public_html/$1";
    $ENV{SCRIPT_URI} = $1;
    $ENV{QUERY_STRING} = $3 || '';
    $ENV{REQUEST_METHOD} = "GET";
    $ENV{REQUEST_URI} = $url;
    print $c "HTTP/1.0 200 OK\n";
    print $c `$cgi_script` if -x $cgi_script;
    close F;
} 
\end{perl}
 
{\tiny Source:  \href{http://cgi.cse.unsw.edu.au/~cs2041/code/web/webserver-simple-cgi.pl}{http://cgi.cse.unsw.edu.au/{\textasciitilde}cs2041/code/web/webserver-simple-cgi.pl}}
\end{frame}

\begin{frame}[fragile]
\frametitle{Web server - CGI}

A fuller CGI implementation implementing both GET and POST requests can be found here:

{\tiny Source:  \href{http://cgi.cse.unsw.edu.au/~cs2041/code/web/webserver-cgi.pl}{http://cgi.cse.unsw.edu.au/{\textasciitilde}cs2041/code/web/webserver-cgi.pl}}
\end{frame}

\begin{frame}
\frametitle{HTML \& CSS}

\begin{itemize}
\item  We're (hopefully) all familiar with HTML .

\item  HTML \& CSS not covered (much) in lectures.

\item  If not familiar with HTML \& CSS, may need to do extra reading.

\item Tutes \& labs will help.
\end{itemize}
\end{frame}


\begin{frame}
\frametitle{Semantic Markup}

\begin{itemize}
\item  Use HTML tags to indicate the nature of the content. 

\item  Use CSS to indicate how content type should be displayed
\end{itemize}


\end{frame}

\begin{frame}
\frametitle{Document Object Model}
\begin{itemize}
\item  content marked up with {\em{tags}} to describe appearance
\item  browser reads HTML and builds Document Object Model (DOM)
\item  browser produces a visible rendering of DOM
\end{itemize}
    \begin{figure}
        \centering
        \includegraphics[width = 0.9\textwidth]{Pic/staticpage}
        % \caption{Caption}
    \end{figure}
\end{frame}

\begin{frame}
\frametitle{Dynamic Web Pages}
HTML tags are {\em{static}}
    ~ {\small  (i.e. produce a fixed rendering).}

``Dynamic'' web content can be generated on server

\begin{itemize}
\item  Generated on the server:
\begin{itemize}
\item  SSP ~ {\small (program running in web server generates HTML)}
\item  CGI ~ {\small (program running outside web server generates HTML)}
\item  many other variants
\end{itemize}

\item  Generated in the browser
\begin{itemize}
\item  JavaScript ~ {\small (browser manipulates document object model)}
\item  Java Applet ~ {\small (JVM in browser executes Java program) - dead}
\item  Silverlight ~ {\small (Microsoft browser plugin) - dying}
\item  Flash ~ {\small (Adobe browser plugin) - dying}
\end{itemize}
\end{itemize}
\end{frame}

\begin{frame}
\frametitle{Dynamic Web Pages}
For CGI and SSP, the scripts {\small (HTML generators)} are invoked
\begin{itemize}
\item  via a URL
    ~ {\small (giving the name and type of application)}
\item  passing some data values
    ~ {\small (either in the URL or via stdin)}
\end{itemize}
The data values are typically
\begin{itemize}
\item  collected in a fill-in form which invokes the script
\item  passed from page to page (script to script) via GET/POST
\end{itemize}
{\small 
(other mechanisms for data transmission include cookies and server-state)
}
\end{frame}

\begin{frame}
\frametitle{CGI ~{\small (Common Gateway Interface)}}

    \begin{figure}
        \centering
        \includegraphics[width = 0.9\textwidth]{Pic/cgi}
        % \caption{Caption}
    \end{figure}

Data is passed as \textbf{\tt{name=value}} pairs
    ~ {\small (all values are strings)}.

Application outputs (normally) HTML, which server passes to client.

For HTML documents, header is ~ \textbf{\tt{Content-type: text/html}}

{\small 
Header can be any MIME-type (e.g. \textbf{\tt{text/html}}, ~ \textbf{\tt{image/gif}}, ...)
}
\end{frame}

\begin{frame}
\frametitle{Perl and CGI}
So how does Perl fit into this scenario?

CGI scripts typically:
\begin{itemize}
\item  do lots of complex string manipulation
\item  write many complex strings (HTML) to output
\end{itemize}
Perl is good at manipulating strings - good for CGI.

Libraries for Perl make CGI processing even easier.

CGI.pm is one such library 
    ~ {\small (see later for more details)}
\end{frame}

\begin{frame}
\frametitle{SSP ~{\small (Server-side Programming)}}


    \begin{figure}
        \centering
        \includegraphics[width = 0.9\textwidth]{Pic/sscript}
        % \caption{Caption}
    \end{figure}

Data is available via library functions (e.g. \textbf{\tt{param}}).

Script produces HTML output, which is sent to client (browser).

\end{frame}

\begin{frame}
\frametitle{JavaScript ~{\small (Client-side DOM Scripting)}}


    \begin{figure}
        \centering
        \includegraphics[width = 0.9\textwidth]{Pic/jscript}
        % \caption{Caption}
    \end{figure}
Executing script can modify browser's internal representation of document (DOM)

Browser then changes appearance of document on screen.

This can happen at script load time or in response to {\em{events}}
(such as \textbf{\tt{onClick}}, \textbf{\tt{onMouseOver}}, \textbf{\tt{onKeyPress}}) after
script has loaded.

Can also access data in form controls {\small (because they are also document elements)}.
\end{frame}

\begin{frame}[fragile]
\frametitle{JavaScript ~{\small (Client-side DOM Scripting)}}
For example, this web page has JavaScript embedded to sum two numbers from input fields and 
store the result in a third  field.

The function is run whenever a character is entered in either field.
\begin{perl}
<input type=text id="x" onkeyup="sum();"> +
<input type=text id="y" onkeyup="sum();"> =
<input type=text id="sum" readonly="readonly">
<script type="text/javascript">
function sum() {
  var x = parseInt(document.getElementById('x').value);
  var y = parseInt(document.getElementById('y').value);
  document.getElementById('sum').value = num1 + num2;
}
</script>
\end{perl}

{\tiny Source:  \href{http://cgi.cse.unsw.edu.au/~cs2041/code/web/javascript_sum_two_numbers.html}{http://cgi.cse.unsw.edu.au/{\textasciitilde}cs2041/code/web/javascript\_sum\_two\_numbers.html}}
\end{frame}

\begin{frame}
\frametitle{Ajax}
Ajax provides a variation on the above approach:
\begin{itemize}
\item  ``normal'' browser-to-server interaction is HTTP request
\item  this causes browser to read response as HTML {\small (new page)}
\item  Ajax sends XMLHttpRequests from browser-to-server
\item  browser does not refresh, but waits for a response
\item  response data {\small (not HTML)} is read and added into DOM
\end{itemize}
Leads to interaction appearing more like traditional GUI.

Examples: Gmail, Google calendar, Flickr, ....

The popular JQuery library is an easy way to use AJAX.
\end{frame}

\begin{frame}
\frametitle{Ajax}
Ajax-style browser/server interaction:

    \begin{figure}
        \centering
        \includegraphics[width = 0.9\textwidth]{Pic/webdb-ajax}
        % \caption{Caption}
    \end{figure}
\end{frame}

\begin{frame}[fragile]
\frametitle{Ajax showing result of matching Perl regex}
\begin{perl}
$(document).ready(
 function() {
  $("#match").click(
   function() {
    $.get(
     "match.cgi",
     {string:$("#string").val(), regex:$("#regex").val()},
     function(data) {
      $("#show").html(data)
     }
    )
   }
  )
 }
)
\end{perl}

{\tiny Source:  \href{http://cgi.cse.unsw.edu.au/~cs2041cgi/code/web/match.html}{http://cgi.cse.unsw.edu.au/{\textasciitilde}cs2041cgi/code/web/match.html}}
\end{frame}

\begin{frame}[fragile]
\frametitle{Ajax}

A new page is not loaded when the match button is pressed.

JQuery only updates a field on the page.

It fetches by http the results of the match from this CGI script:

\begin{perl}
use CGI qw/:all/;
print header;
if (param('string') =~ param('regex')) {
    print b('Match succeeded, this substring matched: ');
    print tt(escapeHTML($&));
} else {
    print b('Match failed');
}
\end{perl}

{\tiny Source:  \href{http://cgi.cse.unsw.edu.au/~cs2041/code/web/match.cgi}{http://cgi.cse.unsw.edu.au/{\textasciitilde}cs2041/code/web/match.cgi}}

\end{frame}

\begin{frame}[fragile]
\frametitle{HTML Forms}
An HTML {\em{form}} combines the notions of
     ~ {\em{user input}} ~\&~ {\em{function call}} :
\begin{itemize}
\item  collects data via {\em{form control}} elements
\item  invokes a URL to process the collected data when \textbf{\tt{submit}}ted
\end{itemize}
Syntax:
\begin{perl}
    <form method=RequestMethod action=URL ...>
    any HTML except another form
       mixed with
    data collection (form control) elements
    </form>
\end{perl}

An HTML document may contain any number of \textbf{\tt{{\textless}form>}}'s.

Forms can be arbitrarily interleaved with HMTL layout elements
     {\small (e.g. \textbf{\tt{{\textless}table>}})}
\end{frame}

\begin{frame}
\frametitle{\textbf{\tt{METHOD}} Attribute}
The {\it{RequestMethod}} value indicates how data is passed to \textbf{\tt{action}} URL.

Two {\it{RequestMethod}}s are available: \textbf{\tt{GET}} and \textbf{\tt{POST}}
\begin{itemize}
\item 
\textbf{\tt{GET}}: data attached to URL
    ~ (\textbf{\tt{{\it{URL}}{\bf{?}}{\it{name\_{1}}}={\it{val\_{1}}}{\bf{\&}}{\it{name\_{2}}}={\it{val\_{2}}}{\bf{\&}}...}})
\item 
\textbf{\tt{POST}}: data available to script via standard input
\end{itemize}
Within a server script all we see is a collection of variables:
\begin{itemize}
\item  with the same names as those used in the form elements
\item  initialised with the values collected in the form
\end{itemize}
\end{frame}

\begin{frame}[fragile]
\frametitle{URL-encoded Strings}
Data is passed from browser to server as a single string in the form:

{\it{name}}{\bf{=}}{\it{val}}{\bf{\&}}{\it{name}}{\bf{=}}{\it{val}}{\bf{\&}}{\it{name}}{\bf{=}}{\it{val}}{\bf{\&}}...

with no spaces and where '\textbf{\tt{{\bf{=}}}}' and '\textbf{\tt{{\bf{\&}}}}'
are treated as special characters.

To achieve this strings are "url-encoded" e.g: \\[2ex]

\begin{tabular}{|l|l|}
\hline
\textbf{\tt{andrewt}} & \textbf{\tt{andrewt}} \\
\hline
\textbf{\tt{John Shepherd}}  &  \textbf{\tt{John{\em{+}}Shepherd}}  \\
\hline
\textbf{\tt{{\textasciitilde}cs2041 = /home/cs2041}}  & \textbf{\tt{{\em{\%7E}}cs2041{\em{+}}{\em{\%3D}}{\em{+}}{\em{\%2F}}home{\em{\%2F}}cs2041}}  \\
\hline
\textbf{\tt{1 + 1 = 2}}  & \textbf{\tt{1{\em{+}}{\em{\%2B}}{\em{+}}1{\em{+}}{\em{\%3D}}{\em{+}}2}}  \\
\hline
\textbf{\tt{Jack \& Jill = Love!}}  & \textbf{\tt{Jack{\em{+}}{\em{\%26}}{\em{+}}Jill{\em{+}}{\em{\%3D}}{\em{+}}Love{\em{\%21}} }}  \\
\hline
\end{tabular}
 \\[2ex]
URL-encoded strings are usually decoded by library before your code sees them.
\end{frame}

\begin{frame}
\frametitle{\textbf{\tt{ACTION}} Attribute}
\textbf{\tt{{\textless}form ... {\em{{\bf{action}}}}='{\it{URL}}' ... >}}
\begin{itemize}
\item  specifies script \textbf{\tt{{\it{URL}}}} to process form data
\end{itemize}
When the form is submitted ...
\begin{itemize}
\item  invoke the URL specified in \textbf{\tt{action}}
\item  pass all form data to it
\end{itemize}
If no \textbf{\tt{action}} attribute, re-invoke the current script.
\end{frame}

\begin{frame}
\frametitle{Other \textbf{\tt{{\textless}form>}} Attributes}
\textbf{\tt{{\textless}form ... {\em{{\bf{name}}}}='{\it{FormName}}' ... >}}
\begin{itemize}
\item  associates the name \textbf{\tt{{\it{FormName}}}} with the entire form
\item  useful for referring to form in JavaScript
\end{itemize}
\textbf{\tt{{\textless}form ... {\em{{\bf{target}}}}='{\it{WindowName}}' ... >}}
\begin{itemize}
\item  causes output from executing script to be placed in specified window
\item  useful when dealing with frames ~ {\small (see later)}
\end{itemize}
\textbf{\tt{{\textless}form ... {\em{{\bf{onSubmit}}}}='{\it{Script}}' ... >}}
\begin{itemize}
\item  specifies actions to be carried out just before sending data to script
\end{itemize}
\end{frame}

\begin{frame}
\frametitle{Form Controls}
{\em{Form controls}} are the individual data collection elements within a form.

Data can be collected in the following styles:
\begin{center}


\begin{center}\begin{tabular}{|l|l|}
\hline
 text  &  single line or region of text \\ \hline
 password  &  single line of text, value is hidden \\ \hline
 menu  &  choose 1 or many from a number of options \\ \hline
 checkbox  &  on/off toggle switch \\ \hline
 radio  &  choose only 1 from a number of options \\ \hline
 hidden  &  data item not displayed to user \\ \hline
 submit  &  button, sends collected data to script \\ \hline
 reset  &  button, resets all data elements in form \\ \hline
 button  &  button, effect needs to be scripted \\ \hline
\end{tabular}
\end{center}

\end{center}
\end{frame}

\end{document}
