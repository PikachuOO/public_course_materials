\documentclass[twocolumn,12pt]{article}
%\usepackage{twocolumn}
\usepackage{aa4-most}
\usepackage{alltt}
\begin{document}
\noindent


\newpage
\noindent
\textbf{\Huge C Information}

\section*{C Programs}
\begin{alltt}
/*
 * Program name and purpose
 */
#include <stdio.h>
#include \emph{other libraries ...}

#define CONST \emph{constant-expression}

typedef \emph{type-expression} Type; ...

\emph{type-expression} globalVar; ...

Status main(int argc, char *argv[])
\{
   \emph{main program variables};

   \emph{main program code};

   return 0;
\}

/*
 * Function name and purpose
 * Pre- and post- conditions
 */
\emph{return-type} funcName(\emph{parameters} ...)
\{
   \emph{local variables};

   \emph{function code};
\}
\end{alltt}
\section*{C Data}
\begin{alltt}
\textrm{Basic data types:}

int, long     - \textrm{integer numbers}
                (e.g. 1, -1, -27, 1000)
float, double - \textrm{real numbers}
                (e.g. 1.0, 1e23, -0.005)
char          - \textrm{characters}
                (e.g. 'a', 'b', '2', '!')
char *        - \textrm{character strings}
                (e.g. "a string", "John")

\textrm{Structured data types:}

\emph{type} a[\emph{N}];
   \textrm{Array} a \textrm{of} \emph{N} \textrm{objects of type} \emph{type}
   \textrm{Indexed via} 0 .. \emph{N}-1

struct \{\emph{t\(\sb{1}\)} a; \emph{t\(\sb{2}\)} b; \emph{t\(\sb{3}\)} c\} d;
   \textrm{Record} d \textrm{with fields} (a,b,c) \textrm{of types} (\emph{t\(\sb{1}\)},\emph{t\(\sb{2}\)},\emph{t\(\sb{3}\)})
\end{alltt}

\section*{C Expressions}
\begin{alltt}
\textrm{Arithmetic operators:}
   +   -   *   /   % \textrm{\tiny(modulus)}

\textrm{Relational operators:}
   ==   !=   <   >   <=   >=

\textrm{Logical operators:}
   ! \textrm{\tiny(NOT)}   && \textrm{\tiny(AND)}   || \textrm{\tiny(OR)}

\textrm{Bitwise operators:}
   ~ \textrm{\tiny(NOT)}   & \textrm{\tiny(AND)}   | \textrm{\tiny(OR)}   ^ \textrm{\tiny(XOR)}
\end{alltt}

\section*{C Statements}
\begin{alltt}
\textrm{Assignment:}

   \emph{var} = \emph{expression};
   x++;  ++x;  x+=\emph{expr};  x-=\emph{expr}; ...

{\large if} \textrm{Statement:}

   if (\emph{condition\(\sb{1}\)}) \{ \emph{statements\(\sb{1}\)}; \}
   else if (\emph{condition\(\sb{2}\)}) \{ \emph{statements\(\sb{2}\)}; \}
   ...
   else if (\emph{condition\(\sb{n}\)}) \{ \emph{statements\(\sb{n}\)}; \}
   else \{ \emph{statements\(\sb{n+1}\)}; \}

{\large switch} \textrm{Statement:}

   switch (\emph{expression}) \{
   case \emph{const\(\sb{1}\)}: \emph{statements\(\sb{1}\)}; break;
   case \emph{const\(\sb{2}\)}: \emph{statements\(\sb{2}\)}; break;
   ...
   case \emph{const\(\sb{n}\)}: \emph{statements\(\sb{n}\)}; break;
   default: \emph{statements\(\sb{n+1}\)};
   \}

{\large while} \textrm{Statement:}

   while (\emph{condition}) \{ \emph{statements}; \}

   break - \textrm{exit the loop}
   continue - \textrm{go to next iteration}

{\large for} \textrm{Statement:}

   for (\emph{init}; \emph{condition}; \emph{next})
       \{ \emph{statements}; \}

      \textrm{\small which is equivalent to}

   \emph{init};  while (\emph{condition})
             \{ \emph{statements}; \emph{next}; \}
\end{alltt}

% \section*{C Idioms}
% Read in text input, character by character:
% \begin{alltt}
%    int c;
%    while ((c = getchar()) != EOF) \{
%       \emph{process} c;
%    \}
% \end{alltt}
% Read in text input, line by line:
% \begin{alltt}
%    char s[\emph{N}];
%    while (gets(s) != NULL) \{
%       \emph{process} s;
%    \}
% \end{alltt}
% Read in all integer value input:
% \begin{alltt}
%    int i;
%    while (scanf("%d", &i) != EOF) \{
%        \emph{process} i;
%    \}
% \end{alltt}
% Scan along an array:
% \begin{alltt}
%    int i, a[\emph{N}];
%    for (i = 0; i < \emph{N}; ++i) \{
%       \emph{process} a[i];
%    \}
% \end{alltt}
% Scan along a linked-list:
% \begin{alltt}
%    struct n \{int data; struct n *next;\};
%    typedef struct n Node;
%    typedef Node *List;
%    List L;  Node *p;
%    for (p = L; p!= NULL;  p = p->next) \{
%       \emph{process} p->data;
%    \}
% \end{alltt}
% 
\section*{\texttt{<stdio.h>} Library}
\begin{alltt}
FILE *f; - \textrm{external file handle}

int getchar(void);
int getc(FILE *f);
   \textrm{Fetch one character from input}

int putchar(int ch);
int putc(int ch, FILE *f);
   \textrm{Display a character on output}

char *gets(char *s);
char *fgets(char *s, int len, FILE *f);
   \textrm{Fetch one line into buffer;}
   gets \textrm{discards the} \verb|'\n'|\textrm{;} fgets \textrm{keeps it}
   \textrm{both return pointer to buffer or} NULL \textrm{if} EOF

int puts(char *s);
int fputs(char *s, FILE *f);
   \textrm{Display buffer as line of text on output;}
   puts \textrm{appends} \verb|'\n'| \textrm{on output;} fputs \textrm{doesn't}

scanf(\emph{fmt-string}, \emph{address\(\sb{1}\)}, ...);
fscanf(FILE *f, \emph{fmt-string}, \emph{address\(\sb{1}\)}, ...);
   \textrm{Interpret input according to formats;}
   \textrm{store results in specified addresses;}
   \textrm{return number of valid, stored values}
\textrm{e.g.}
   scanf("%d", &i); - \textrm{read an integer}
   scanf("%f", &r); - \textrm{read a real number}
   scanf("%c", &c); - \textrm{read a character}
   scanf("%s", s);  - \textrm{read a string}

printf(\emph{fmt-string}, \emph{expr\(\sb{1}\)}, ...);
fprintf(FILE *f, \emph{fmt-string}, \emph{expr\(\sb{1}\)}, ...);
   \textrm{Display expressions according to specified}
   \textrm{formats; expression type must match format}
\textrm{e.g.}
   printf("%3d", i);  - \textrm{show an integer}
   printf("%.2f", r); - \textrm{show a real number}
   printf("%c", c);   - \textrm{show a character}
   printf("%s", s);    - \textrm{show a string}
   printf("Hello\(\backslash\)n");  - \textrm{show literal text}
\end{alltt}

\section*{\texttt{<stdlib.h>} Library}
\begin{alltt}
EOF  - \textrm{end of file marker for i/o}
NULL - \textrm{canonical invalid pointer}

void *malloc(int size);
   \textrm{allocate memory and return address}
void free(void *addr);
   \textrm{release previously allocated memory}
int rand(void);
   \textrm{generate a random} int \textrm{value}
void exit(int status);
   \textrm{terminate program with return status}
\end{alltt}

\section*{\texttt{<ctype.h>} Library}
\begin{alltt}
Bool isalpha(char), isdigit(char), isspace(char),
     ispunct(char), islower(char), isupper(char);
   \textrm{Character classification functions}

char toupper(char), tolower(char);
   \textrm{Case conversion functions}
\end{alltt}

\section*{\texttt{<math.h>} Library}
\begin{alltt}
double sin(double), cos(double), tan(double);
   \textrm{Trigonometric functions - inputs are radians}

double log(double x);
   \textrm{Compute the natural logarithm of} x

double sqrt(double);
   \textrm{Compute the square root of} x

double pow(double x, double y);
   \textrm{Computes} x {raised to the power} y
\end{alltt}

\section*{\texttt{<string.h>} Library}
\begin{alltt}
char *strcpy(char *buff, char *src);
char *strncpy(char *buff, char *src,
                            int len);
   \textrm{Copy} src \textrm{string int} buffer;
   strncpy \textrm{stops copying after first} len \textrm{chars}

char *strcat(char *dest, char *src);
char *strncat(char *dest, char *src,
                            int len);
   \textrm{Concatenate} src \textrm{onto text in} buffer
   strncat \textrm{stops copying after first} len \textrm{chars}

int strcmp(char *s1, char *s2);
int strncmp(char *s1, char *s2, int len);
int strcasecmp(char *s1, char *s2);
int strncasecmp(char *s1, char *s2);
   \textrm{Compare two strings; return difference}
   strncmp \textrm{stops comparing after first} len \textrm{chars}
   strcasecmp \textrm{ignores case of alphabetic letters in comparison}

int strlen(char *s)
   \textrm{Return length of string} s

char *strchr(char *s, int c);
   \textrm{Find leftmost occurence of char} c \textrm{in string} s\textrm{;}
   \textrm{returns pointer to} c \textrm{or} NULL \textrm{if not found}

char *strrchr(char *s, int c);
   \textrm{Find rightmost occurence of} c \textrm{in string} s\textrm{;}
   \textrm{returns pointer to} c \textrm{or} NULL \textrm{if not found}

char *strstr(char *str, char *pat);
   \textrm{Look for occurence of} pat \textrm{in string} str\textrm{;}
   \textrm{returns a pointer or} NULL \textrm{if} pat \textrm{not found}
\end{alltt}


\end{document}

