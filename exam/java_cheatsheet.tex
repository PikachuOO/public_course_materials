\documentclass[twocolumn,12pt]{article}
%\usepackage{twocolumn}
\usepackage{aa4-most}
\usepackage{alltt}
\begin{document}
\noindent

\noindent
\textbf{\huge Java Information}

\section*{Applications}
\begin{alltt}
//
// \emph{Program name and purpose}
//
import \emph{package}; ...

public class \emph{ApplicationName}
\{
   public static int main(String args[])
   \{
      \emph{object definitions and code}
   \}
\}
\end{alltt}
\section*{Classes}
\begin{alltt}
package \emph{package-for-this-class};

import \emph{package}; ...

public class \emph{ClassName}
\{
   // Fields (members)
   // \emph{data associated with each object} 
   // \emph{except} static \emph{means class variable}

   // Constructor methods \emph{e.g.}
   public \emph{ClassName}(\emph{zero or more args})
   \{
      \emph{code to initialise object}
   \}

   // Other methods
   // \emph{operations for objects in class}
   // \emph{except} static \emph{means class method}
\}
\end{alltt}
\section*{Data Types}
\begin{alltt}
\textrm{Primitive data types:}

int, long     - \textrm{integer numbers}
short, byte     (\textrm{e.g.} 1, -1, -27, 1000)
float, double - \textrm{real numbers}
                (e.g. 1.0, 1e23, -0.005)
char          - \textrm{characters}
                (\textrm{e.g.} 'a', 'b', '2', '!')
boolean       - \textrm{truth values}
                (\textrm{e.g.} true, false)

\textrm{Structured data types:}

\emph{type}[]      - \textrm{array of} \emph{type} \textrm{values}
\emph{Class}       - \textrm{reference to object in} \emph{Class}
\end{alltt}
\section*{Variables/Objects}
\begin{alltt}
int x, y;
   \textrm{two integer variables}

final static char EXCLAIM = '!';
   \textrm{character constant}

\emph{Class} r = new \emph{Class}();
   r \textrm{is a reference to a new} \emph{Class} \textrm{Object}

\emph{type}[] a = new \emph{type}[\emph{N}];
   \textrm{Array} a \textrm{of} \emph{N} \textrm{objects of type} \emph{type}
   \textrm{Indexed via} 0 .. \emph{N}-1

\textrm{Unitialised variables have zero or null value.}
\end{alltt}

\section*{Expressions}
\begin{alltt}
\textrm{Arithmetic operators:}
   +   -   *   /   % \textrm{\footnotesize(modulus)}

\textrm{Relational operators:}
   ==   !=   <   >   <=   >=

\textrm{Logical operators:}
   ! \textrm{\tiny(NOT)}   && \textrm{\tiny(AND)}   || \textrm{\tiny(OR)}

\textrm{Bitwise operators:}
   ~ \textrm{\tiny(NOT)}   & \textrm{\tiny(AND)}   | \textrm{\tiny(OR)}   ^ \textrm{\tiny(XOR)}

\textrm{String concatentation:}
   + \textrm{\footnotesize(convert operands to string and join)}
   \textrm{e.g.} "Here is " + 3.141 + " and " + (i+1)

\textrm{Object comparison:}
   \emph{obj}\(\sb{1}\) == \emph{obj}\(\sb{2}\)
      \textrm{Do} \emph{obj}\(\sb{1}\) \textrm{and} \emph{obj}\(\sb{2}\) \textrm{refer to same object?}
   \emph{obj}\(\sb{1}\).equals(\emph{obj}\(\sb{2}\))
      \textrm{Do} \emph{obj}\(\sb{1}\) \textrm{and} \emph{obj}\(\sb{2}\) \textrm{contain same values?}
\end{alltt}

\section*{Statements}
\begin{alltt}
\textrm{Assignment:}

\emph{var} = \emph{expression};
x++;  ++x;  x+=\emph{expr};  x-=\emph{expr}; ...

\textrm{Selection:}

if (\emph{condition\(\sb{1}\)}) \{ \emph{statements\(\sb{1}\)}; \}
else if (\emph{condition\(\sb{2}\)}) \{ \emph{statements\(\sb{2}\)}; \}
...
else if (\emph{condition\(\sb{n}\)}) \{ \emph{statements\(\sb{n}\)}; \}
else \{ \emph{statements\(\sb{n+1}\)}; \}

switch (\emph{expression}) \{
case \emph{const\(\sb{1}\)}: \emph{statements\(\sb{1}\)}; break;
...
case \emph{const\(\sb{n}\)}: \emph{statements\(\sb{n}\)}; break;
default: \emph{statements\(\sb{n+1}\)};
\}

\textrm{Iteration:}

while (\emph{condition}) \{ \emph{statements}; \}

for (\emph{init};\emph{condition};\emph{next}) \{ \emph{statements}; \}

break - \textrm{exit the loop}
continue - \textrm{go to next iteration}

\textrm{Exceptions:}

try \{ \emph{statements\(\sb{1}\)}; \}
catch (\emph{Exception\(\sb{1}\)A} \emph{e\(\sb{1}\)})
   \{ \emph{statements\(\sb{e\sb{1}}\)}; \}
catch (\emph{Exception\(\sb{2}\)A} \emph{e\(\sb{2}\)})
   \{ \emph{statements\(\sb{e\sb{2}}\)}; \}
...
finally \{ \emph{statements\(\sb{final}\)}; \}
\end{alltt}

\section*{Class \texttt{java.lang.System}}
\begin{alltt}
PrintStream out;  // \emph{standard output}
PrintStream err;  // \emph{standard error}
InputStream in;   // \emph{standard input}

void exit(int status)
   \textrm{terminate Java process and return} status
\end{alltt}

\section*{Class \texttt{java.io.PrintStream}}
\begin{alltt}
void print(\emph{object})
void println(\emph{object})
   \textrm{Prints \emph{object} on} stdout\textrm{, usually a} String
   println \textrm{also appends a newline character}
\end{alltt}

\section*{Class \texttt{java.io.BufferedReader}}
\begin{alltt}
String readLine()
   \textrm{read next line of text from stream}

\textrm{Reading from \texttt{stdin}:}
BufferedReader stdin = new BufferedReader
      (new InputStreamReader(System.in));
\end{alltt}

\section*{Class \texttt{java.lang.Math}}
\begin{alltt}
\emph{num} = \textrm{one of} int, long, float, double

\emph{num} abs(\emph{num})
   \textrm{absolute value of number}
double cos(double a)
double sin(double a)
double tan(double a)
   \textrm{cosine,sine,tangent of angle} a
\emph{num} max(\emph{num} x, \emph{num} y)
   \textrm{larger of two numbers}
\emph{num} min(\emph{num} x, \emph{num} y)
   \textrm{smaller of two numbers}
double pow(double x, double y)
   \textrm{return \(x\sp{y}\)}
double random()
   \textrm{return a random floating value}
int round(float x)
long round(double x)
   \textrm{round to nearest integer value}
double sqrt(double a)
   \textrm{compute square root of} a
\end{alltt}

\section*{Class \texttt{java.util.BitSet}}
\begin{alltt}
BitSet(int nbits)
   \textrm{create} BitSet \textrm{of specified size}
void and(BitSet bits)
   \textrm{AND two} BitStrings \textrm{together}
void or(BitSet bits)
   \textrm{OR two} BitStrings \textrm{together}
void clear(int n)
   \textrm{set bit} n \textrm{to have value of zero}
boolean get(int n)
   \textrm{read value contained at bit} n
void set(int n)
   \textrm{set bit} n \textrm{to have value of one}
\end{alltt}

\section*{Class \texttt{java.util.Date}}
\begin{alltt}
boolean after(Date when)
boolean before(Date when)
boolean equals(Date when)
   Date \textrm{comparison methods}
long getTime()
   \textrm{get current time as big number}
\end{alltt}

\section*{Class \texttt{java.util.HashTable}}
\begin{alltt}
boolean contains(Object elem)
   \textrm{check whether} elem \textrm{is in table}
boolean containsKey(Object key)
   \textrm{check whether} key \textrm{is in table}
Enumeration elements()
   \textrm{return enumeration of all elements}
Object get(Object key)
   \textrm{return} Object \textrm{associated with} key
boolean isEmpty()
   \textrm{check whether table has any keys}
Enumeration keys()
   \textrm{return enumeration of key values}
void put(Object key, Object value)
   \textrm{insert} (key,value) \textrm{into table}
Object remove(Object key)
   \textrm{delete element with} key \textrm{from table}
int size()
   \textrm{return number of elements in table}
\end{alltt}

\section*{Class \texttt{java.util.Stack}}
\begin{alltt}
boolean empty()
   \textrm{inidcate whether} Stack \textrm{has any items}
Object peek()
   \textrm{return item at top of} Stack
Object pop()
   \textrm{remove item from top of} Stack
Object push(Object item)
   \textrm{place} item \textrm{on top of} Stack
\end{alltt}

\section*{Class \texttt{java.util.Vector}}
\begin{alltt}
void addElement(Object elem)
   \textrm{appends new element to} Vector
Object elementAt(int index)
   \textrm{obtains element at position} index
void copyInto(Object array[])
   \textrm{copy elements of} Vector \textrm{into} array
Enumeration elements()
   \textrm{return enumeration of all elements}
Object firstElement()
   \textrm{return first element of} Vector
int indexOf(Object elem)
   \textrm{return location of} elem \textrm{in} Vector
void insertElementAt(int index)
   \textrm{insert new element at position} index
boolean isEmpty()
   \textrm{check whether} Vector \textrm{has any elements}
boolean removeElement(Object elem)
   \textrm{remove given} Object \textrm{from} Vector
boolean removeElementAt(int index)
   \textrm{remove object at position} index
void setElementAt(Object elem, int index)
   \textrm{replace element at} index \textrm{by} elem
int size()
   \textrm{return number of elements in} Vector
\end{alltt}

\end{document}

